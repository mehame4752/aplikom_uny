\documentclass[a4paper,10pt]{article}
\usepackage{eumat}

\begin{document}
\begin{eulernotebook}
\begin{eulercomment}
23030630053\\
Hamemayu Hayuningrat
\end{eulercomment}
\eulerheading{EMT untuk Statistika}
\begin{eulercomment}
Dalam notebook ini, kami akan mendemonstrasikan plot utama statistika,
uji dan distribusi dalam Euler.

Mari kita mulai dengan beberapa deskriptif statistik. Ini bukanlah
pengantar untuk statistik. Jadi kamu mungkin membutuhkan beberapa
latar belakang untuk memahami secara detail.

Asumsikan ukurn mengikuti berikut. Kita berharap untuk menghitung
nilai rata-rata dan menghitung standar deviasi.
\end{eulercomment}
\begin{eulerprompt}
>M=[1000,1004,998,997,1002,1001,998,1004,998,997]; ...
>median(M), mean(M), dev(M),
\end{eulerprompt}
\begin{euleroutput}
  999
  999.9
  2.72641400622
\end{euleroutput}
\begin{eulercomment}
Kami dapat membuat plot box-and-whiskers untuk data. Dalam kasus kami dimana tidak ada
data yang keluar.
\end{eulercomment}
\begin{eulerprompt}
>aspect(1.75); boxplot(M):
\end{eulerprompt}
\eulerimg{15}{images/Hamemayu Hayuningrat_23030630053_EMTStatistika-001.png}
\begin{eulercomment}
Kami menghitung kemungkinan bahwa sebuah nilai lebih dari 1005, asumsikan hitungan nilai
dari sebuah distribusi normal.

Semua fungsi untuk distibusi-distribusi dalam Euler dan dengan ...dis dan menghitung
cumulative probability distribution (CPF)

\end{eulercomment}
\begin{eulerformula}
\[
\text{normaldis(x,m,d)}=\int_{-\infty}^x \frac{1}{d\sqrt{2\pi}}e^{-\frac{1}{2}(\frac{t-m}{d})^2}\ dt.
\]
\end{eulerformula}
\begin{eulercomment}
Kami mencetak hasil dalam \% dengan akurasi 2 digit menggunakan fungsi print.
\end{eulercomment}
\begin{eulerprompt}
>print((1-normaldis(1005,mean(M),dev(M)))*100,2,unit=" %")
\end{eulerprompt}
\begin{euleroutput}
        3.07 %
\end{euleroutput}
\begin{eulercomment}
Untuk contoh selanjutnya, kami asumsi angka-angka berikut dari pria dengan diberikan
sebuah rentang ukuran.
\end{eulercomment}
\begin{eulerprompt}
>r=155.5:4:187.5; v=[22,71,136,169,139,71,32,8];
\end{eulerprompt}
\begin{eulercomment}
Ini merupakan sebuah plot distribusi.
\end{eulercomment}
\begin{eulerprompt}
>plot2d(r,v,a=150,b=200,c=0,d=190,bar=1,style="\(\backslash\)/"):
\end{eulerprompt}
\eulerimg{15}{images/Hamemayu Hayuningrat_23030630053_EMTStatistika-002.png}
\begin{eulercomment}
Kami dapat mengambil sebarang data mentah ke sebuah tabel.

Tabels merupakan sebuah method untuk menyimpan data statistik. Tabel kami seharusnya
mengandung tiga kolom: rentang mulai, rentang akhir, banyaknya pria dalam rentang.

Tables dapat juga dicetak dengan headers. Kami menggunakan vektor string untuk mengatur
headers.
\end{eulercomment}
\begin{eulerprompt}
>T:=r[1:8]' | r[2:9]' | v'; writetable(T,labc=["BB","BA","Frek"])
\end{eulerprompt}
\begin{euleroutput}
          BB        BA      Frek
       155.5     159.5        22
       159.5     163.5        71
       163.5     167.5       136
       167.5     171.5       169
       171.5     175.5       139
       175.5     179.5        71
       179.5     183.5        32
       183.5     187.5         8
\end{euleroutput}
\begin{eulercomment}
Jika kami menggunakan nilai rata-rata dan statistik lain dari ukuran, kami membutuhkan
untuk menghitung titik tengah dari rentang. Kami dapat menggunakan kolom kedua pertama
dari tabel kami untuk ini.

Simbol "\textbar{}" digunakan untuk memisahkan kolom, fungsi "writetable" digunakan untuk menulis
tabel, dengan pilihan "labc" adalah secara spesifik kolom headers.
\end{eulercomment}
\begin{eulerprompt}
>(T[,1]+T[,2])/2 // the midpoint of each interval
\end{eulerprompt}
\begin{euleroutput}
          157.5 
          161.5 
          165.5 
          169.5 
          173.5 
          177.5 
          181.5 
          185.5 
\end{euleroutput}
\begin{eulercomment}
Tetapi ini lebih mudah, untuk melipat rentang dengan vektor [1/2, 1/2].
\end{eulercomment}
\begin{eulerprompt}
>M=fold(r,[0.5,0.5])
\end{eulerprompt}
\begin{euleroutput}
  [157.5,  161.5,  165.5,  169.5,  173.5,  177.5,  181.5,  185.5]
\end{euleroutput}
\begin{eulercomment}
Sekarang kita menghitung rata-rata dan deviasi dari contoh dengan frekuensi yang
diberikan.
\end{eulercomment}
\begin{eulerprompt}
>\{m,d\}=meandev(M,v); m, d,
\end{eulerprompt}
\begin{euleroutput}
  169.901234568
  5.98912964449
\end{euleroutput}
\begin{eulercomment}
Mari kita tambahkan distribusi normal dari nilai nilai menjadi plot batang diatas. Formula
untuk distribusi normal dengan rata-rata m dan standar deviasi d adalah:

\end{eulercomment}
\begin{eulerformula}
\[
y=\frac{1}{d\sqrt{2\pi}}e^{\frac{-(x-m)^2}{2d^2}}.
\]
\end{eulerformula}
\begin{eulercomment}
Karena ini nilai-nilai diantara 0 dan 1, untuk membuat plot itu dalam plot batang kita
harus mengalikan dengan 4 kali jumlahan angka dari data.
\end{eulercomment}
\begin{eulerprompt}
>plot2d("qnormal(x,m,d)*sum(v)*4", ...
>  xmin=min(r),xmax=max(r),thickness=3,add=1):
\end{eulerprompt}
\eulerimg{15}{images/Hamemayu Hayuningrat_23030630053_EMTStatistika-003.png}
\eulerheading{Tables}
\begin{eulercomment}
Dalam direktori dari notebook ini kamu menemukan sebuah file dengan sebuah tabel. Data
tersebut menunjukkan hasil dari sebuah survey. Disini merupakan empat baris pertama dari
file. Data berasal dari sebuah buku online Jerman "Einführung in die Statistik mit R" oleh
A. Handl.
\end{eulercomment}
\begin{eulerprompt}
>printfile("table.dat",4);
\end{eulerprompt}
\begin{euleroutput}
  Person Sex Age Titanic Evaluation Tip Problem
  1 m 30 n . 1.80 n
  2 f 23 y g 1.80 n
  3 f 26 y g 1.80 y
\end{euleroutput}
\begin{eulercomment}
Tabel berisi 7 kolom dari angka-angka atau token (string). Kami ingin membaca tabel dari
file. Pertama, kami menggunakan translasi sendiri untuk tokens.

Untuk ini, kami mendefinisikan himpunan dari token. Fungsi strtokens() mengambil sebuah
vektor string dari token dari sebuah string yang diberikan.
\end{eulercomment}
\begin{eulerprompt}
>mf:=["m","f"]; yn:=["y","n"]; ev:=strtokens("g vg m b vb");
\end{eulerprompt}
\begin{eulercomment}
Sekarang kita membaca tabel dengan translasi berikut.

Argumen-argumen tok2, tok4, dll merupakan translasi dari kolom tabel. Argumen-argumen ini
tidak dalam daftar parameter dari readtable(), jadi kamu butuh untuk menyediakan mereka
dengan ":=".
\end{eulercomment}
\begin{eulerprompt}
>\{MT,hd\}=readtable("table.dat",tok2:=mf,tok4:=yn,tok5:=ev,tok7:=yn);
>load over statistics;
\end{eulerprompt}
\begin{eulercomment}
Untuk mencetak, kami membutuhkan secara spesifik himpunan token yang sama. Kami mencetak
empat baris pertama saja.
\end{eulercomment}
\begin{eulerprompt}
>writetable(MT[1:10],labc=hd,wc=5,tok2:=mf,tok4:=yn,tok5:=ev,tok7:=yn);
\end{eulerprompt}
\begin{euleroutput}
   Person  Sex  Age Titanic Evaluation  Tip Problem
        1    m   30       n          .  1.8       n
        2    f   23       y          g  1.8       n
        3    f   26       y          g  1.8       y
        4    m   33       n          .  2.8       n
        5    m   37       n          .  1.8       n
        6    m   28       y          g  2.8       y
        7    f   31       y         vg  2.8       n
        8    m   23       n          .  0.8       n
        9    f   24       y         vg  1.8       y
       10    m   26       n          .  1.8       n
\end{euleroutput}
\begin{eulercomment}
Titik "." mewakili nilai, yang mana tidak tersedia.

Jika kita tidak ingin untuk secara spesifik token untuk translasi lebih lanjut, kami hanya
butuh untuk secara spesifik, yang mana kolom berisi token dan bukan angka-angka.
\end{eulercomment}
\begin{eulerprompt}
>ctok=[2,4,5,7]; \{MT,hd,tok\}=readtable("table.dat",ctok=ctok);
\end{eulerprompt}
\begin{eulercomment}
Sekarang fungsi readtable() mengembalikan sebuah himpunan dari token.
\end{eulercomment}
\begin{eulerprompt}
>tok
\end{eulerprompt}
\begin{euleroutput}
  m
  n
  f
  y
  g
  vg
\end{euleroutput}
\begin{eulercomment}
Tabel berisi isian dari file dengan tokens yang diubah ke angka.

String spesial NA="." diintrepretasikan sebagai "Not Available" dan menghasilkan NAN (Not
A Number) dalam tabel. Translasi ini dapat diubah dengan parameter-parameter NA dan NAval.
\end{eulercomment}
\begin{eulerprompt}
>MT[1]
\end{eulerprompt}
\begin{euleroutput}
  [1,  1,  30,  2,  NAN,  1.8,  2]
\end{euleroutput}
\begin{eulercomment}
Disini merupakan isi tabel dengan angka-angka yang tidak diubah.
\end{eulercomment}
\begin{eulerprompt}
>writetable(MT,wc=5)
\end{eulerprompt}
\begin{euleroutput}
      1    1   30    2    .  1.8    2
      2    3   23    4    5  1.8    2
      3    3   26    4    5  1.8    4
      4    1   33    2    .  2.8    2
      5    1   37    2    .  1.8    2
      6    1   28    4    5  2.8    4
      7    3   31    4    6  2.8    2
      8    1   23    2    .  0.8    2
      9    3   24    4    6  1.8    4
     10    1   26    2    .  1.8    2
     11    3   23    4    6  1.8    4
     12    1   32    4    5  1.8    2
     13    1   29    4    6  1.8    4
     14    3   25    4    5  1.8    4
     15    3   31    4    5  0.8    2
     16    1   26    4    5  2.8    2
     17    1   37    2    .  3.8    2
     18    1   38    4    5    .    2
     19    3   29    2    .  3.8    2
     20    3   28    4    6  1.8    2
     21    3   28    4    1  2.8    4
     22    3   28    4    6  1.8    4
     23    3   38    4    5  2.8    2
     24    3   27    4    1  1.8    4
     25    1   27    2    .  2.8    4
\end{euleroutput}
\begin{eulercomment}
Untuk lebih jelas, kamu dapat mengambil keluaran dari readtable() ke sebuah list.
\end{eulercomment}
\begin{eulerprompt}
>Table=\{\{readtable("table.dat",ctok=ctok)\}\};
\end{eulerprompt}
\begin{eulercomment}
Menggunakan kolom-kolom token yang sama dan token token membaca dari file, kami dapat
mencetak tabel. Kami dapat juga secara spesifik ctok, tok, dll. atau menggunakan list
Table.
\end{eulercomment}
\begin{eulerprompt}
>writetable(Table,ctok=ctok,wc=5);
\end{eulerprompt}
\begin{euleroutput}
   Person  Sex  Age Titanic Evaluation  Tip Problem
        1    m   30       n          .  1.8       n
        2    f   23       y          g  1.8       n
        3    f   26       y          g  1.8       y
        4    m   33       n          .  2.8       n
        5    m   37       n          .  1.8       n
        6    m   28       y          g  2.8       y
        7    f   31       y         vg  2.8       n
        8    m   23       n          .  0.8       n
        9    f   24       y         vg  1.8       y
       10    m   26       n          .  1.8       n
       11    f   23       y         vg  1.8       y
       12    m   32       y          g  1.8       n
       13    m   29       y         vg  1.8       y
       14    f   25       y          g  1.8       y
       15    f   31       y          g  0.8       n
       16    m   26       y          g  2.8       n
       17    m   37       n          .  3.8       n
       18    m   38       y          g    .       n
       19    f   29       n          .  3.8       n
       20    f   28       y         vg  1.8       n
       21    f   28       y          m  2.8       y
       22    f   28       y         vg  1.8       y
       23    f   38       y          g  2.8       n
       24    f   27       y          m  1.8       y
       25    m   27       n          .  2.8       y
\end{euleroutput}
\begin{eulercomment}
Fungsi tablecol() mengembalikan nilai-nilai dari kolom tabel. Melewati sebarang baris
dengan nilai-nilai("." dalam file) NAN, dan indeks-indeks dari kolom yang mana berisi
nilai-nilai ini.
\end{eulercomment}
\begin{eulerprompt}
>\{c,i\}=tablecol(MT,[5,6]);
\end{eulerprompt}
\begin{eulercomment}
Kami dapat menggunakan ini untuk mengekstrasi kolom-kolom dari tabel utuk sebuah tabel
baru.
\end{eulercomment}
\begin{eulerprompt}
>j=[1,5,6]; writetable(MT[i,j],labc=hd[j],ctok=[2],tok=tok)
\end{eulerprompt}
\begin{euleroutput}
      Person Evaluation       Tip
           2          g       1.8
           3          g       1.8
           6          g       2.8
           7         vg       2.8
           9         vg       1.8
          11         vg       1.8
          12          g       1.8
          13         vg       1.8
          14          g       1.8
          15          g       0.8
          16          g       2.8
          20         vg       1.8
          21          m       2.8
          22         vg       1.8
          23          g       2.8
          24          m       1.8
\end{euleroutput}
\begin{eulercomment}
Tentu saja, kami butuh untuk mengekstrasi tabel itu sendiri dari Table list dalam kasus
ini.
\end{eulercomment}
\begin{eulerprompt}
>MT=Table[1];
\end{eulerprompt}
\begin{eulercomment}
Of course, we can also use it to determine the mean value of a column or any other statistical
value.
\end{eulercomment}
\begin{eulerprompt}
>mean(tablecol(MT,6))
\end{eulerprompt}
\begin{euleroutput}
  2.175
\end{euleroutput}
\begin{eulercomment}
Fungsi getstatistics() mengembalikan elemen-elemen dalam sebuah vektor dan hitungannya.
Kami menerapkannya ke nilai "m" dan "f" dalam kolom kedua dari tabel kami
\end{eulercomment}
\begin{eulerprompt}
>\{xu,count\}=getstatistics(tablecol(MT,2)); xu, count,
\end{eulerprompt}
\begin{euleroutput}
  [1,  3]
  [12,  13]
\end{euleroutput}
\begin{eulercomment}
Kami dapat mencetak hasil dalam sebuah tabel baru.
\end{eulercomment}
\begin{eulerprompt}
>writetable(count',labr=tok[xu])
\end{eulerprompt}
\begin{euleroutput}
           m        12
           f        13
\end{euleroutput}
\begin{eulercomment}
Fungsi selecttable() mengembalikan sebuah tabel baru dengan nilai-nilai dalam satu kolom
yang dipilih dari indeks vektor. Pertama kami lihat indeks-indeks dari dua dari nilai kami
dalam tabel token.
\end{eulercomment}
\begin{eulerprompt}
>v:=indexof(tok,["g","vg"])
\end{eulerprompt}
\begin{euleroutput}
  [5,  6]
\end{euleroutput}
\begin{eulercomment}
Sekarang kami dapat memilih baris-baris dari tabel, yang mana memiliki sebarang nilai dari
v dalam baris ke-5 mereka.
\end{eulercomment}
\begin{eulerprompt}
>MT1:=MT[selectrows(MT,5,v)]; i:=sortedrows(MT1,5);
\end{eulerprompt}
\begin{eulercomment}
Sekarang kami dapat mencetak tabel, dengan nilai-nilai yang diekstrasi dan diurutkan dalam
kolom ke-5.
\end{eulercomment}
\begin{eulerprompt}
>writetable(MT1[i],labc=hd,ctok=ctok,tok=tok,wc=7);
\end{eulerprompt}
\begin{euleroutput}
   Person    Sex    Age Titanic Evaluation    Tip Problem
        2      f     23       y          g    1.8       n
        3      f     26       y          g    1.8       y
        6      m     28       y          g    2.8       y
       18      m     38       y          g      .       n
       16      m     26       y          g    2.8       n
       15      f     31       y          g    0.8       n
       12      m     32       y          g    1.8       n
       23      f     38       y          g    2.8       n
       14      f     25       y          g    1.8       y
        9      f     24       y         vg    1.8       y
        7      f     31       y         vg    2.8       n
       20      f     28       y         vg    1.8       n
       22      f     28       y         vg    1.8       y
       13      m     29       y         vg    1.8       y
       11      f     23       y         vg    1.8       y
\end{euleroutput}
\begin{eulercomment}
Untuk statistik selanjutnya, kami ingin utnuk menghubungkan dua kolom dari tabel. Jadi
kami mengekstrak kolom 2 dan 4 dan mengurutkan tabel.
\end{eulercomment}
\begin{eulerprompt}
>i=sortedrows(MT,[2,4]);  ...
>  writetable(tablecol(MT[i],[2,4])',ctok=[1,2],tok=tok)
\end{eulerprompt}
\begin{euleroutput}
           m         n
           m         n
           m         n
           m         n
           m         n
           m         n
           m         n
           m         y
           m         y
           m         y
           m         y
           m         y
           f         n
           f         y
           f         y
           f         y
           f         y
           f         y
           f         y
           f         y
           f         y
           f         y
           f         y
           f         y
           f         y
\end{euleroutput}
\begin{eulercomment}
Dengan getstatistics(), kami dapat juga menghubungkan hitungan dalam dua kolom dari tabel
pada tiap-tiap yang lain.
\end{eulercomment}
\begin{eulerprompt}
>MT24=tablecol(MT,[2,4]); ...
>\{xu1,xu2,count\}=getstatistics(MT24[1],MT24[2]); ...
>writetable(count,labr=tok[xu1],labc=tok[xu2])
\end{eulerprompt}
\begin{euleroutput}
                     n         y
           m         7         5
           f         1        12
\end{euleroutput}
\begin{eulercomment}
Sebuah tabel dapat juga ditulis ke sebuah file.
\end{eulercomment}
\begin{eulerprompt}
>filename="test.dat"; ...
>writetable(count,labr=tok[xu1],labc=tok[xu2],file=filename);
\end{eulerprompt}
\begin{eulercomment}
Lalu kami dapat membaca tabel dari file.
\end{eulercomment}
\begin{eulerprompt}
>\{MT2,hd,tok2,hdr\}=readtable(filename,>clabs,>rlabs); ...
>writetable(MT2,labr=hdr,labc=hd)
\end{eulerprompt}
\begin{euleroutput}
                     n         y
           m         7         5
           f         1        12
\end{euleroutput}
\begin{eulercomment}
Dan menghapus file.
\end{eulercomment}
\begin{eulerprompt}
>fileremove(filename);
\end{eulerprompt}
\eulerheading{Distribusi-Distribusi}
\begin{eulercomment}
Dengan plot2d, ini merupakan method paling mudah untuk membuat plot sebuah distribusi dari
data eksperimental.
\end{eulercomment}
\begin{eulerprompt}
>p=normal(1,1000); //1000 random normal-distributed sample p
>plot2d(p,distribution=20,style="\(\backslash\)/"); // plot the random sample p
>plot2d("qnormal(x,0,1)",add=1): // add the standard normal distribution plot
\end{eulerprompt}
\eulerimg{15}{images/Hamemayu Hayuningrat_23030630053_EMTStatistika-004.png}
\begin{eulercomment}
Tolong catat perbedaan antara plot batang (contoh) dan kurva normal (distribusi real).
Enter kembali tiga perintah untuk melihat hasil lainnya.
\end{eulercomment}
\begin{eulercomment}
Ini merupakan sebuah perbandingan dari 10 simulasi dari 1000 nilai berdistribusi normal
menggunakan sebuah yang dipanggil box plot. Plot ini menunjukkan median, quartil 25\% dan
75\%, nilai minimal dan maksimal dan percilan.
\end{eulercomment}
\begin{eulerprompt}
>p=normal(10,1000); boxplot(p):
\end{eulerprompt}
\eulerimg{15}{images/Hamemayu Hayuningrat_23030630053_EMTStatistika-005.png}
\begin{eulercomment}
Untuk menghasilkan bilangan acak, Euler memiliki intrandom. Mari kita simulasikan lemparan
dadu dan plot distribusinya.

Kami menggunakan fungsi getmultiplicities(v,x), yang mana menghitung seberapa sering
elemen dari v muncul dalam x. Lantas kami membuat plot dari hasilnya menggunakan
columnsplot().
\end{eulercomment}
\begin{eulerprompt}
>k=intrandom(1,6000,6);  ...
>columnsplot(getmultiplicities(1:6,k));  ...
>ygrid(1000,color=red):
\end{eulerprompt}
\eulerimg{15}{images/Hamemayu Hayuningrat_23030630053_EMTStatistika-006.png}
\begin{eulercomment}
Ketika intrandom(n,m,k) mengembalikan bilangan bulat secara seragam berdistribusi dari 1
ke k, ini memungkinkan untuk menggunakan sebarang lainnya diberikan distribusi dari
bilangan bulat dengan randpint().

Contoh berikut, peluang untuk 1,2,3 adalah 0.4,0.1,0.5 secara berurutan.
\end{eulercomment}
\begin{eulerprompt}
>randpint(1,1000,[0.4,0.1,0.5]); getmultiplicities(1:3,%)
\end{eulerprompt}
\begin{euleroutput}
  [378,  102,  520]
\end{euleroutput}
\begin{eulercomment}
Euler dapat menghasilkan nilai-nilai acak dari distibusi-distribusi lainnya. Lihatlah ke
referensi berikut.

Contoh, kami mencoba distribusi eksponensial. Sebuah variabel kontinu acak X dikatakan
memiliki distribusi eksponensial, jika PDF diberikan sebagai

\end{eulercomment}
\begin{eulerformula}
\[
f_X(x)=\lambda e^{-\lambda x},\quad x>0,\quad \lambda>0,
\]
\end{eulerformula}
\begin{eulercomment}
dengan parameter

\end{eulercomment}
\begin{eulerformula}
\[
\lambda=\frac{1}{\mu},\quad \mu \text{ is the mean, and denoted by } X \sim \text{Exponential}(\lambda).
\]
\end{eulerformula}
\begin{eulerprompt}
>plot2d(randexponential(1,1000,2),>distribution):
\end{eulerprompt}
\eulerimg{15}{images/Hamemayu Hayuningrat_23030630053_EMTStatistika-007.png}
\begin{eulercomment}
Untuk distribusi lainnya, Euler dapat menghitung fungsi distribusi dan sebaliknya.
\end{eulercomment}
\begin{eulerprompt}
>plot2d("normaldis",-4,4): 
\end{eulerprompt}
\eulerimg{15}{images/Hamemayu Hayuningrat_23030630053_EMTStatistika-008.png}
\begin{eulercomment}
Berikut merupakan sebuah cara untuk membuat plot sebuah quantile.
\end{eulercomment}
\begin{eulerprompt}
>plot2d("qnormal(x,1,1.5)",-4,6);  ...
>plot2d("qnormal(x,1,1.5)",a=2,b=5,>add,>filled):
\end{eulerprompt}
\eulerimg{15}{images/Hamemayu Hayuningrat_23030630053_EMTStatistika-009.png}
\begin{eulerformula}
\[
\text{normaldis(x,m,d)}=\int_{-\infty}^x \frac{1}{d\sqrt{2\pi}}e^{-\frac{1}{2}(\frac{t-m}{d})^2}\ dt.
\]
\end{eulerformula}
\begin{eulercomment}
Peluang dalam luasan hijau seperti berikut.
\end{eulercomment}
\begin{eulerprompt}
>normaldis(5,1,1.5)-normaldis(2,1,1.5)
\end{eulerprompt}
\begin{euleroutput}
  0.248662156979
\end{euleroutput}
\begin{eulercomment}
Ini dapat dihitung secara numerik dengan integral berikut.

\end{eulercomment}
\begin{eulerformula}
\[
\int_2^5 \frac{1}{1.5\sqrt{2\pi}}e^{-\frac{1}{2}(\frac{x-1}{1.5})^2}\ dx.
\]
\end{eulerformula}
\begin{eulerprompt}
>gauss("qnormal(x,1,1.5)",2,5)
\end{eulerprompt}
\begin{euleroutput}
  0.248662156979
\end{euleroutput}
\begin{eulercomment}
Mari kita bandingkan distribusi binomial dengan distribusi normal dari rata-rata yang sama
dan deviasi. Fungsi invbindis() menyelesaikan sebuah interpolasi linear antara nilai-nilai
bilangan bulat.
\end{eulercomment}
\begin{eulerprompt}
>invbindis(0.95,1000,0.5), invnormaldis(0.95,500,0.5*sqrt(1000))
\end{eulerprompt}
\begin{euleroutput}
  525.516721219
  526.007419394
\end{euleroutput}
\begin{eulercomment}
Fungsi qdis() adalah sebuah kepadatan dari distribusi chi-square. Seperti biasa, Euler
memetakan vektor-vektor ke fungsi ini. Lantas kami mendapatkan sebuah plot dari semua
distribusi chi-square dengan derajat 5 ke 30 secara mudah mengikuti langkah berikut.
\end{eulercomment}
\begin{eulerprompt}
>plot2d("qchidis(x,(5:5:50)')",0,50):
\end{eulerprompt}
\eulerimg{15}{images/Hamemayu Hayuningrat_23030630053_EMTStatistika-010.png}
\begin{eulercomment}
Euler memiliki fungsi akurasi untuk mengevaluasi distribusi. Mari kita periksa chidis()
dengan sebuah integral.

Penamaan mencoba untuk lebih konsisten. Contoh,

- distribusi chi-square adalah chidis(),\\
- fungsi invers adalah invchidis(),\\
- kepadatan adalah qchidis().

Komplemen dari distribusi (ekor atas) merupakan chicdis().
\end{eulercomment}
\begin{eulerprompt}
>chidis(1.5,2), integrate("qchidis(x,2)",0,1.5)
\end{eulerprompt}
\begin{euleroutput}
  0.527633447259
  0.527633447259
\end{euleroutput}
\eulerheading{Distribusi-Distribusi Diskret}
\begin{eulercomment}
Untuk mendefinisikan distribusi diskret milik anda, kamu dapat menggunakan method berikut.

Pertama kami mengatur fungsi distribusi.
\end{eulercomment}
\begin{eulerprompt}
>wd = 0|((1:6)+[-0.01,0.01,0,0,0,0])/6
\end{eulerprompt}
\begin{euleroutput}
  [0,  0.165,  0.335,  0.5,  0.666667,  0.833333,  1]
\end{euleroutput}
\begin{eulercomment}
Arti ini bahwa dengan peluang wd[i+1]-wd[i] kami menghasilkan nilai i acak.

Ini hampir sebuah distribusi seragam. Mari kita definisikan sebuah pencetak angka acak
untuk ini. Fungsi find(v,x) menemukan nilai x dalam vektor v. Ini bekerja untuk vektor x
juga.
\end{eulercomment}
\begin{eulerprompt}
>function wrongdice (n,m) := find(wd,random(n,m))
\end{eulerprompt}
\begin{eulercomment}
Error merupakan jelas bahwa kami melihatnya hanya dengan iterasi banyak sekali.
\end{eulercomment}
\begin{eulerprompt}
>columnsplot(getmultiplicities(1:6,wrongdice(1,1000000))):
\end{eulerprompt}
\eulerimg{15}{images/Hamemayu Hayuningrat_23030630053_EMTStatistika-011.png}
\begin{eulercomment}
Ini merupakan fungsi sederhana untuk memeriksa untuk distribusi seragam dari nilai-nilai
1...K dalam v. Kami menerima hasil ini, jika untuk semua frekuensi.

\end{eulercomment}
\begin{eulerformula}
\[
\left|f_i-\frac{1}{K}\right| < \frac{\delta}{\sqrt{n}}.
\]
\end{eulerformula}
\begin{eulerprompt}
>function checkrandom (v, delta=1) ...
\end{eulerprompt}
\begin{eulerudf}
    K=max(v); n=cols(v);
    fr=getfrequencies(v,1:K);
    return max(fr/n-1/K)<delta/sqrt(n);
    endfunction
\end{eulerudf}
\begin{eulercomment}
Tentusaja fungsi akan menolak distribusi seragam.
\end{eulercomment}
\begin{eulerprompt}
>checkrandom(wrongdice(1,1000000))
\end{eulerprompt}
\begin{euleroutput}
  0
\end{euleroutput}
\begin{eulercomment}
Dan ini menerima pencetak acak built-in.
\end{eulercomment}
\begin{eulerprompt}
>checkrandom(intrandom(1,1000000,6))
\end{eulerprompt}
\begin{euleroutput}
  1
\end{euleroutput}
\begin{eulercomment}
Kami dapat menghitung distribusi binomial. Pertama ini adalah binomialsum(), yang mana
mengembalikan peluang dari i kurang dari n percobaan.
\end{eulercomment}
\begin{eulerprompt}
>bindis(410,1000,0.4)
\end{eulerprompt}
\begin{euleroutput}
  0.751401349654
\end{euleroutput}
\begin{eulercomment}
Fungsi invers Beta digunakan untuk menghitung konfidens interval Clopper-Pearson untuk
parameter p. Level bawaan adalah alpha.

Arti dari interval ini adalah bahwa jika p berada diluar interval, hasil yang diamati dari
410 dalam 1000 adalah jarang.
\end{eulercomment}
\begin{eulerprompt}
>clopperpearson(410,1000)
\end{eulerprompt}
\begin{euleroutput}
  [0.37932,  0.441212]
\end{euleroutput}
\begin{eulercomment}
Perintah-perintah berikut merupakan cara langsung untuk mendapatkan hasil diatas. Tetapi
untuk n besar, penjumlahan langsung tidak akurat dan lambat.
\end{eulercomment}
\begin{eulerprompt}
>p=0.4; i=0:410; n=1000; sum(bin(n,i)*p^i*(1-p)^(n-i))
\end{eulerprompt}
\begin{euleroutput}
  0.751401349655
\end{euleroutput}
\begin{eulercomment}
Selain itu, invbinsum() menghitung invers dari binomialsum().
\end{eulercomment}
\begin{eulerprompt}
>invbindis(0.75,1000,0.4)
\end{eulerprompt}
\begin{euleroutput}
  409.932733047
\end{euleroutput}
\begin{eulercomment}
Dalam Bridge, kami asumsikan 5 kartu yang keluar (dari 52) dalam dua tangan (26 kartu),
kami mengasumsikan peluang dari sebuah distribusi yang buruk dari 3:2 (contohnya 0:5, 1:4,
4:1 atau 5:0).
\end{eulercomment}
\begin{eulerprompt}
>2*hypergeomsum(1,5,13,26)
\end{eulerprompt}
\begin{euleroutput}
  0.321739130435
\end{euleroutput}
\begin{eulercomment}
Ini juga sebuah simulasi dari distribusi multinomial.
\end{eulercomment}
\begin{eulerprompt}
>randmultinomial(10,1000,[0.4,0.1,0.5])
\end{eulerprompt}
\begin{euleroutput}
            381           100           519 
            376            91           533 
            417            80           503 
            440            94           466 
            406           112           482 
            408            94           498 
            395           107           498 
            399            96           505 
            428            87           485 
            400            99           501 
\end{euleroutput}
\eulerheading{Membuat Plot Data}
\begin{eulercomment}
Untuk memplot data, kami mencoba hasil dari pemilihan Jerman sejak 1990, dihitung dalam
tempat duduk.
\end{eulercomment}
\begin{eulerprompt}
>BW := [ ...
>1990,662,319,239,79,8,17; ...
>1994,672,294,252,47,49,30; ...
>1998,669,245,298,43,47,36; ...
>2002,603,248,251,47,55,2; ...
>2005,614,226,222,61,51,54; ...
>2009,622,239,146,93,68,76; ...
>2013,631,311,193,0,63,64];
\end{eulerprompt}
\begin{eulercomment}
Untuk partisi-partisi, kami gunakan sebuah string dari nama-nama.
\end{eulercomment}
\begin{eulerprompt}
>P:=["CDU/CSU","SPD","FDP","Gr","Li"];
\end{eulerprompt}
\begin{eulercomment}
Mari kita cetak presentase secara bagus.

Pertama kami ekstrak kolom-kolom yang dibutuhkan. Kolom 3 ke 7 merupakan tempat duduk dari
setiap partai, dan kolom 2 merupakan banyaknya tempat duduk. Kolom merupakan tahun
pemilihan.
\end{eulercomment}
\begin{eulerprompt}
>BT:=BW[,3:7]; BT:=BT/sum(BT); YT:=BW[,1]';
\end{eulerprompt}
\begin{eulercomment}
Lalu kami mencetak statistik dalam bentuk tabel. Kami menggunakan nama-nama seperti kolom
headers dan tahun sebagai headers untuk baris-baris. Lebar bawaan untuk kolom adalah
wc=10, tapi kami menyarankan output yang lebih padat. Kolom-kolom akan di ekspansi untuk
label-label dari kolom-kolom, jika dibutuhkan.
\end{eulercomment}
\begin{eulerprompt}
>writetable(BT*100,wc=6,dc=0,>fixed,labc=P,labr=YT)
\end{eulerprompt}
\begin{euleroutput}
         CDU/CSU   SPD   FDP    Gr    Li
    1990      48    36    12     1     3
    1994      44    38     7     7     4
    1998      37    45     6     7     5
    2002      41    42     8     9     0
    2005      37    36    10     8     9
    2009      38    23    15    11    12
    2013      49    31     0    10    10
\end{euleroutput}
\begin{eulercomment}
Perkalian matriks berikut ekstraks penjumlahan dari persentase dari dua partai besar
menunjukkan partai kecil mendapatkan sisa dari parlemen hingga 2009.
\end{eulercomment}
\begin{eulerprompt}
>BT1:=(BT.[1;1;0;0;0])'*100
\end{eulerprompt}
\begin{euleroutput}
  [84.29,  81.25,  81.1659,  82.7529,  72.9642,  61.8971,  79.8732]
\end{euleroutput}
\begin{eulercomment}
Ini juga merupakan sebuah plot statistikal sederhana. Kami menggunakan ini untuk
menampilkan garis dan titik secara simultan. Cara lainnya adalah memanggil plot2d dua kali
dengan \textgreater{}add.
\end{eulercomment}
\begin{eulerprompt}
>statplot(YT,BT1,"b"):
\end{eulerprompt}
\eulerimg{15}{images/Hamemayu Hayuningrat_23030630053_EMTStatistika-012.png}
\begin{eulercomment}
Definisi beberapa warna untuk setiap partai.
\end{eulercomment}
\begin{eulerprompt}
>CP:=[rgb(0.5,0.5,0.5),red,yellow,green,rgb(0.8,0,0)];
\end{eulerprompt}
\begin{eulercomment}
Sekarang kami dapat membuat plot hasil dari pemilihan 2009 dan mengubah menjadi satu plot
menggunakan gambar. Kami dapat menambahkan sebuah vektor dari kolum ke setiap plot.
\end{eulercomment}
\begin{eulerprompt}
>figure(2,1);  ...
>figure(1); columnsplot(BW[6,3:7],P,color=CP); ...
>figure(2); columnsplot(BW[6,3:7]-BW[5,3:7],P,color=CP);  ...
>figure(0):
\end{eulerprompt}
\eulerimg{15}{images/Hamemayu Hayuningrat_23030630053_EMTStatistika-013.png}
\begin{eulercomment}
Plot data menggabungkan baris-baris dari data statistik dalam satu plot.
\end{eulercomment}
\begin{eulerprompt}
>J:=BW[,1]'; DP:=BW[,3:7]'; ...
>dataplot(YT,BT',color=CP);  ...
>labelbox(P,colors=CP,styles="[]",>points,w=0.2,x=0.3,y=0.4):
\end{eulerprompt}
\eulerimg{15}{images/Hamemayu Hayuningrat_23030630053_EMTStatistika-014.png}
\begin{eulercomment}
Sebuah plot 3D kolom menunjukkan baris baris dari data statistik dalam bentuk kolom. Kami
menyediakan label-label untuk baris-baris dan kolom-kolom. Sudut adalah sudut pandangan.
\end{eulercomment}
\begin{eulerprompt}
>columnsplot3d(BT,scols=P,srows=YT, ...
>  angle=30°,ccols=CP):
\end{eulerprompt}
\eulerimg{15}{images/Hamemayu Hayuningrat_23030630053_EMTStatistika-015.png}
\begin{eulercomment}
Representasi lainnya adalah plot mosaik. Catatan bahwa kolom-kolom dari plot mewakili
kolom-kolom dari matrik ini. Karena panjang dari label CDU/CSU, kami menggambil jendela
terkecil daripada biasanya.
\end{eulercomment}
\begin{eulerprompt}
>shrinkwindow(>smaller);  ...
>mosaicplot(BT',srows=YT,scols=P,color=CP,style="#"); ...
>shrinkwindow():
\end{eulerprompt}
\eulerimg{15}{images/Hamemayu Hayuningrat_23030630053_EMTStatistika-016.png}
\begin{eulercomment}
Kami dapat juga melakukan sebuah chat pie. Karena hitam dan kuning membentuk sebuah
koalisi, kami mengurutkan kembali elemen-elemen.
\end{eulercomment}
\begin{eulerprompt}
>i=[1,3,5,4,2]; piechart(BW[6,3:7][i],color=CP[i],lab=P[i]):
\end{eulerprompt}
\eulerimg{15}{images/Hamemayu Hayuningrat_23030630053_EMTStatistika-017.png}
\begin{eulercomment}
Ini merupakan jenis plot lainnya.
\end{eulercomment}
\begin{eulerprompt}
>starplot(normal(1,10)+4,lab=1:10,>rays):
\end{eulerprompt}
\eulerimg{15}{images/Hamemayu Hayuningrat_23030630053_EMTStatistika-018.png}
\begin{eulercomment}
Beberapa plot dalam plot2d bagus dalam statistik. Ini merupakan plot implus dari data acak
secara seragam berdistribusi dalam [0,1].
\end{eulercomment}
\begin{eulerprompt}
>plot2d(makeimpulse(1:10,random(1,10)),>bar):
\end{eulerprompt}
\eulerimg{15}{images/Hamemayu Hayuningrat_23030630053_EMTStatistika-019.png}
\begin{eulercomment}
Tetapi secara eksponensial data berdistribusi, kami membutuhkan sebuah plot logaritmik
\end{eulercomment}
\begin{eulerprompt}
>logimpulseplot(1:10,-log(random(1,10))*10):
\end{eulerprompt}
\eulerimg{15}{images/Hamemayu Hayuningrat_23030630053_EMTStatistika-020.png}
\begin{eulercomment}
Fungsi columnsplot() mudah untuk digunakan, karena ini hanya membutuhkan sebuah vektor
dari nilai-nilai. Selebihnya, ini dapat diatur label-labelnya untuk sebarang yang kami
inginkan, kami mendemonstrasikan ini setelah tutorial ini.

Ini merupakan aplikasi lainnya, dimana kami menghitung karakter dalam sebuah kalimat dan
plot sebuah statistik.
\end{eulercomment}
\begin{eulerprompt}
>v=strtochar("the quick brown fox jumps over the lazy dog"); ...
>w=ascii("a"):ascii("z"); x=getmultiplicities(w,v); ...
>cw=[]; for k=w; cw=cw|char(k); end; ...
>columnsplot(x,lab=cw,width=0.05):
\end{eulerprompt}
\eulerimg{15}{images/Hamemayu Hayuningrat_23030630053_EMTStatistika-021.png}
\begin{eulercomment}
Ini juga memungkinkan secara manual mengatur sumbu-sumbu.
\end{eulercomment}
\begin{eulerprompt}
>n=10; p=0.4; i=0:n; x=bin(n,i)*p^i*(1-p)^(n-i); ...
>columnsplot(x,lab=i,width=0.05,<frame,<grid); ...
>yaxis(0,0:0.1:1,style="->",>left); xaxis(0,style="."); ...
>label("p",0,0.25), label("i",11,0); ...
>textbox(["Binomial distribution","with p=0.4"]):
\end{eulerprompt}
\eulerimg{15}{images/Hamemayu Hayuningrat_23030630053_EMTStatistika-022.png}
\begin{eulercomment}
Berikut merupakan sebuah cara untuk membuat plot frekuensi dari angka-angka dalam sebuah
vektor.

Kami membuat sebuah vektor dari bilangan bulat acak 1 ke 6.
\end{eulercomment}
\begin{eulerprompt}
>v:=intrandom(1,10,10)
\end{eulerprompt}
\begin{euleroutput}
  [8,  5,  8,  8,  6,  8,  8,  3,  5,  5]
\end{euleroutput}
\begin{eulercomment}
Lalu mengekstrasi angka-angka unique dalam v.
\end{eulercomment}
\begin{eulerprompt}
>vu:=unique(v)
\end{eulerprompt}
\begin{euleroutput}
  [3,  5,  6,  8]
\end{euleroutput}
\begin{eulercomment}
Dan plot frekuensi-frekuensi dalam sebuah plot kolom.
\end{eulercomment}
\begin{eulerprompt}
>columnsplot(getmultiplicities(vu,v),lab=vu,style="/"):
\end{eulerprompt}
\eulerimg{15}{images/Hamemayu Hayuningrat_23030630053_EMTStatistika-023.png}
\begin{eulercomment}
Kami ingin mendemonstrasikan fungsi untuk distribusi empiris dari nilai-nilai.
\end{eulercomment}
\begin{eulerprompt}
>x=normal(1,20);
\end{eulerprompt}
\begin{eulercomment}
Fungsi empdist(x,vs) membutuhkan sebuah nilai yang terurut dari nilai-nilai. Jadi kami
harus mengurutkan x sebelum kami digunakan.
\end{eulercomment}
\begin{eulerprompt}
>xs=sort(x);
\end{eulerprompt}
\begin{eulercomment}
Lalu kami plot distribusi empirik dan beberapa kotak kepadatan kedalam satu plot. Daripada
membuat plot batang untuk distribusi kami dengan sebuah sawtooth plot pada waktu ini.
\end{eulercomment}
\begin{eulerprompt}
>figure(2,1); ...
>figure(1); plot2d("empdist",-4,4;xs); ...
>figure(2); plot2d(histo(x,v=-4:0.2:4,<bar));  ...
>figure(0):
\end{eulerprompt}
\eulerimg{15}{images/Hamemayu Hayuningrat_23030630053_EMTStatistika-024.png}
\begin{eulercomment}
Sebuah plot pencar mudah untuk dilakukan pada Euler dengan titik plot umum. Graph berikut
menunjukkan bahwa X dan X+Y terlihat memiliki korelasi positif.
\end{eulercomment}
\begin{eulerprompt}
>x=normal(1,100); plot2d(x,x+rotright(x),>points,style=".."):
\end{eulerprompt}
\eulerimg{15}{images/Hamemayu Hayuningrat_23030630053_EMTStatistika-025.png}
\begin{eulercomment}
Terkadang, kami berharap untuk membandingkan dua contoh dari distribusi yang berbeda. Ini
dapat dilakukan dengan sebuah quantile-quantile-plot.

Untuk menguji, kami mencoba distribusi student-t dan distribusi eksponensial.
\end{eulercomment}
\begin{eulerprompt}
>x=randt(1,1000,5); y=randnormal(1,1000,mean(x),dev(x)); ...
>plot2d("x",r=6,style="--",yl="normal",xl="student-t",>vertical); ...
>plot2d(sort(x),sort(y),>points,color=red,style="x",>add):
\end{eulerprompt}
\eulerimg{15}{images/Hamemayu Hayuningrat_23030630053_EMTStatistika-026.png}
\begin{eulercomment}
Dengan jelas plot menunjukkan bahwa nilai nilai normal terdistribusi cenderung lebih kecil
pada ekstrim terakhirnya.

Jika kita memiliki dua distribusi dari ukuran yang berbeda, kami dapat memperluas yang
satu lebih kecil atau menyusut yang terbesar. Fungsi berikut bagus dikeduanya. Ini
mengambil nilai median dengn persentase diantara 0 dan 1.
\end{eulercomment}
\begin{eulerprompt}
>function medianexpand (x,n) := median(x,p=linspace(0,1,n-1));
\end{eulerprompt}
\begin{eulercomment}
Mari kita membandingkan dua distribusi yang seimbang.
\end{eulercomment}
\begin{eulerprompt}
>x=random(1000); y=random(400); ...
>plot2d("x",0,1,style="--"); ...
>plot2d(sort(medianexpand(x,400)),sort(y),>points,color=red,style="x",>add):
\end{eulerprompt}
\eulerimg{15}{images/Hamemayu Hayuningrat_23030630053_EMTStatistika-027.png}
\eulerheading{Regresi dan Korelasi}
\begin{eulercomment}
Regresi linear dapat dilakukan dengan fungsi polyfit() atau fungsi fit
lainnya.

Untuk memulainnya kami menemukan garis regresi untuk data univariat
dengan polyfit(x, y, l).
\end{eulercomment}
\begin{eulerprompt}
>x=1:10; y=[2,3,1,5,6,3,7,8,9,8]; writetable(x'|y',labc=["x","y"])
\end{eulerprompt}
\begin{euleroutput}
           x         y
           1         2
           2         3
           3         1
           4         5
           5         6
           6         3
           7         7
           8         8
           9         9
          10         8
\end{euleroutput}
\begin{eulercomment}
Kami ingin membandingkan tidak memiliki beban dan memiliki beban yang
cocok. Pertama koefisien dari kecocokan linear.
\end{eulercomment}
\begin{eulerprompt}
>p=polyfit(x,y,1)
\end{eulerprompt}
\begin{euleroutput}
  [0.733333,  0.812121]
\end{euleroutput}
\begin{eulercomment}
Sekarang koefisien dengan bobot yang menekankan nilai terakhir.
\end{eulercomment}
\begin{eulerprompt}
>w &= "exp(-(x-10)^2/10)"; pw=polyfit(x,y,1,w=w(x))
\end{eulerprompt}
\begin{euleroutput}
  [4.71566,  0.38319]
\end{euleroutput}
\begin{eulercomment}
Kita ambil semuanya ke dalam sebuah plot untuk titik titik dan garis
regresi, dan bobot yang digunakan.
\end{eulercomment}
\begin{eulerprompt}
>figure(2,1);  ...
>figure(1); statplot(x,y,"b",xl="Regression"); ...
>  plot2d("evalpoly(x,p)",>add,color=blue,style="--"); ...
>  plot2d("evalpoly(x,pw)",5,10,>add,color=red,style="--"); ...
>figure(2); plot2d(w,1,10,>filled,style="/",fillcolor=red,xl=w); ...
>figure(0):
\end{eulerprompt}
\eulerimg{15}{images/Hamemayu Hayuningrat_23030630053_EMTStatistika-028.png}
\begin{eulercomment}
Untuk contoh lainnya kami membaca sebuah survey dari murid-murid, umur
mereka dan umur orang tua mereka dan banyaknya saudara dari sebuah
file.

Tabel ini berisi "m" dan "f" di kolom kedua. Kami gunakan variabel
tok2 untuk mengatur translasi yang pas daripada membiarkan readtable()
mengumpulkan translasi.
\end{eulercomment}
\begin{eulerprompt}
>\{MS,hd\}:=readtable("table1.dat",tok2:=["m","f"]);  ...
>writetable(MS,labc=hd,tok2:=["m","f"]);
\end{eulerprompt}
\begin{euleroutput}
      Person       Sex       Age    Mother    Father  Siblings
           1         m        29        58        61         1
           2         f        26        53        54         2
           3         m        24        49        55         1
           4         f        25        56        63         3
           5         f        25        49        53         0
           6         f        23        55        55         2
           7         m        23        48        54         2
           8         m        27        56        58         1
           9         m        25        57        59         1
          10         m        24        50        54         1
          11         f        26        61        65         1
          12         m        24        50        52         1
          13         m        29        54        56         1
          14         m        28        48        51         2
          15         f        23        52        52         1
          16         m        24        45        57         1
          17         f        24        59        63         0
          18         f        23        52        55         1
          19         m        24        54        61         2
          20         f        23        54        55         1
\end{euleroutput}
\begin{eulercomment}
Bagaimana umur bergantung dari yang lainnya? Kesan pertama datang dari
plot pencar berpasangan.
\end{eulercomment}
\begin{eulerprompt}
>scatterplots(tablecol(MS,3:5),hd[3:5]):
\end{eulerprompt}
\eulerimg{15}{images/Hamemayu Hayuningrat_23030630053_EMTStatistika-029.png}
\begin{eulercomment}
Jelas bahwa umur dari ayah dan ibu tergantung pada setiapnya. Mari
kita tentukan dan plot garis regresi.
\end{eulercomment}
\begin{eulerprompt}
>cs:=MS[,4:5]'; ps:=polyfit(cs[1],cs[2],1)
\end{eulerprompt}
\begin{euleroutput}
  [17.3789,  0.740964]
\end{euleroutput}
\begin{eulercomment}
Ini secara jelas merupakan model yang salah. Garis regresi seharusnya
=17+0.74t, dimana t adalah umur dari ibu dan s adalah umur dari ayah.
Perbedaan umur mungkin tergantung sedikit dari umur, tetapi tidak
banyak.

Selain itu, kami mengira sebuah fungsi seperti s=a+t. Lalu a adalah
rata rata dari s-t. Ini merupakan usia rata-rata perbedaan antara ayah
dan ibu.
\end{eulercomment}
\begin{eulerprompt}
>da:=mean(cs[2]-cs[1])
\end{eulerprompt}
\begin{euleroutput}
  3.65
\end{euleroutput}
\begin{eulercomment}
Mari kita plot ini ke satu plot pencar.
\end{eulercomment}
\begin{eulerprompt}
>plot2d(cs[1],cs[2],>points);  ...
>plot2d("evalpoly(x,ps)",color=red,style=".",>add);  ...
>plot2d("x+da",color=blue,>add):
\end{eulerprompt}
\eulerimg{15}{images/Hamemayu Hayuningrat_23030630053_EMTStatistika-030.png}
\begin{eulercomment}
Ini merupakan plot kotak dari dua umur. Ini hanya menunjukkan, bahwa
umur berbeda.
\end{eulercomment}
\begin{eulerprompt}
>boxplot(cs,["mothers","fathers"]):
\end{eulerprompt}
\eulerimg{15}{images/Hamemayu Hayuningrat_23030630053_EMTStatistika-031.png}
\begin{eulercomment}
Ini menarik bahwa perbedaan dalam nilai tengah tidak sebesar seperti
perbedaan dalam nilai rata-rata.
\end{eulercomment}
\begin{eulerprompt}
>median(cs[2])-median(cs[1])
\end{eulerprompt}
\begin{euleroutput}
  1.5
\end{euleroutput}
\begin{eulercomment}
Koefisien korelasi menyarankan korelasi positif.
\end{eulercomment}
\begin{eulerprompt}
>correl(cs[1],cs[2])
\end{eulerprompt}
\begin{euleroutput}
  0.7588307236
\end{euleroutput}
\begin{eulercomment}
Korelasi dari peringkat merupakan sebuah ukuran untuk urutan yang sama
dalam vektor keduannya. Ini juga sedikit positif.
\end{eulercomment}
\begin{eulerprompt}
>rankcorrel(cs[1],cs[2])
\end{eulerprompt}
\begin{euleroutput}
  0.758925292358
\end{euleroutput}
\eulerheading{Membuat fungsi-fungsi baru}
\begin{eulercomment}
Tentu saja, bahasa EMT dapat digunakan untuk membuat program fungsi
baru. Sebagai contoh kami mendefinisikan fungsi skewness.

\end{eulercomment}
\begin{eulerformula}
\[
\text{sk}(x) = \dfrac{\sqrt{n} \sum_i (x_i-m)^3}{\left(\sum_i (x_i-m)^2\right)^{3/2}}
\]
\end{eulerformula}
\begin{eulercomment}
dimana m adalah rata-rata dari x.
\end{eulercomment}
\begin{eulerprompt}
>function skew (x:vector) ...
\end{eulerprompt}
\begin{eulerudf}
  m=mean(x);
  return sqrt(cols(x))*sum((x-m)^3)/(sum((x-m)^2))^(3/2);
  endfunction
\end{eulerudf}
\begin{eulercomment}
Seperti yang kamu lihat, kami dengan mudah menggunakan bahasa matriks
untuk mendapatkan sebuah implementasi yang sangat singkat dan efisien.
Mari kita coba fungsi ini.
\end{eulercomment}
\begin{eulerprompt}
>data=normal(20); skew(normal(10))
\end{eulerprompt}
\begin{euleroutput}
  -0.198710316203
\end{euleroutput}
\begin{eulercomment}
Ini merupakan fungsi lainnya, sebut saja koefisien skewness Pearson.
\end{eulercomment}
\begin{eulerprompt}
>function skew1 (x) := 3*(mean(x)-median(x))/dev(x)
>skew1(data)
\end{eulerprompt}
\begin{euleroutput}
  -0.0801873249135
\end{euleroutput}
\eulerheading{Simulasi Monte Carlo}
\begin{eulercomment}
Euler dapat digunakan untuk mensimulasikan kejadian acak. Kami telah
melihatnya contoh sederhana diatas. Ini merupakan yang laninya, yang
mana mensimulasikan 1000 kali 3 dadu dilempar, dan menanyakan
distribusi dari jumlahannya.
\end{eulercomment}
\begin{eulerprompt}
>ds:=sum(intrandom(1000,3,6))';  fs=getmultiplicities(3:18,ds)
\end{eulerprompt}
\begin{euleroutput}
  [5,  17,  35,  44,  75,  97,  114,  116,  143,  116,  104,  53,  40,
  22,  13,  6]
\end{euleroutput}
\begin{eulercomment}
Kami dapat plot ini sekarang.
\end{eulercomment}
\begin{eulerprompt}
>columnsplot(fs,lab=3:18):
\end{eulerprompt}
\eulerimg{15}{images/Hamemayu Hayuningrat_23030630053_EMTStatistika-032.png}
\begin{eulercomment}
Untuk menentukan distribusi yang diinginkan tidak mudah. Kami
menggunakan sebuah rekursi lanjut untuk ini.

Berdasarkan perhitungan fungsi jumlah cara angka k dapat
merepresentasikan sebagai jumlahan dari angka n dalam rentang 1 ke m.
ini bekerja secara rekursif dalam cara yang jelas.
\end{eulercomment}
\begin{eulerprompt}
>function map countways (k; n, m) ...
\end{eulerprompt}
\begin{eulerudf}
    if n==1 then return k>=1 && k<=m
    else
      sum=0; 
      loop 1 to m; sum=sum+countways(k-#,n-1,m); end;
      return sum;
    end;
  endfunction
\end{eulerudf}
\begin{eulercomment}
Ini merupakan hasil dari lemparan tiga dadu.
\end{eulercomment}
\begin{eulerprompt}
>countways(5:25,5,5)
\end{eulerprompt}
\begin{euleroutput}
  [1,  5,  15,  35,  70,  121,  185,  255,  320,  365,  381,  365,  320,
  255,  185,  121,  70,  35,  15,  5,  1]
\end{euleroutput}
\begin{eulerprompt}
>cw=countways(3:18,3,6)
\end{eulerprompt}
\begin{euleroutput}
  [1,  3,  6,  10,  15,  21,  25,  27,  27,  25,  21,  15,  10,  6,  3,
  1]
\end{euleroutput}
\begin{eulercomment}
Kami tambahkan nilai harapan ke plot.
\end{eulercomment}
\begin{eulerprompt}
>plot2d(cw/6^3*1000,>add); plot2d(cw/6^3*1000,>points,>add):
\end{eulerprompt}
\eulerimg{15}{images/Hamemayu Hayuningrat_23030630053_EMTStatistika-033.png}
\begin{eulercomment}
Untuk simulasi lainnya, deviasi dari nilai rata-rata n 0-1-normal
variabel acak distribusi adalah 1/sqrt(n).
\end{eulercomment}
\begin{eulerprompt}
>longformat; 1/sqrt(10)
\end{eulerprompt}
\begin{euleroutput}
  0.316227766017
\end{euleroutput}
\begin{eulercomment}
Mari kita periksa ini dengan sebuah simulasi. Kami menghasilkan 10000
kali 10 vektor acak.
\end{eulercomment}
\begin{eulerprompt}
>M=normal(10000,10); dev(mean(M)')
\end{eulerprompt}
\begin{euleroutput}
  0.319493614817
\end{euleroutput}
\begin{eulerprompt}
>plot2d(mean(M)',>distribution):
\end{eulerprompt}
\eulerimg{15}{images/Hamemayu Hayuningrat_23030630053_EMTStatistika-034.png}
\begin{eulercomment}
Nilai median dari 10 0-1-normal distribusi angka acak memiliki deviasi
yang lebih besar.
\end{eulercomment}
\begin{eulerprompt}
>dev(median(M)')
\end{eulerprompt}
\begin{euleroutput}
  0.374460271535
\end{euleroutput}
\begin{eulercomment}
Karena kami dapat dengan mudah membuat langkah acak, kami dapat
mensimulasikan proses Wiener. Kami mengambil 1000 langkah dari 1000
proses. Kami lalu plot standar deviasi dan rata-rata dari langkah ke-n
dari proses bersama ini dengan nilai harapan dalam merah.
\end{eulercomment}
\begin{eulerprompt}
>n=1000; m=1000; M=cumsum(normal(n,m)/sqrt(m)); ...
>t=(1:n)/n; figure(2,1); ...
>figure(1); plot2d(t,mean(M')'); plot2d(t,0,color=red,>add); ...
>figure(2); plot2d(t,dev(M')'); plot2d(t,sqrt(t),color=red,>add); ...
>figure(0):
\end{eulerprompt}
\eulerimg{15}{images/Hamemayu Hayuningrat_23030630053_EMTStatistika-035.png}
\eulerheading{Pengujian}
\begin{eulercomment}
Pengujuan merupakan alat yang penting dalam statistika. Dalam Euler,
banyak uji yang diimplementasikan. Semua uji ini mengembalikan error
yang dapat kita terima apabila menolak hipotesis nol.

Sebagai contoh, kami menguji pelemparan dadu untuk distribusi seragam.
Pada lemparan 600, kami mendapatkan nilai berikut, yang mana dapat
kami masukkan kedalam uji chi-square.
\end{eulercomment}
\begin{eulerprompt}
>chitest([90,103,114,101,103,89],dup(100,6)')
\end{eulerprompt}
\begin{euleroutput}
  0.498830517952
\end{euleroutput}
\begin{eulercomment}
Uji chi-square juga memiliki sebuah mode, yang mana menggunakan
simulasi Monte Carlo untuk menguji statistik. Hasilnya seharusnya
hampir sama. Parameter \textgreater{}p menunjukkan vektor-y sebagai vektor peluang.
\end{eulercomment}
\begin{eulerprompt}
>chitest([90,103,114,101,103,89],dup(1/6,6)',>p,>montecarlo)
\end{eulerprompt}
\begin{euleroutput}
  0.526
\end{euleroutput}
\begin{eulercomment}
Error ini terlalu besar. Jadi kami dapat menolak distribusi seragam.
Ini tidak menunjukkan bahwa dadu kami tidak adil. Tapi kami tidak
dapat menolak hipotesis kami.

Selanjutnya kami membuat 1000 dadu dilemparkan menggunakan pembuat
angka acak dan menggunakan uji yang sama.
\end{eulercomment}
\begin{eulerprompt}
>n=1000; t=random([1,n*6]); chitest(count(t*6,6),dup(n,6)')
\end{eulerprompt}
\begin{euleroutput}
  0.528028118442
\end{euleroutput}
\begin{eulercomment}
Mari kita uji untuk nilai rata-rata 100 dengan uji-t.
\end{eulercomment}
\begin{eulerprompt}
>s=200+normal([1,100])*10; ...
>ttest(mean(s),dev(s),100,200)
\end{eulerprompt}
\begin{euleroutput}
  0.0218365848476
\end{euleroutput}
\begin{eulercomment}
ttest() fungsi membutuhkan nilai rata-rata, deviasi, banyaknya data,
dan nilai rata-rata untuk mengujinya.

Sekarang mari kita periksa perhitungan untuk nilai rata-rata yang
sama. Kami menolak hipotesis bahwa mereka memiliki rata-rata yang
sama, jika hasilnya adalah \textless{} 0.05.
\end{eulercomment}
\begin{eulerprompt}
>tcomparedata(normal(1,10),normal(1,10))
\end{eulerprompt}
\begin{euleroutput}
  0.38722000942
\end{euleroutput}
\begin{eulercomment}
Jika kita menambahkan sebuah bias ke distribusi satu, kami mendapatkan
lebih banyak penolakan. Mengulang simulasi ini beberapa kali untuk
melihat pengaruhnya.
\end{eulercomment}
\begin{eulerprompt}
>tcomparedata(normal(1,10),normal(1,10)+2)
\end{eulerprompt}
\begin{euleroutput}
  5.60009101758e-07
\end{euleroutput}
\begin{eulercomment}
Contoh selanjutnya, kami membuat 20 dadu acak dilempar 100 kali dan
menghitung satunya. Terdapat satu 20/6=3.3 dalam rata-rata.
\end{eulercomment}
\begin{eulerprompt}
>R=random(100,20); R=sum(R*6<=1)'; mean(R)
\end{eulerprompt}
\begin{euleroutput}
  3.28
\end{euleroutput}
\begin{eulercomment}
Kami sekarang membandingkan angkka satu dengan distribusi binomial.
Pertama kami plot distribusi angka satu.
\end{eulercomment}
\begin{eulerprompt}
>plot2d(R,distribution=max(R)+1,even=1,style="\(\backslash\)/"):
\end{eulerprompt}
\eulerimg{15}{images/Hamemayu Hayuningrat_23030630053_EMTStatistika-036.png}
\begin{eulerprompt}
>t=count(R,21);
\end{eulerprompt}
\begin{eulercomment}
Lalu kami hitung nilai harapan.
\end{eulercomment}
\begin{eulerprompt}
>n=0:20; b=bin(20,n)*(1/6)^n*(5/6)^(20-n)*100;
\end{eulerprompt}
\begin{eulercomment}
Kami mengumpulkan beberapa angka untuk mendapatkan kategori, yang mana
cukup besar.
\end{eulercomment}
\begin{eulerprompt}
>t1=sum(t[1:2])|t[3:7]|sum(t[8:21]); ...
>b1=sum(b[1:2])|b[3:7]|sum(b[8:21]);
\end{eulerprompt}
\begin{eulercomment}
Uji chi-square menolak hipotesis bahwa distribusi kami merupakan
sebuah distribusi binomial, jika hasinya adalah \textless{} 0.05.
\end{eulercomment}
\begin{eulerprompt}
>chitest(t1,b1)
\end{eulerprompt}
\begin{euleroutput}
  0.53921579764
\end{euleroutput}
\begin{eulercomment}
Contoh berikut memiliki hasil dari dua grup dari orang-orang
(laki-laki dan perempuan, katakanlah) memilih satu dari enam partai.
\end{eulercomment}
\begin{eulerprompt}
>A=[23,37,43,52,64,74;27,39,41,49,63,76];  ...
>  writetable(A,wc=6,labr=["m","f"],labc=1:6)
\end{eulerprompt}
\begin{euleroutput}
             1     2     3     4     5     6
       m    23    37    43    52    64    74
       f    27    39    41    49    63    76
\end{euleroutput}
\begin{eulercomment}
Kami berharap menguji secara bebas dari suara untuk jenis kelamin.
Tabel uji chi\textasciicircum{}2 melakukan ini. Hasilnya terlalu besar untuk menolak
saling bebas. Jadi kita tidak dapat mengatakan, jika suara tergantung
pada jenis kelamin untuk data ini.
\end{eulercomment}
\begin{eulerprompt}
>tabletest(A)
\end{eulerprompt}
\begin{euleroutput}
  0.990701632326
\end{euleroutput}
\begin{eulercomment}
Tabel nilai harapan berikut, jika kita berasumsi pemungutan suara yang
diamati.
\end{eulercomment}
\begin{eulerprompt}
>writetable(expectedtable(A),wc=6,dc=1,labr=["m","f"],labc=1:6)
\end{eulerprompt}
\begin{euleroutput}
             1     2     3     4     5     6
       m  24.9  37.9  41.9  50.3  63.3  74.7
       f  25.1  38.1  42.1  50.7  63.7  75.3
\end{euleroutput}
\begin{eulercomment}
Kami dapat menghitung koefisien kontigensi yang dikoreksi. Karena ini
sangat dekat dengan 0, kami menyimpulkan bahwa pemungutan suara tidak
tergantung pada jenis kelamin.
\end{eulercomment}
\begin{eulerprompt}
>contingency(A)
\end{eulerprompt}
\begin{euleroutput}
  0.0427225484717
\end{euleroutput}
\begin{eulercomment}
\begin{eulercomment}
\eulerheading{Beberapa Pengujian Lainnya}
\begin{eulercomment}
Selanjutnya kami  menggunakan analisis variansi (uji-F) untuk menguji
tiga sample dari data distribusi secara normal untuk nilai rata-rata
yang sama. Metode ini dinamakan ANOVA (variansi analisis). Dalam
Euluer, fungsi varanalysis() digunakan.
\end{eulercomment}
\begin{eulerprompt}
>x1=[109,111,98,119,91,118,109,99,115,109,94]; mean(x1),
\end{eulerprompt}
\begin{euleroutput}
  106.545454545
\end{euleroutput}
\begin{eulerprompt}
>x2=[120,124,115,139,114,110,113,120,117]; mean(x2),
\end{eulerprompt}
\begin{euleroutput}
  119.111111111
\end{euleroutput}
\begin{eulerprompt}
>x3=[120,112,115,110,105,134,105,130,121,111]; mean(x3)
\end{eulerprompt}
\begin{euleroutput}
  116.3
\end{euleroutput}
\begin{eulerprompt}
>varanalysis(x1,x2,x3)
\end{eulerprompt}
\begin{euleroutput}
  0.0138048221371
\end{euleroutput}
\begin{eulercomment}
Ini berarti, kami menolak hipotesis dari nilai rata-rata yang sama.
Kami melakukan ini dengan peluang error dari 1.3\%.

Ini juga pengujian median, yang mana menolak data sampel dengan
perbedaan rata-rata distribusi pengujian median dari sampel gabungan.
\end{eulercomment}
\begin{eulerprompt}
>a=[56,66,68,49,61,53,45,58,54];
>b=[72,81,51,73,69,78,59,67,65,71,68,71];
>mediantest(a,b)
\end{eulerprompt}
\begin{euleroutput}
  0.0241724220052
\end{euleroutput}
\begin{eulercomment}
Uji lainnya dalam kesamaan adalah uji rank. Ini lebih tajam daripada
uji median.
\end{eulercomment}
\begin{eulerprompt}
>ranktest(a,b)
\end{eulerprompt}
\begin{euleroutput}
  0.00199969612469
\end{euleroutput}
\begin{eulercomment}
Contoh berikut, distribusi keduanya memiliki nilai rata-rata yang
sama.
\end{eulercomment}
\begin{eulerprompt}
>ranktest(random(1,100),random(1,50)*3-1)
\end{eulerprompt}
\begin{euleroutput}
  0.129608141484
\end{euleroutput}
\begin{eulercomment}
Mari kita coba untuk mensimulasikan perlakuan a dan b diaplikasikan ke
orang-orang yang berbeda.
\end{eulercomment}
\begin{eulerprompt}
>a=[8.0,7.4,5.9,9.4,8.6,8.2,7.6,8.1,6.2,8.9];
>b=[6.8,7.1,6.8,8.3,7.9,7.2,7.4,6.8,6.8,8.1];
\end{eulerprompt}
\begin{eulercomment}
Uji signum memutuskan, jika a lebih baik daripada b.
\end{eulercomment}
\begin{eulerprompt}
>signtest(a,b)
\end{eulerprompt}
\begin{euleroutput}
  0.0546875
\end{euleroutput}
\begin{eulercomment}
Ini memiliki banyak error. Kami tidak bisa menolak bahwa a sebaik b.

Uji Wilcoxon memberikan hasil lebih tajam dari uji ini, tetapi
bergantung pada nilai kuantitatif dari perbedaan.
\end{eulercomment}
\begin{eulerprompt}
>wilcoxon(a,b)
\end{eulerprompt}
\begin{euleroutput}
  0.0296680599405
\end{euleroutput}
\begin{eulercomment}
Mari kita coba uji dua tes lagi menggunakan deret yang dibuat.
\end{eulercomment}
\begin{eulerprompt}
>wilcoxon(normal(1,20),normal(1,20)-1)
\end{eulerprompt}
\begin{euleroutput}
  0.0068706451766
\end{euleroutput}
\begin{eulerprompt}
>wilcoxon(normal(1,20),normal(1,20))
\end{eulerprompt}
\begin{euleroutput}
  0.275145971064
\end{euleroutput}
\eulerheading{Angka-Angka Acak}
\begin{eulercomment}
Uji berikut untuk pembuat angka acak. Euler menggunakan sebuah
pengacak angka yang bagus, jadi kita tidak mengharapkan masalah
lainnya.

Pertama kita buat sepuluh juta angka acak dalam [0, 1].
\end{eulercomment}
\begin{eulerprompt}
>n:=10000000; r:=random(1,n);
\end{eulerprompt}
\begin{eulercomment}
Sekarang kita hitung jarak antara dua angka kurang dari 0.05.
\end{eulercomment}
\begin{eulerprompt}
>a:=0.05; d:=differences(nonzeros(r<a));
\end{eulerprompt}
\begin{eulercomment}
Terakhir, kami plot angka beberapa kali, setiap jarak yang terjadi dan
bandingkan dengan nilai harapan.
\end{eulercomment}
\begin{eulerprompt}
>m=getmultiplicities(1:100,d); plot2d(m); ...
>  plot2d("n*(1-a)^(x-1)*a^2",color=red,>add):
\end{eulerprompt}
\eulerimg{15}{images/Hamemayu Hayuningrat_23030630053_EMTStatistika-037.png}
\begin{eulercomment}
Hapus data.
\end{eulercomment}
\begin{eulerprompt}
>remvalue n;
\end{eulerprompt}
\begin{eulercomment}
\begin{eulercomment}
\eulerheading{Pengantar untuk pengguna dalam Projek R}
\begin{eulercomment}
Secara jelas, EMT tidak bersaing dengan R sebagai package statistik.
Namun, ini banyak prosedur statistik dan fungsi yang tersedia dalam
EMT juga. Jadi EMT mungkin memenuhi kebutuhan dasar. Kemudian, EMT
datang dengan numerical package dan sistem komputer aljabar.

Notebook ini untuk kamu jika umum dengan R, tetapi ingin tahu
perbedaan sintaks dari EMT dan R. Kami mencoba untuk memberikan
gambaran jelas dan kurang jelas yang kamu ketahui.

Selebihnya, kami lihat pada cara untuk bertukar data pada kedua
sistem.
\end{eulercomment}
\begin{eulercomment}
Catatan ini dalam proses pengerjaan.
\end{eulercomment}
\eulerheading{Sintaks Dasar}
\begin{eulercomment}
Hal pertama yang dipelajari dalam R adalah membuat sebuah vektor.
Dalam EMT, perbedaan utamanya pada operator :  yang dapat mengambil
jarak langkah. Lebih dari itu memiliki daya ikat rendah.
\end{eulercomment}
\begin{eulerprompt}
>n=10; 0:n/20:n-1
\end{eulerprompt}
\begin{euleroutput}
  [0,  0.5,  1,  1.5,  2,  2.5,  3,  3.5,  4,  4.5,  5,  5.5,  6,  6.5,
  7,  7.5,  8,  8.5,  9]
\end{euleroutput}
\begin{eulercomment}
Fungsi c() tidak ada. Ini memungkinkan untuk membuat  vektor menjadi
benda-benda yang bergabung.

Contoh berikut, seperti lainnya, dari "Introduction to R" datang dari
projek R. Jika kamu membaca PDF ini, kamu akan menemukan bahwa I
mengikuti path dalam tutorial ini.
\end{eulercomment}
\begin{eulerprompt}
>x=[10.4, 5.6, 3.1, 6.4, 21.7]; [x,0,x]
\end{eulerprompt}
\begin{euleroutput}
  [10.4,  5.6,  3.1,  6.4,  21.7,  0,  10.4,  5.6,  3.1,  6.4,  21.7]
\end{euleroutput}
\begin{eulercomment}
Operator colon dengan ukuran langkah dari EMT menggantikan fungsi
seq() dalam R. Kami dapat menulis fungsi ini dalam EMT.
\end{eulercomment}
\begin{eulerprompt}
>function seq(a,b,c) := a:b:c; ...
>seq(0,-0.1,-1)
\end{eulerprompt}
\begin{euleroutput}
  [0,  -0.1,  -0.2,  -0.3,  -0.4,  -0.5,  -0.6,  -0.7,  -0.8,  -0.9,  -1]
\end{euleroutput}
\begin{eulercomment}
Fungsi rep() dari R tidak menunjukkan dalam EMT. Untuk masukan vektor,
ini dapat ditulis seperti berikut.
\end{eulercomment}
\begin{eulerprompt}
>function rep(x:vector,n:index) := flatten(dup(x,n)); ...
>rep(x,2)
\end{eulerprompt}
\begin{euleroutput}
  [10.4,  5.6,  3.1,  6.4,  21.7,  10.4,  5.6,  3.1,  6.4,  21.7]
\end{euleroutput}
\begin{eulercomment}
Catatan bahwa "=" atau ":=" digunakan untuk penugasan. Operator "-\textgreater{}"
digunakan untuk ukuran dalam EMT.
\end{eulercomment}
\begin{eulerprompt}
>125km -> " miles"
\end{eulerprompt}
\begin{euleroutput}
  77.6713990297 miles
\end{euleroutput}
\begin{eulercomment}
Operator "\textless{}-" untuk penugasan adalah cara yang menyesatkan, dan ide
buruk dari R. Berikut akan dibandingkan a dan -4 dalam EMT.
\end{eulercomment}
\begin{eulerprompt}
>a=2; a<-4
\end{eulerprompt}
\begin{euleroutput}
  0
\end{euleroutput}
\begin{eulercomment}
Dalam R, "a \textless{} -4 \textless{} 3" bekerja, tetapi "a \textless{} -4 \textless{} -3" tidak. Saya
memiliki ambiguitas yang sama dalam EMT juga, tetapi mencoba untuk
mengapuskan mereka sedikit demi sedikit.

EMT dan R memiliki vektor dari tipe boolean. Tetapi dalam EMT, angka 0
dan 1 digunakan untuk mewakili salah dan benar. Dalam R, nilai true
dan false dapat demikian digunakan dalam aritmetik biasa seperti dalam
EMT.
\end{eulercomment}
\begin{eulerprompt}
>x<5, %*x
\end{eulerprompt}
\begin{euleroutput}
  [0,  0,  1,  0,  0]
  [0,  0,  3.1,  0,  0]
\end{euleroutput}
\begin{eulercomment}
EMT mengeluarkan error atau menghasilkan NAN tergantung pada "error".
\end{eulercomment}
\begin{eulerprompt}
>errors off; 0/0, isNAN(sqrt(-1)), errors on;
\end{eulerprompt}
\begin{euleroutput}
  NAN
  1
\end{euleroutput}
\begin{eulercomment}
String merupakan hal yang sama dalam R dan EMT. Keduanya memiliki
locale sekarang, bukan dalam Unicode.

Dalam R terdapat packages dalam Unicode. Dalam EMT, sebuah string
dapat berupa string Unicode. Sebuah string unicode dapat diubah
kedalam encode local dan sebaliknya. Selanjutnya, u"..." dapat berisi
entitas HTML.
\end{eulercomment}
\begin{eulerprompt}
>u"&#169; Ren&eacut; Grothmann"
\end{eulerprompt}
\begin{euleroutput}
  © René Grothmann
\end{euleroutput}
\begin{eulercomment}
Berikut mungkin atau mungkin tidak menampilkan secara benar dalam
sitem kamu sebagai A dengan dot dan dash diatas. Tergantung pada font
yang digunakan.
\end{eulercomment}
\begin{eulerprompt}
>chartoutf([480])
\end{eulerprompt}
\begin{euleroutput}
  Ǡ
\end{euleroutput}
\begin{eulercomment}
Gabungan string dapat dilakukan dengan "+" atau "\textbar{}". Ini berisi angka,
yang mana akan mencetak dalam format sekarang.
\end{eulercomment}
\begin{eulerprompt}
>"pi = "+pi
\end{eulerprompt}
\begin{euleroutput}
  pi = 3.14159265359
\end{euleroutput}
\eulerheading{Pengindeks-an}
\begin{eulercomment}
Hampir setiap kali, ini bekerja pada R.

Tetapi EMT mengintrepetasikan indeks negatis dari belakang vektor
sementara R mengintrepetasikan x[n] sebagai x tanpa elemen ke-n.
\end{eulercomment}
\begin{eulerprompt}
>x, x[1:3], x[-2]
\end{eulerprompt}
\begin{euleroutput}
  [10.4,  5.6,  3.1,  6.4,  21.7]
  [10.4,  5.6,  3.1]
  6.4
\end{euleroutput}
\begin{eulercomment}
Perilaku dari R dapat di dapatkan dalam EMT dengan drop().
\end{eulercomment}
\begin{eulerprompt}
>drop(x,2)
\end{eulerprompt}
\begin{euleroutput}
  [10.4,  3.1,  6.4,  21.7]
\end{euleroutput}
\begin{eulercomment}
Vektor logika tidak jauh berbeda seperti indeks dalam EMT, dalam beda
ke R. Kamu butuh untuk mengekstrak elemen bukan nol pertama dalam EMT.
\end{eulercomment}
\begin{eulerprompt}
>x, x>5, x[nonzeros(x>5)]
\end{eulerprompt}
\begin{euleroutput}
  [10.4,  5.6,  3.1,  6.4,  21.7]
  [1,  1,  0,  1,  1]
  [10.4,  5.6,  6.4,  21.7]
\end{euleroutput}
\begin{eulercomment}
Seperti dalam R, indeks vektor dapat berisi repetisi.
\end{eulercomment}
\begin{eulerprompt}
>x[[1,2,2,1]]
\end{eulerprompt}
\begin{euleroutput}
  [10.4,  5.6,  5.6,  10.4]
\end{euleroutput}
\begin{eulercomment}
Tetapi nama untuk indeks tidak mungkin dalam EMT. Untuk sebuah paket
statistika, ini mungkin dibutuhkan untuk memudahkan akses elemen dari
vektor.

Untuk meniru perilaku ini, kami dapat mendefinisikan sebuah fungsi
seperti berikut.
\end{eulercomment}
\begin{eulerprompt}
>function sel (v,i,s) := v[indexof(s,i)]; ...
>s=["first","second","third","fourth"]; sel(x,["first","third"],s)
\end{eulerprompt}
\begin{euleroutput}
  
  Trying to overwrite protected function sel!
  Error in:
  function sel (v,i,s) := v[indexof(s,i)]; ... ...
               ^
  
  Trying to overwrite protected function sel!
  Error in:
  function sel (v,i,s) := v[indexof(s,i)]; ... ...
               ^
  
  Trying to overwrite protected function sel!
  Error in:
  function sel (v,i,s) := v[indexof(s,i)]; ... ...
               ^
  
  Trying to overwrite protected function sel!
  Error in:
  function sel (v,i,s) := v[indexof(s,i)]; ... ...
               ^
  [10.4,  3.1]
\end{euleroutput}
\eulerheading{Tipe Data}
\begin{eulercomment}
EMT memiliki tipe data tetap dari pada R. Secara jelas, dalam R
memiliki vektor bertumbuh. KAmu dapat mengatur sebuah vektor numerik
kosong v dan menugaskan sebuah nilai ke elemen v[17]. Ini tidak
mungkin dalam EMT.

Berikut sedikit tidak efisien.
\end{eulercomment}
\begin{eulerprompt}
>v=[]; for i=1 to 10000; v=v|i; end;
\end{eulerprompt}
\begin{eulercomment}
EMT sekarang akan membuat sebuah vektor dengan v dan i diatambahkan
pada stck dan menyalin vektor itu kembali ke variabel global v.

Lebih efisien pra-definisi vektor.
\end{eulercomment}
\begin{eulerprompt}
>v=zeros(10000); for i=1 to 10000; v[i]=i; end;
\end{eulerprompt}
\begin{eulercomment}
Untuk mengubah tipe data dalam EMT, kamu dapat menggunakan fungsi
seperti complex().
\end{eulercomment}
\begin{eulerprompt}
>complex(1:4)
\end{eulerprompt}
\begin{euleroutput}
  [ 1+0i ,  2+0i ,  3+0i ,  4+0i  ]
\end{euleroutput}
\begin{eulercomment}
Pengubahan ke string memungkinkan untuk tipe data dasar saja. Format
sekarang digunakan untuk penggabungan string sederhana. Tetapi
terdapat fungsi seperti print() atau frac().

Untuk vektor, kamu dapat dengan mudah menulis fungsi kamu sendiri.
\end{eulercomment}
\begin{eulerprompt}
>function tostr (v) ...
\end{eulerprompt}
\begin{eulerudf}
  s="[";
  loop 1 to length(v);
     s=s+print(v[#],2,0);
     if #<length(v) then s=s+","; endif;
  end;
  return s+"]";
  endfunction
\end{eulerudf}
\begin{eulerprompt}
>tostr(linspace(0,1,10))
\end{eulerprompt}
\begin{euleroutput}
  [0.00,0.10,0.20,0.30,0.40,0.50,0.60,0.70,0.80,0.90,1.00]
\end{euleroutput}
\begin{eulercomment}
Untuk komunikasi dengan Maxima, terdapat sebuah fungsi convertmxm(),
yang mana juga dapat digunakan mengubah sebuah vektor untuk keluaran.
\end{eulercomment}
\begin{eulerprompt}
>convertmxm(1:10)
\end{eulerprompt}
\begin{euleroutput}
  [1,2,3,4,5,6,7,8,9,10]
\end{euleroutput}
\begin{eulercomment}
Untuk Latex perintah tex dapat digunakan untuk mendapatkan perintah
Latex.
\end{eulercomment}
\begin{eulerprompt}
>tex(&[1,2,3])
\end{eulerprompt}
\begin{euleroutput}
  \(\backslash\)left[ 1 , 2 , 3 \(\backslash\)right] 
\end{euleroutput}
\eulerheading{Faktor Faktor dan Tabel Tabel}
\begin{eulercomment}
Dalam pengantar R terdapat sebuah contoh bernama faktor-faktor.

Berikut merupakan sebuah daftar dari wilayah 30 negara bagian.
\end{eulercomment}
\begin{eulerprompt}
>austates = ["tas", "sa", "qld", "nsw", "nsw", "nt", "wa", "wa", ...
>"qld", "vic", "nsw", "vic", "qld", "qld", "sa", "tas", ...
>"sa", "nt", "wa", "vic", "qld", "nsw", "nsw", "wa", ...
>"sa", "act", "nsw", "vic", "vic", "act"];
\end{eulerprompt}
\begin{eulercomment}
Asumsikan, kita memiliki pendapatan yang berkorespodensi dalam setiap
negara bagian.
\end{eulercomment}
\begin{eulerprompt}
>incomes = [60, 49, 40, 61, 64, 60, 59, 54, 62, 69, 70, 42, 56, ...
>61, 61, 61, 58, 51, 48, 65, 49, 49, 41, 48, 52, 46, ...
>59, 46, 58, 43];
\end{eulerprompt}
\begin{eulercomment}
Sekarang, kami ingin menghitung rata-rata pendapatan dalam wilayah.
Menjadi sebuah program statistik, R memiliki factor dan tappy() untuk
ini.

EMT dapat membuat ini dengan menemukan indeks dari wilayah dalam
daftar unik dari wilayah.
\end{eulercomment}
\begin{eulerprompt}
>auterr=sort(unique(austates)); f=indexofsorted(auterr,austates)
\end{eulerprompt}
\begin{euleroutput}
  [6,  5,  4,  2,  2,  3,  8,  8,  4,  7,  2,  7,  4,  4,  5,  6,  5,  3,
  8,  7,  4,  2,  2,  8,  5,  1,  2,  7,  7,  1]
\end{euleroutput}
\begin{eulercomment}
Pada titik ini, kami dapat menulis fungsi putaran untuk melakukan hal
ini pada hanya faktor.

Atau kami dapat menyerupai fungsi tapply() dalam cara yang sama.
\end{eulercomment}
\begin{eulerprompt}
>function map tappl (i; f$:call, cat, x) ...
\end{eulerprompt}
\begin{eulerudf}
  u=sort(unique(cat));
  f=indexof(u,cat);
  return f$(x[nonzeros(f==indexof(u,i))]);
  endfunction
\end{eulerudf}
\begin{eulercomment}
Ini sedikit tidak efisien, karena ini menghitung wilayah unik untuk
setiap i, tetapi ini bekerja.
\end{eulercomment}
\begin{eulerprompt}
>tappl(auterr,"mean",austates,incomes)
\end{eulerprompt}
\begin{euleroutput}
  [44.5,  57.3333333333,  55.5,  53.6,  55,  60.5,  56,  52.25]
\end{euleroutput}
\begin{eulercomment}
Catatan bahwa ini bekerja untuk setiap vektor dalam wilayah.
\end{eulercomment}
\begin{eulerprompt}
>tappl(["act","nsw"],"mean",austates,incomes)
\end{eulerprompt}
\begin{euleroutput}
  [44.5,  57.3333333333]
\end{euleroutput}
\begin{eulercomment}
Sekarang, package statistikal dari EMT mendefinisikan tabel seperti
dalam R. Fungsi readtable() dan writetable() dapat digunakan untuk
input dan output.

Jadi kita dapat mencetak rata-rata pendapatan negara bagian dalam
wilayah sebuah cara yang mudah.
\end{eulercomment}
\begin{eulerprompt}
>writetable(tappl(auterr,"mean",austates,incomes),labc=auterr,wc=7)
\end{eulerprompt}
\begin{euleroutput}
      act    nsw     nt    qld     sa    tas    vic     wa
     44.5  57.33   55.5   53.6     55   60.5     56  52.25
\end{euleroutput}
\begin{eulercomment}
Sekarang dapat juga mencoba untuk meniru cara kerja dari R secara
utuh.

Faktor seharusnya dengan jelas menyimpan dalam sebuah collection
dengan tipe dan kategori (negara bagian dan wilayan dalam contoh
kita). Untuk EMT, kami menambahkan indeks pra-perhitungan.
\end{eulercomment}
\begin{eulerprompt}
>function makef (t) ...
\end{eulerprompt}
\begin{eulerudf}
  ## Factor data
  ## Returns a collection with data t, unique data, indices.
  ## See: tapply
  u=sort(unique(t));
  return \{\{t,u,indexofsorted(u,t)\}\};
  endfunction
\end{eulerudf}
\begin{eulerprompt}
>statef=makef(austates);
\end{eulerprompt}
\begin{eulercomment}
Sekarang elemen ketiga dari collection akan berisi indeks.
\end{eulercomment}
\begin{eulerprompt}
>statef[3]
\end{eulerprompt}
\begin{euleroutput}
  [6,  5,  4,  2,  2,  3,  8,  8,  4,  7,  2,  7,  4,  4,  5,  6,  5,  3,
  8,  7,  4,  2,  2,  8,  5,  1,  2,  7,  7,  1]
\end{euleroutput}
\begin{eulercomment}
Sekarang kami dapat meniru tapply() dalam cara berikut. Ini akan
mengembalikan sebuah tabel sebagai sebuah collection dari data tabel
dan headings kolom.
\end{eulercomment}
\begin{eulerprompt}
>function tapply (t:vector,tf,f$:call) ...
\end{eulerprompt}
\begin{eulerudf}
  ## Makes a table of data and factors
  ## tf : output of makef()
  ## See: makef
  uf=tf[2]; f=tf[3]; x=zeros(length(uf));
  for i=1 to length(uf);
     ind=nonzeros(f==i);
     if length(ind)==0 then x[i]=NAN;
     else x[i]=f$(t[ind]);
     endif;
  end;
  return \{\{x,uf\}\};
  endfunction
\end{eulerudf}
\begin{eulercomment}
Kita tidak menambahkan banyak pemeriksaan tipe disini. Hanya
pendekatan pencegahan kategori (faktor-faktor) tanpa data. Tetapi yang
seharusnya diperiksa untuk panjang yang benar dari t dan untuk
kebenaran dari tf collection.

Tabel ini dapat dicetak sebagai sebuah tabel dengan writetable().
\end{eulercomment}
\begin{eulerprompt}
>writetable(tapply(incomes,statef,"mean"),wc=7)
\end{eulerprompt}
\begin{euleroutput}
      act    nsw     nt    qld     sa    tas    vic     wa
     44.5  57.33   55.5   53.6     55   60.5     56  52.25
\end{euleroutput}
\eulerheading{Arrays}
\begin{eulercomment}
EMT hanya memiliki dua dimensi untuk array. Tipe data ini disebut
matriks. Ini akan lebih mudah untuk menulis fungsi untuk dimensi yang
lebih tinggi atau sebuah library C untuk ini, namun

R memiliki lebih dari dua dimensi. Dalam R array adalah sebuah vektor
dengan bidang dimensi.

Dalam EMT, sebuah vektor adalah sebuah matriks dengan satu baris. Ini
dapat dibuat sebuah matriks dengan redim().
\end{eulercomment}
\begin{eulerprompt}
>shortformat; X=redim(1:20,4,5)
\end{eulerprompt}
\begin{euleroutput}
          1         2         3         4         5 
          6         7         8         9        10 
         11        12        13        14        15 
         16        17        18        19        20 
\end{euleroutput}
\begin{eulercomment}
Ekstrasi baris dan kolom, atau sub-matriks, seperti dalam R.
\end{eulercomment}
\begin{eulerprompt}
>X[,2:3]
\end{eulerprompt}
\begin{euleroutput}
          2         3 
          7         8 
         12        13 
         17        18 
\end{euleroutput}
\begin{eulercomment}
Namun, dalam R ini memungkinkan untuk menngatur sebuah baris dari
indeks spesifik dari vektor ke sebuah nilai. Kemungkinan sama dalam
EMT hanya dengan sebuah putaran.
\end{eulercomment}
\begin{eulerprompt}
>function setmatrixvalue (M, i, j, v) ...
\end{eulerprompt}
\begin{eulerudf}
  loop 1 to max(length(i),length(j),length(v))
     M[i\{#\},j\{#\}] = v\{#\};
  end;
  endfunction
\end{eulerudf}
\begin{eulercomment}
Kami mendemonstrasikan ini untuk menunjukkan bahwa matriks dilalui
dalam EMT. Jika kamu tidak ingin mengubah matriks asli M, kamu harus
menyalinnya dalam fungsi.
\end{eulercomment}
\begin{eulerprompt}
>setmatrixvalue(X,1:3,3:-1:1,0); X,
\end{eulerprompt}
\begin{euleroutput}
          1         2         0         4         5 
          6         0         8         9        10 
          0        12        13        14        15 
         16        17        18        19        20 
\end{euleroutput}
\begin{eulercomment}
Perkalian luar dalam EMT hanya dapat dilakukan antara vektor-vektor.
Ini secara otomatis karena bahasa matriks. Satu vektor butuh menjadi
sebuah kolom vektor dan yang lainnya sebuah vektor baris.
\end{eulercomment}
\begin{eulerprompt}
>(1:5)*(1:5)'
\end{eulerprompt}
\begin{euleroutput}
          1         2         3         4         5 
          2         4         6         8        10 
          3         6         9        12        15 
          4         8        12        16        20 
          5        10        15        20        25 
\end{euleroutput}
\begin{eulercomment}
Dalam PDF pengantar untuk R disini terdapat sebuah contoh, yang mana
menghitung distribusi dari ab-cd untuk a,b,c,d dipilih dari 0 ke n
secara acak. Solusi dalam R adalah bentuk sebuah matriks 4-dimensi dan
menjalankan table() setelah itu.

Tentu saja, ini dapat tercapai dengan sebuah putaran. Tetapi loop
tidak efektif dalam EMT atau R. Dalam EMT, kami dapat menuliskan loop
dalam C dan akan menjadi solusi tercepat.

Tetapi kami ingin menyerupai perilaku dari R. Untuk ini, kami butuh
untuk meratakan perkalian ab dan membuat sebuah matriks dari ab-cd.
\end{eulercomment}
\begin{eulerprompt}
>a=0:6; b=a'; p=flatten(a*b); q=flatten(p-p'); ...
>u=sort(unique(q)); f=getmultiplicities(u,q); ...
>statplot(u,f,"h"):
\end{eulerprompt}
\eulerimg{15}{images/Hamemayu Hayuningrat_23030630053_EMTStatistika-038.png}
\begin{eulercomment}
Disamping perkalian eksak, EMT dapat menghitung frekuensi dalam
vektor-vektor.
\end{eulercomment}
\begin{eulerprompt}
>getfrequencies(q,-50:10:50)
\end{eulerprompt}
\begin{euleroutput}
  [0,  23,  132,  316,  602,  801,  333,  141,  53,  0]
\end{euleroutput}
\begin{eulercomment}
Cara termudah untuk plot ini seperti sebuah distribusi berikut.
\end{eulercomment}
\begin{eulerprompt}
>plot2d(q,distribution=11):
\end{eulerprompt}
\eulerimg{15}{images/Hamemayu Hayuningrat_23030630053_EMTStatistika-039.png}
\begin{eulercomment}
Tetapi ini juga dapat untuk pra-komputasi menghitung dalam interval
yang dipilih sebelumnya. Tentu saja, berikut menggunakan
getfrequencies() secara internal.

Karena fungsi histo() mengembalikan frekuensi, kami butuh untuk
men-sklakan ini jadi bahwa integral dibawah grafik batang adalah 1.
\end{eulercomment}
\begin{eulerprompt}
>\{x,y\}=histo(q,v=-55:10:55); y=y/sum(y)/differences(x); ...
>plot2d(x,y,>bar,style="/"):
\end{eulerprompt}
\eulerimg{15}{images/Hamemayu Hayuningrat_23030630053_EMTStatistika-040.png}
\eulerheading{Lists}
\begin{eulercomment}
EMT memiliki dua macam list. Satu dari list global yang mana mutable
dan lainnya adalah sebuah tipe list yang mana immutable. Kami tidak
peduli tentang list global disini.

Tipe list immutable dinamakan sebuah collection dalam EMT. Ini
berperilaku seperti struktur dalam C, tetapi elemen-elemenya hanya
diberi angka dan tidak dinamai.
\end{eulercomment}
\begin{eulerprompt}
>L=\{\{"Fred","Flintstone",40,[1990,1992]\}\}
\end{eulerprompt}
\begin{euleroutput}
  Fred
  Flintstone
  40
  [1990,  1992]
\end{euleroutput}
\begin{eulercomment}
Sekarang elemen-elemen tidak memiliki nama-nama, melalui nama dapat
diatur untuk tujuan khusus. Mereka dapat diakses dengan angka.
\end{eulercomment}
\begin{eulerprompt}
>(L[4])[2]
\end{eulerprompt}
\begin{euleroutput}
  1992
\end{euleroutput}
\begin{eulercomment}
\begin{eulercomment}
\eulerheading{Input dan Output File (Membaca dan Menulis Data)}
\begin{eulercomment}
Kamu akan sering ingin untuk import sebuah matriks data dari sumber
lainnya ke EMT. Tutorial ini memberitahu kamu tentang banyak cara
untuk meraihnya. Fungsi sederhana adalah writematrix() dan
readmatrix().

Mari kita demonstrasikan bagaimana membaca dan menulis sebuah vektor
ke file yang sebenarnya.
\end{eulercomment}
\begin{eulerprompt}
>a=random(1,100); mean(a), dev(a),
\end{eulerprompt}
\begin{euleroutput}
  0.49815
  0.28037
\end{euleroutput}
\begin{eulercomment}
Untuk menulis data ke sebuah file, kami menggunakan fungsi
writematrix().

Karena pengantar ini banyak seperti dalam direktori, dimana pengguna
tidak memiliki akses, kami menulis data ke direktori pengguna home.
Untuk notebook sendiri, ini tidak dibutuhkan, karena file data akan
ditulis kedalam direktori yang sama.
\end{eulercomment}
\begin{eulerprompt}
>filename="test.dat";
\end{eulerprompt}
\begin{eulercomment}
Sekarang kami menulis vektor kolom a' kedalam file. Ini menghasilkan
angka satu dalam setiap bari dari file.
\end{eulercomment}
\begin{eulerprompt}
>writematrix(a',filename);
\end{eulerprompt}
\begin{eulercomment}
Untuk membaca data, kami gunakan readmatrix().
\end{eulercomment}
\begin{eulerprompt}
>a=readmatrix(filename)';
\end{eulerprompt}
\begin{eulercomment}
Dan menghapus file.
\end{eulercomment}
\begin{eulerprompt}
>fileremove(filename);
>mean(a), dev(a),
\end{eulerprompt}
\begin{euleroutput}
  0.49815
  0.28037
\end{euleroutput}
\begin{eulercomment}
Fungsi writematrix() atau writetable() dapat diatur untuk bahasa lain.

Sebagai contoh, jika kamu menggunakan sistem Indonesia (titik desimal
dengan koma), Excel kamu harusnya memiliki nilai dengan desimal koba
dipisahkan dengan titik koma dalam sebuah csv (secara bawaan koma
memisahkan nilai) file. File berikut "test.csv" harusnya tampak dalam
folder kamu sekarang.
\end{eulercomment}
\begin{eulerprompt}
>filename="test.csv"; ...
>writematrix(random(5,3),file=filename,separator=",");
\end{eulerprompt}
\begin{eulercomment}
Kamu sekarang dapat membuka file ini dengan Excel Indonesia secara
langsung.
\end{eulercomment}
\begin{eulerprompt}
>fileremove(filename);
\end{eulerprompt}
\begin{eulercomment}
Terkadang kami memiliki strings dengan token seperti berikut.
\end{eulercomment}
\begin{eulerprompt}
>s1:="f m m f m m m f f f m m f";  ...
>s2:="f f f m m f f";
\end{eulerprompt}
\begin{eulercomment}
Untuk membuat token ini, kami mendefinisikan sebuah vektor dari
token-token.
\end{eulercomment}
\begin{eulerprompt}
>tok:=["f","m"]
\end{eulerprompt}
\begin{euleroutput}
  f
  m
\end{euleroutput}
\begin{eulercomment}
Lalu kami dapat menghitung angka dari banyaknya token yang muncul
dalam string, dan mengambil hasilnya kedalam sebuah tabel.
\end{eulercomment}
\begin{eulerprompt}
>M:=getmultiplicities(tok,strtokens(s1))_ ...
>  getmultiplicities(tok,strtokens(s2));
\end{eulerprompt}
\begin{eulercomment}
Tulis tabel dengan header token.
\end{eulercomment}
\begin{eulerprompt}
>writetable(M,labc=tok,labr=1:2,wc=8)
\end{eulerprompt}
\begin{euleroutput}
                 f       m
         1       6       7
         2       5       2
\end{euleroutput}
\begin{eulercomment}
Untuk statistik, EMT dapat membaca dan menulis tabel.
\end{eulercomment}
\begin{eulerprompt}
>file="test.dat"; open(file,"w"); ...
>writeln("A,B,C"); writematrix(random(3,3)); ...
>close();
\end{eulerprompt}
\begin{eulercomment}
File terlihat seperti ini.
\end{eulercomment}
\begin{eulerprompt}
>printfile(file)
\end{eulerprompt}
\begin{euleroutput}
  A,B,C
  0.7003664386138074,0.1875530821001213,0.3262339279660414
  0.5926249243193858,0.1522927283984059,0.368140583062521
  0.8065535209872989,0.7265910840408142,0.7332619844597152
  
\end{euleroutput}
\begin{eulercomment}
fungsi readtable() dalam bentuk yang sederhana dapat membaca ini dan
mengembalikan sebuah collection dari nilai nilai dan garis heading.
\end{eulercomment}
\begin{eulerprompt}
>L=readtable(file,>list);
\end{eulerprompt}
\begin{eulercomment}
Collection ini dapat mencetak dengan writetable() untuk notebook atau
ke dalam sebuah file.
\end{eulercomment}
\begin{eulerprompt}
>writetable(L,wc=10,dc=5)
\end{eulerprompt}
\begin{euleroutput}
           A         B         C
     0.70037   0.18755   0.32623
     0.59262   0.15229   0.36814
     0.80655   0.72659   0.73326
\end{euleroutput}
\begin{eulercomment}
Nilai-nilai matriks adalah elemen pertama dari L. Catatan bahwa mean()
dalam EMT menghitung nilai mean dari baris-baris dari sebuah matriks.
\end{eulercomment}
\begin{eulerprompt}
>mean(L[1])
\end{eulerprompt}
\begin{euleroutput}
    0.40472 
    0.37102 
    0.75547 
\end{euleroutput}
\eulerheading{File-File CSV}
\begin{eulercomment}
Pertama, mari kita tulis sebuah matriks kedalam sebuah file. Untuk
keluaran, kami menghasilkan sebuah file dalam direktori sekarang yang
dikerjakan.
\end{eulercomment}
\begin{eulerprompt}
>file="test.csv";  ...
>M=random(3,3); writematrix(M,file);
\end{eulerprompt}
\begin{eulercomment}
Here is the content of this file.
\end{eulercomment}
\begin{eulerprompt}
>printfile(file)
\end{eulerprompt}
\begin{euleroutput}
  0.8221197733097619,0.821531098722547,0.7771240608094004
  0.8482947121863489,0.3237767724883862,0.6501422353377985
  0.1482301827518109,0.3297459716109594,0.6261901074210923
  
\end{euleroutput}
\begin{eulercomment}
CSV ini dapat dibuka dalam sistem English kedalam Excel dengan sebuah
klik dobel. Jika kamu mendapatkan sebuah file dalam sistem German,
perlu untuk mengimpor data ke Excel untuk berhati hati dalam titik
desimal.

Tetapi titik desimal adalah format bawaan untuk EMT juga. Kamu dapat
membaca matriks dari sebuah file dengan readmatrix().
\end{eulercomment}
\begin{eulerprompt}
>readmatrix(file)
\end{eulerprompt}
\begin{euleroutput}
    0.82212   0.82153   0.77712 
    0.84829   0.32378   0.65014 
    0.14823   0.32975   0.62619 
\end{euleroutput}
\begin{eulercomment}
Ini memungkinkan untuk menulis beberapa matriks kedalam satu file.
Perintah open() dapat membukan file untuk menulisnya dengan parameter
"w". Secara bawaan adalah "r" untuk membaca.
\end{eulercomment}
\begin{eulerprompt}
>open(file,"w"); writematrix(M); writematrix(M'); close();
\end{eulerprompt}
\begin{eulercomment}
Matriks-matriks dipisahkan dengan baris kosong. Untuk membaca
matriks-matriks, buka file dan panggil readmatrix() beberapa kali.
\end{eulercomment}
\begin{eulerprompt}
>open(file); A=readmatrix(); B=readmatrix(); A==B, close();
\end{eulerprompt}
\begin{euleroutput}
          1         0         0 
          0         1         0 
          0         0         1 
\end{euleroutput}
\begin{eulercomment}
Dalam Excel atau spreadsheet yang sama, kamu dapat mengexport sebuah
matriks sebagai CSV(comma separated values). Dalam Excel 2007, gunakan
"save as" dan "other formats", lalu pilih "CSV". Pastikan, tabel
sekarang hanya memiliki daya yang ingin kamu ekspor.

Ini sebagai contohnya.
\end{eulercomment}
\begin{eulerprompt}
>printfile("excel-data.csv")
\end{eulerprompt}
\begin{euleroutput}
  0;1000;1000
  1;1051,271096;1072,508181
  2;1105,170918;1150,273799
  3;1161,834243;1233,67806
  4;1221,402758;1323,129812
  5;1284,025417;1419,067549
  6;1349,858808;1521,961556
  7;1419,067549;1632,31622
  8;1491,824698;1750,6725
  9;1568,312185;1877,610579
  10;1648,721271;2013,752707
\end{euleroutput}
\begin{eulercomment}
Seperti yang dapat kamu lihat, sister German saya menggunakan titik
koma sebagai pemisah dan sebuah koma desimal. Kamu dapat merubah ini
dalam perarturan sistem atau dalam Excel, tetapi ini tidak dibutuhkan
untuk membaca matriks ke dalam EMT.

Cara termudah untuk membaca ini kedalam Euler adalah readmatrix().
Semua komma digantikan dengan titik dengan parameter \textgreater{}comma. Untuk CSV
English, secara sederhana buang parameter ini.
\end{eulercomment}
\begin{eulerprompt}
>M=readmatrix("excel-data.csv",>comma)
\end{eulerprompt}
\begin{euleroutput}
          0      1000      1000 
          1    1051.3    1072.5 
          2    1105.2    1150.3 
          3    1161.8    1233.7 
          4    1221.4    1323.1 
          5      1284    1419.1 
          6    1349.9      1522 
          7    1419.1    1632.3 
          8    1491.8    1750.7 
          9    1568.3    1877.6 
         10    1648.7    2013.8 
\end{euleroutput}
\begin{eulercomment}
Mari kita buat plot ini.
\end{eulercomment}
\begin{eulerprompt}
>plot2d(M'[1],M'[2:3],>points,color=[red,green]'):
\end{eulerprompt}
\eulerimg{15}{images/Hamemayu Hayuningrat_23030630053_EMTStatistika-041.png}
\begin{eulercomment}
Banyak cara yang lebih mendasar untuk membaca data dari sebuah file.
Kamu dapat membuka file dan membaca angka per baris. Fungsi
getvectorline() akan membaca angka-angka dari sebuah baris dari data.
Secara bawaan, ini mempunyai harapan sebuah titik desimal. Tetapi ini
dapat juga menggunakan sebuah koma desimal, jika kamu memanggil
setdecialdot(",") sebelum kamu menggunakan fungsi ini.

Fungsi berikut merupakan sebuah contoh untuk  ini. Ini akan berhenti
pada akhir file atau baris kosong.
\end{eulercomment}
\begin{eulerprompt}
>function myload (file) ...
\end{eulerprompt}
\begin{eulerudf}
  open(file);
  M=[];
  repeat
     until eof();
     v=getvectorline(3);
     if length(v)>0 then M=M_v; else break; endif;
  end;
  return M;
  close(file);
  endfunction
\end{eulerudf}
\begin{eulerprompt}
>myload(file)
\end{eulerprompt}
\begin{euleroutput}
    0.82212   0.82153   0.77712 
    0.84829   0.32378   0.65014 
    0.14823   0.32975   0.62619 
\end{euleroutput}
\begin{eulercomment}
Ini juga akan memungkinkan membaca semua angka dalam file tersebut
dengan getvector().
\end{eulercomment}
\begin{eulerprompt}
>open(file); v=getvector(10000); close(); redim(v[1:9],3,3)
\end{eulerprompt}
\begin{euleroutput}
    0.82212   0.82153   0.77712 
    0.84829   0.32378   0.65014 
    0.14823   0.32975   0.62619 
\end{euleroutput}
\begin{eulercomment}
Lalu ini sangat mudah untuk menyimpan sebuah vektor dari nilai-nilai,
satu nilai setiap baris dan membaca kembali vektor ini.
\end{eulercomment}
\begin{eulerprompt}
>v=random(1000); mean(v)
\end{eulerprompt}
\begin{euleroutput}
  0.50303
\end{euleroutput}
\begin{eulerprompt}
>writematrix(v',file); mean(readmatrix(file)')
\end{eulerprompt}
\begin{euleroutput}
  0.50303
\end{euleroutput}
\eulerheading{Menggunakan Tabel Tabel}
\begin{eulercomment}
Tabel dapat digunakan untuk membaca atau menulis data numerik. Sebagai
contoh, kami menulis sebuah tabel dengan baris dan headers kolom ke
sebuah file.
\end{eulercomment}
\begin{eulerprompt}
>file="test.tab"; M=random(3,3);  ...
>open(file,"w");  ...
>writetable(M,separator=",",labc=["one","two","three"]);  ...
>close(); ...
>printfile(file)
\end{eulerprompt}
\begin{euleroutput}
  one,two,three
        0.09,      0.39,      0.86
        0.39,      0.86,      0.71
         0.2,      0.02,      0.83
\end{euleroutput}
\begin{eulercomment}
Ini dapat diimpor ke Excel.

Untuk membaca file dalam EMT, kami menggunakan readtable().
\end{eulercomment}
\begin{eulerprompt}
>\{M,headings\}=readtable(file,>clabs); ...
>writetable(M,labc=headings)
\end{eulerprompt}
\begin{euleroutput}
         one       two     three
        0.09      0.39      0.86
        0.39      0.86      0.71
         0.2      0.02      0.83
\end{euleroutput}
\eulerheading{Menganalisis Sebuah Garis}
\begin{eulercomment}
Kamu bahkan dapat mengevaluasi setiap baris dengan tangan. Andaikan,
kami memiliki baris dengan format berikut.
\end{eulercomment}
\begin{eulerprompt}
>line="2020-11-03,Tue,1'114.05"
\end{eulerprompt}
\begin{euleroutput}
  2020-11-03,Tue,1'114.05
\end{euleroutput}
\begin{eulercomment}
Pertama kami dapat membuat token untuk baris.
\end{eulercomment}
\begin{eulerprompt}
>vt=strtokens(line)
\end{eulerprompt}
\begin{euleroutput}
  2020-11-03
  Tue
  1'114.05
\end{euleroutput}
\begin{eulercomment}
Lalu kami dapat mengevaluasi setiap elemen dari baris menggunakan
evaluasi yang dibutuhkan.
\end{eulercomment}
\begin{eulerprompt}
>day(vt[1]),  ...
>indexof(["mon","tue","wed","thu","fri","sat","sun"],tolower(vt[2])),  ...
>strrepl(vt[3],"'","")()
\end{eulerprompt}
\begin{euleroutput}
  7.3816e+05
  2
  1114
\end{euleroutput}
\begin{eulercomment}
Menggunakan ekspresi reguler, ini memungkinkan untuk mengekstrak
hampir sebarang informasi dari sebuah baris data.

Asumsikan kita memiliki baris berikut dalam dokumen HTML.
\end{eulercomment}
\begin{eulerprompt}
>line="<tr><td>1145.45</td><td>5.6</td><td>-4.5</td><tr>"
\end{eulerprompt}
\begin{euleroutput}
  <tr><td>1145.45</td><td>5.6</td><td>-4.5</td><tr>
\end{euleroutput}
\begin{eulercomment}
Untuk mengekstrak ini, kami menggunakan regular expression, yang mana
mencari

\end{eulercomment}
\begin{eulerttcomment}
 - sebuah braket tertutup >,
 - sebarang string yang tidak memiliki bracket dengan sebuah sub-cocok
\end{eulerttcomment}
\begin{eulercomment}
"(...)",\\
\end{eulercomment}
\begin{eulerttcomment}
 - sebuah pembuka dan penutup bracket menggunakan solusi terpendek,
 - kembali sebarang string uang tidak memiliki bracket,
 - dan sebuah bracket terbuka <.
\end{eulerttcomment}
\begin{eulercomment}

Regular expression merupakan hal yang sulit untuk dipelajari tetapi
sangat kuat.
\end{eulercomment}
\begin{eulerprompt}
>\{pos,s,vt\}=strxfind(line,">([^<>]+)<.+?>([^<>]+)<");
\end{eulerprompt}
\begin{eulercomment}
Hasil merupakan posisi yang cocok, string yang cocok dan sebuah vektor
string untuk sub-cocok.
\end{eulercomment}
\begin{eulerprompt}
>for k=1:length(vt); vt[k](), end;
\end{eulerprompt}
\begin{euleroutput}
  1145.5
  5.6
\end{euleroutput}
\begin{eulercomment}
Ini merupakan sebuah fungsi, yang mana membaca bagian bagian numerik
diantara \textless{}td\textgreater{} dan \textless{}/td\textgreater{}.
\end{eulercomment}
\begin{eulerprompt}
>function readtd (line) ...
\end{eulerprompt}
\begin{eulerudf}
  v=[]; cp=0;
  repeat
     \{pos,s,vt\}=strxfind(line,"<td.*?>(.+?)</td>",cp);
     until pos==0;
     if length(vt)>0 then v=v|vt[1]; endif;
     cp=pos+strlen(s);
  end;
  return v;
  endfunction
\end{eulerudf}
\begin{eulerprompt}
>readtd(line+"<td>non-numerical</td>")
\end{eulerprompt}
\begin{euleroutput}
  1145.45
  5.6
  -4.5
  non-numerical
\end{euleroutput}
\eulerheading{Membaca dari Web}
\begin{eulercomment}
Sebuah web site atau sebuah file dengan sebuah URL dapat dibuka dalam
EMT dan dapat dibaca baris per baris.

Dalam contoh, kami membaca versi sekarang dari situs EMT. Kami
menggunakan regular expression untuk mengamati "Version ..." dalam
sebuah heading.
\end{eulercomment}
\begin{eulerprompt}
>function readversion () ...
\end{eulerprompt}
\begin{eulerudf}
  urlopen("http://www.euler-math-toolbox.de/Programs/Changes.html");
  repeat
    until urleof();
    s=urlgetline();
    k=strfind(s,"Version ",1);
    if k>0 then substring(s,k,strfind(s,"<",k)-1), break; endif;
  end;
  urlclose();
  endfunction
\end{eulerudf}
\begin{eulerprompt}
>readversion
\end{eulerprompt}
\begin{euleroutput}
  Version 2024-01-12
\end{euleroutput}
\eulerheading{Variabel Input dan Output}
\begin{eulercomment}
Kamu dapat menulis sebuah variabel dalam bentuk definisi Euler kedalam
file atau ke baris perintah.
\end{eulercomment}
\begin{eulerprompt}
>writevar(pi,"mypi");
\end{eulerprompt}
\begin{euleroutput}
  mypi = 3.141592653589793;
\end{euleroutput}
\begin{eulercomment}
Untuk sebuah pengujian, kami membuat sebuah file Euler dalam direktori
bekerja dari EMT.
\end{eulercomment}
\begin{eulerprompt}
>file="test.e"; ...
>writevar(random(2,2),"M",file); ...
>printfile(file,3)
\end{eulerprompt}
\begin{euleroutput}
  M = [ ..
  0.5991820585590205, 0.7960280262224293;
  0.5167243983231363, 0.2996684599070898];
\end{euleroutput}
\begin{eulercomment}
Kami sekarang dapat memuat file. Ini dapat didefinisikan matriks M.
\end{eulercomment}
\begin{eulerprompt}
>load(file); show M,
\end{eulerprompt}
\begin{euleroutput}
  M = 
    0.59918   0.79603 
    0.51672   0.29967 
\end{euleroutput}
\begin{eulercomment}
Omong-omong, jika writevar() digunakan dalam sebuah variabel, ini akan
mencetak definisi variabel dengan nama dari variabel ini.
\end{eulercomment}
\begin{eulerprompt}
>writevar(M); writevar(inch$)
\end{eulerprompt}
\begin{euleroutput}
  M = [ ..
  0.5991820585590205, 0.7960280262224293;
  0.5167243983231363, 0.2996684599070898];
  inch$ = 0.0254;
\end{euleroutput}
\begin{eulercomment}
Kami dapat juga membukan file baru atau menambahkan file yang telah
ada. Dalam contoh kami menambahkkan file yang dibuat sebelumnya.
\end{eulercomment}
\begin{eulerprompt}
>open(file,"a"); ...
>writevar(random(2,2),"M1"); ...
>writevar(random(3,1),"M2"); ...
>close();
>load(file); show M1; show M2;
\end{eulerprompt}
\begin{euleroutput}
  M1 = 
    0.30287   0.15372 
     0.7504   0.75401 
  M2 = 
    0.27213 
   0.053211 
    0.70249 
\end{euleroutput}
\begin{eulercomment}
Untuk menghapus sebarang file gunakan fileremove().
\end{eulercomment}
\begin{eulerprompt}
>fileremove(file);
\end{eulerprompt}
\begin{eulercomment}
vektor baris dalam sebuah file tidak butuh koma, jika setiap angka
dalam baris baru. Mari kita buat sedemikian hingga sebuah file,
menulis setiap baris satu per satu dengan writeln().
\end{eulercomment}
\begin{eulerprompt}
>open(file,"w"); writeln("M = ["); ...
>for i=1 to 5; writeln(""+random()); end; ...
>writeln("];"); close(); ...
>printfile(file)
\end{eulerprompt}
\begin{euleroutput}
  M = [
  0.344851384551
  0.0807510017715
  0.876519562911
  0.754157709472
  0.688392638934
  ];
\end{euleroutput}
\begin{eulerprompt}
>load(file); M
\end{eulerprompt}
\begin{euleroutput}
  [0.34485,  0.080751,  0.87652,  0.75416,  0.68839]
\end{euleroutput}
\eulerheading{Latihan}
\begin{eulercomment}
\end{eulercomment}
\eulersubheading{Latihan 1}
\begin{eulercomment}
diberikan lemparan dadu acak sebanyak 1200 kali, tentukan nilai
chi-kuadratnya apabila nilai harapan dadunya memiliki distribusi
seragam
\end{eulercomment}
\begin{eulerprompt}
>throws=1200; sides=6;
>randomDice = intrandom(throws, sides);
>freqDice = multofsorted(sort(randomDice), 1:sides);
>expectDice = dup(throws/sides, sides)';
>chitest(freqDice, expectDice)
\end{eulerprompt}
\begin{euleroutput}
  0.91075
\end{euleroutput}
\begin{eulercomment}
Karena error memiliki nilai yang besar (lebih dari 0.05) maka dapat
disimpulkan bahwa kita tidak bisa menolak distribusi seragam. Ini
belum membuktikan bahwa dadu kita adil tetapi kita tidak bisa menolak
hipotesis kita
\end{eulercomment}
\eulersubheading{Latihan 2:}
\begin{eulercomment}
Diberikan hasil dari lemparan koin sebagai berikut, dengan G adalah
gambar dan A adalah angka

G, A, A, G, A, A, A, G, A, A

Apakah koin tersebut adil?
\end{eulercomment}
\begin{eulerprompt}
>coinRes = "G, A, A, G, A, A, A, G, A, A";
\end{eulerprompt}
\begin{eulercomment}
buat fungsi untuk memisahkan setiap token
\end{eulercomment}
\begin{eulerprompt}
>function replaceTokenToArr(strToken, separateToken, arrToken, arrSubsToken)...
\end{eulerprompt}
\begin{eulerudf}
  tokens = strtokens(strToken, separateToken);
  transformedArr = [];
  for i = 1 to length(tokens) step i
     token = tokens[i];
     indexToken = indexof(arrToken, token);
     subs = arrSubsToken[indexToken];
     transformedArr = [transformedArr, subs];
  end;
  return transformedArr;
  endfunction
\end{eulerudf}
\begin{eulerprompt}
>coinDis = replaceTokenToArr(coinRes, ", ", ["G", "A"], [1, 2])
\end{eulerprompt}
\begin{euleroutput}
  [1,  2,  2,  1,  2,  2,  2,  1,  2,  2]
\end{euleroutput}
\begin{eulercomment}
Tambahkan fungsi untuk menghitung setiap frekuensi
\end{eulercomment}
\begin{eulerprompt}
>function getFrequencies(arr)...
\end{eulerprompt}
\begin{eulerudf}
  uniqueArr = unique(arr);
  sortArr = sort(arr);
  firstIndexes = [];
  freqs = [];
  for i = 1 to length(uniqueArr) step i
     currUnique = uniqueArr[i];
     currFirstIndex = indexof(sortArr, currUnique) - 1;
     firstIndexes = [firstIndexes,  currFirstIndex];
  end;
  firstIndexes = [firstIndexes, length(arr)];
  for i = 1 to (length(firstIndexes) - 1) step i
     currIndex = firstIndexes[i];
     nextIndex = firstIndexes[i + 1];
     currFreq = nextIndex - currIndex;
     freqs = [freqs, currFreq];
  end;
  return freqs;
  endfunction
\end{eulerudf}
\begin{eulerprompt}
>coinFreq = getFrequencies(coinDis)
\end{eulerprompt}
\begin{euleroutput}
  [3,  7]
\end{euleroutput}
\begin{eulercomment}
Didapatkan frekuensi untuk A sebanyak 3 dan G sebanyak 7
\end{eulercomment}
\begin{eulerprompt}
>coinExpected = dup(sum(coinFreq)/length(coinFreq), length(coinFreq))'
\end{eulerprompt}
\begin{euleroutput}
  [5,  5]
\end{euleroutput}
\begin{eulerprompt}
>chitest(coinFreq, coinExpected)
\end{eulerprompt}
\begin{euleroutput}
  0.2059
\end{euleroutput}
\begin{eulercomment}
Karena nilai error dari chi-kuadratnya besar (lebh dari 0.05), maka
kita dapat menolak hipotesis bahwa koin tersebut adil (setimbang).
\end{eulercomment}
\eulersubheading{Latihan 3}
\begin{eulercomment}
Diberikan tabel yang menyatakan hubungan kondisi keluarga dengan
perilaku deliquensi anak pada sebuah SD

Perilaku Deliquensi \textbar{} Nakal \textbar{} Nurut\\
Harmonis            \textbar{} 25    \textbar{} 85\\
Broken              \textbar{} 75    \textbar{} 15

Apakah terdapat hubungan antara kondisi keluarga dengan perilaku
deliquensi anak?
\end{eulercomment}
\begin{eulerprompt}
>deliquencyChilds = [25, 85; 75, 15];
>writetable(deliquencyChilds, wc=10, labr=["Harmonis", "Broken"], labc=["Nakal", "Nurut"])
\end{eulerprompt}
\begin{euleroutput}
                 Nakal     Nurut
    Harmonis        25        85
      Broken        75        15
\end{euleroutput}
\begin{eulercomment}
Periksa terlebih dahulu dengan uji tabel chi-kuadrat
\end{eulercomment}
\begin{eulerprompt}
>tabletest(deliquencyChilds)
\end{eulerprompt}
\begin{euleroutput}
  0
\end{euleroutput}
\begin{eulercomment}
Dimana hasilnya kecil, kita tidak bisa menentukan apakah menolak
saling asing antara data yang diberikan.

Lalu, dapat dibuat tabel harapan dari matriks tersebut, untuk melihat
harapan yang ada dalam setiap barisnya.
\end{eulercomment}
\begin{eulerprompt}
>writetable(expectedtable(deliquencyChilds), wc=10, dc=1, labr=["Harmonis", "Broken"], labc=["Nakal", "Nurut"])
\end{eulerprompt}
\begin{euleroutput}
                 Nakal     Nurut
    Harmonis        55        55
      Broken        45        45
\end{euleroutput}
\begin{eulercomment}
Selanjutnya, menghitung nilai koefisien kontingensi dari tabel
tersebut
\end{eulercomment}
\begin{eulerprompt}
>contingency(deliquencyChilds)
\end{eulerprompt}
\begin{euleroutput}
  0.7303
\end{euleroutput}
\eulersubheading{Latihan 4}
\begin{eulercomment}
Penelitian dilakukan untuk mengetahui partisipasi warga dalam suatu
kepala desa yang dilihat dari jenis kelamin. Hasil penelitian tersebut
diperoleh data sebagai berikut

Jenis Kelamin \textbar{} Ikut \textbar{} Tidak\\
Pria          \textbar{} 10   \textbar{} 5\\
Wanita        \textbar{} 9    \textbar{} 6

Apakah ada pengaruh jenis kelamin terhadap keikutseraan dalam
pemilihan kepala desa?
\end{eulercomment}
\begin{eulerprompt}
>absenceGenders = [10,5;9,6]
\end{eulerprompt}
\begin{euleroutput}
         10         5 
          9         6 
\end{euleroutput}
\begin{eulercomment}
Buat tabel agar mudah melihatnya
\end{eulercomment}
\begin{eulerprompt}
>writetable(absenceGenders, wc = 10, labr=["Pria", "Wanita"], labc = ["Ikut", "Tidak"])
\end{eulerprompt}
\begin{euleroutput}
                  Ikut     Tidak
        Pria        10         5
      Wanita         9         6
\end{euleroutput}
\begin{eulercomment}
Memeriksa tabel untuk uji chi-kuadratnya
\end{eulercomment}
\begin{eulerprompt}
>tabletest(absenceGenders)
\end{eulerprompt}
\begin{euleroutput}
  0.70479
\end{euleroutput}
\begin{eulercomment}
Karena nilai terlalu besar, maka belum bisa menyimpulkan apakah
terdapat hubungan antara keikutsertaan dengan gender

Lalu, dibuat tabel nilai harapan untuk melihat nilai harapan pada
tabel tersebut
\end{eulercomment}
\begin{eulerprompt}
>writetable(expectedtable(absenceGenders), dc=1, wc=10, labr=["Pria", "Wanita"], labc=["Ikut", "Tidak"])
\end{eulerprompt}
\begin{euleroutput}
                  Ikut     Tidak
        Pria       9.5       5.5
      Wanita       9.5       5.5
\end{euleroutput}
\begin{eulercomment}
Terakhir, adalah menghitung nilai koefisien kontingensi
\end{eulercomment}
\begin{eulerprompt}
>contingency(absenceGenders)
\end{eulerprompt}
\begin{euleroutput}
  0.09759
\end{euleroutput}
\begin{eulercomment}
Karena nilai koefisien kontingensi mendekati 0, maka dapat disimpulkan
bahwa hubungan antara keikutsertaan dengan jenis kelamin tidak ada
(sangat lemah).
Terlihat bahwa koefisien korelasi kontingensinya mendekati 1, artinya
bahwa variabel kategorik ini saling bergantung dimana hubungan antara
kondisi keluarga dengan perilaku deliquensi anak memiliki hubungan
yang kuat.
\end{eulercomment}
\eulersubheading{Latihan 5}
\begin{eulercomment}
Diperkirakan rata-rata berat badan siswa kelas 2 SMA adalah kurang
dari 55 kg. Akan tetapi untuk melakukan pembuktian perkiraan tersebut
diambil data dan didapat sebagai berikut

berat badan:\\
45, 60,  65, 55, 65, 60, 50, 70, 60, 60

Apakah perkiraan ini benar?

Penyelesaian

Rumusan Hipotesis\\
Parameter populasi yang akan diuji disini adalah berat badan dari
populasi siswa kelas 2 SMA

\end{eulercomment}
\begin{eulerformula}
\[
\begin{array}{cc} \text{Hipotesis nol} & H_0: \mu\le 55. \\ \text{Hipotesis alternatif} & H_1: \mu > 55. \\ \end{array}
\]
\end{eulerformula}
\begin{eulerprompt}
>weights = [45, 60,  65, 55, 65, 60, 50, 70, 60, 60];
>meanEst = 55;
>dev(weights)
\end{eulerprompt}
\begin{euleroutput}
  7.3786
\end{euleroutput}
\begin{eulerprompt}
>ttest(mean(weights), dev(weights), length(weights), meanEst)
\end{eulerprompt}
\begin{euleroutput}
  0.06031
\end{euleroutput}
\begin{eulerformula}
\[
t_{\text{hitung}} = \dfrac{\overline{x}-\mu_0}{s/\sqrt{n}} = \dfrac{59-55}{7,\!37/\sqrt{9}} \approx 1,\!63.
\]
\end{eulerformula}
\begin{eulercomment}
Daerah kritis:\\
Uji yang dilakukan diatas merupakan uji satu arah (kanan). Dengan
menggunakan tabel-t, nilai-t, dengan taraf signifikansi 0.05 dan
derajat kebabasan n - 1 = 10 - 1 = 9, maka

\end{eulercomment}
\begin{eulerformula}
\[
t_{\text{tabel}} = t_{0,05; ~9} \approx 1,\!833.
\]
\end{eulerformula}
\begin{eulercomment}
Dengan demikian, diperoleh bahwa daerah kritis terletak  di

\end{eulercomment}
\begin{eulerformula}
\[
t > 1,\!83.
\]
\end{eulerformula}
\begin{eulercomment}
Keputusan:\\
Karena\\
\end{eulercomment}
\begin{eulerformula}
\[
t_{\text{hitung}} = 1,\!63 < 1,\!833 = t_{\text{tabel}},
\]
\end{eulerformula}
\begin{eulercomment}
disimpulkan bahwa statistik uji akan terdapat diluar daerah kritis.
Degan demikian hipotesis nol tidak ditolak

Kesimpulan:\\
Dugaan bahwa perkiraan rata-rata berat badan siswa kelas 2 SMA kurang
dari 55 kg akan di terima
\end{eulercomment}
\eulersubheading{Latihan 6}
\begin{eulercomment}
Berdasarkan 100 laporan kematian di AS yang diambil secara acak,
diperoleh bahwa rata-rata usia hidup orang AS adalah 71.8 tahun dengan
simpangan baku 8.9 tahun. Hal ini memberikan dugaan bahwa rata-rata
usia hidup orang AS lebih dari 70 tahun. Dengan taraf signifikansi 5\%
ujilah kebenaran dugaan tersebut.

Penyelesaian

Rumusan Hipotesis\\
Parameter populasi yang akan diuji  disini adalah rata-rata hidup dari
populasi orang AS

\end{eulercomment}
\begin{eulerformula}
\[
\begin{array}{cc} \text{Hipotesis nol} & H_0: \mu\le 70. \\ \text{Hipotesis alternatif} & H_1: \mu > 70. \\ \end{array}
\]
\end{eulerformula}
\begin{eulerprompt}
>nReport = 100; meanLifeExpect = 71.8; devLife = 8.9; meanLifeEstimate = 70;
>ttest(meanLifeEstimate, devLife, nReport, meanLifeExpect)
\end{eulerprompt}
\begin{euleroutput}
  0.022912
\end{euleroutput}
\begin{eulerformula}
\[
t_{\text{hitung}} = \dfrac{\overline{x}-\mu_0}{s/\sqrt{n}} = \dfrac{71,\!8-70}{8,\!9/\sqrt{100}} \approx 2,\!02.
\]
\end{eulerformula}
\begin{eulercomment}
Daerah kritis:\\
Uji yang dilakukan diatas merupakan uji satu arah (kanan). Dengan
menggunakan tabel-t, nilai-t, dengan taraf signifikansi 0.05 dan
derajat kebabasan n - 1 = 100 - 1 = 99, maka

\end{eulercomment}
\begin{eulerformula}
\[
t_{\text{tabel}} = t_{0,05; ~99} \approx 1,\!66.
\]
\end{eulerformula}
\begin{eulercomment}
Dengan demikian, diperoleh bahwa daerah kritis terletak  di

\end{eulercomment}
\begin{eulerformula}
\[
t > 1,\!66.
\]
\end{eulerformula}
\begin{eulercomment}
Keputusan:\\
Karena\\
\end{eulercomment}
\begin{eulerformula}
\[
t_{\text{hitung}} = 2,\!02 > 1,\!66 = t_{\text{tabel}},
\]
\end{eulerformula}
\begin{eulercomment}
disimpulkan bahwa statistik uji akan terdapat pada daerah kritis.Degan
demikian hipotesis nol akan ditolak

Kesimpulan:\\
Dugaan bahwa rata-rata usia hidup orang AS lebih dari 70 tahun
diterima.

Kesimpulan dari fungsi:\\
Karena nilai ttest \textless{} 0.05, maka hipotesis nol akan ditolak
\end{eulercomment}
\eulersubheading{Latihan 7}
\begin{eulercomment}
Seorang perawat ingin mengetahui apakah terdapat perbedaan antara lama
hari dirawat dengan kelas tempat dirawat, yang mana terdapat kelas I,
II, dan kelas III, sebagai berikut

Pasien \textbar{} Kelas I \textbar{} Kelas II \textbar{} Kelas III\\
1      \textbar{} 5       \textbar{} 2        \textbar{} 3\\
2      \textbar{} 3       \textbar{} 6        \textbar{} 6\\
3      \textbar{} 2       \textbar{} 4        \textbar{} 7\\
4      \textbar{} 6       \textbar{} 2        \textbar{} 3\\
5      \textbar{} 3       \textbar{}          \textbar{} 8

Penyelesaian

Rumusan Hipotesis\\
Parameter populasi yang akan diuji  disini adalah rata-rata lama
perawatan dari masing-masing kelas

\end{eulercomment}
\begin{eulerformula}
\[
\begin{array}{cc} \text{Hipotesis nol} & H_0: k_i = 0, i = k_1, k_2, k_3 \\ \text{Hipotesis alternatif} & H_1: \text{setidaknya ada } k_i \ne 0 \\ \end{array}
\]
\end{eulerformula}
\begin{eulerprompt}
>classOne = [5,3,2,6,3];
>classTwo = [2,6,4,2];
>classThree = [3,6,7,3,8];
>varanalysis(classOne, classTwo, classThree)
\end{eulerprompt}
\begin{euleroutput}
  0.32137
\end{euleroutput}
\begin{eulercomment}
Karena nilai dari fungsi \textgreater{} 0.05, maka kita terima hipotesis nol, yang
artinya tidak ada perbedaan antara lama hari dirawat dengan kelas
tempat pasien dirawat
\end{eulercomment}
\eulersubheading{Latihan 8}
\begin{eulercomment}
Terdapat tiga kelompok subjek penelitian untuk menguji metode
pengajaran mana yang paling baik dalam meningkatkan pembelajaran.
Metode pertama adalah ceramah, kedua adalah diskusi dan ketiga adalah
praktek

Penelitian tersebut menghasilkan data sebagai berikut.

Ceramah \textbar{} Diskusi \textbar{} Praktek\\
25      \textbar{} 17      \textbar{} 26\\
11      \textbar{} 16      \textbar{} 20\\
16      \textbar{} 18      \textbar{} 17\\
26      \textbar{} 20      \textbar{} 26\\
32      \textbar{} 10      \textbar{} 43\\
25      \textbar{} 14      \textbar{} 46\\
30      \textbar{} 19      \textbar{} 35\\
17                \textbar{} 34

Penyelesaian

Rumusan Hipotesis\\
Parameter populasi yang akan diuji disini adalah hasil peningkatan
pembelajaran untuk metode-metode diatas

\end{eulercomment}
\begin{eulerformula}
\[
\begin{array}{cc} \text{Hipotesis nol} & H_0: m_i = 0, i = m_1, m_2, m_3 \\ \text{Hipotesis alternatif} & H_1: \text{setidaknya ada } m_i \ne 0 \\ \end{array}
\]
\end{eulerformula}
\begin{eulerprompt}
>lecture = [25,11,16,26,32,25,30,17];
>discuss = [17,16,18,20,10,14,19];
>practice = [26,20,17,26,43,46,35,34];
>varanalysis(lecture, discuss, practice)
\end{eulerprompt}
\begin{euleroutput}
  0.0060525
\end{euleroutput}
\begin{eulercomment}
karena hasil fungsi \textless{} 0.05, maka kita tolak hipotesis nol, yang
artinya bahwa hasil peningkatan pembelajaran setelah melalui
metode-metode tersebut memiliki hasil yang berbeda.
\end{eulercomment}
\end{eulernotebook}
\end{document}
