\documentclass[a4paper,10pt]{article}
\usepackage{eumat}

\begin{document}
\begin{eulernotebook}
\eulerheading{Menggambar Grafik 2D dengan EMT}
\begin{eulercomment}
Notebook ini menjelaskan tentang cara menggambar berbagaikurva dan grafik 2D dengan software EMT. EMT
menyediakan fungsi plot2d() untuk menggambar berbagai kurva dan grafik dua dimensi (2D).

\end{eulercomment}
\eulersubheading{Plot-Plot Dasar}
\begin{eulercomment}
Ini adalah fungsi-fungsi plot yang sangat umum. Terdapat koordinat-koordinat layar, yang mana selalu berupa
selang dari 0 hingga 1024 dalam setiap sumbu, tidak peduli apakah layarnya yang berupa persegi atau tidak. Dan
disini adalah koordinat-koordinat plot, yang mana dapat diatur dengan setplot(). Pemetaan antara
koordinat-koordinat tergantung pada jendela layar. Sebagai contoh, bawaan shirnkwindow() akan mengatur jarak
untuk sumbu label-label dan sebuah judul plot.

Dalam contoh, kami hanya akan menggambar beberapa garis-garis acak dalam warna-warna yang berbeda. Untuk lebih
jelasnya dalam fungsi-fungsi tersebut, pelajari fungsi-fungsi utama dalam EMT.
\end{eulercomment}
\begin{eulerprompt}
>clg; // membersikan layar
>window(0, 0, 1024, 1024); // menggunakan semua rentang dalam jendela
>setplot(0, 1, 0, 1); // atur koordinat-koordinat plot
>hold on; // mulai dengan mode overwrite
>n = 100; X = random(n, 2); Y = random(n,2); // membuat titik-titik acak
>colors = rgb(random(n), random(n), random(n)); // mendapatkan warna-warna acak
>loop 1 to n; color(colors[#]); plot(X[#], Y[#]); end; // membuat plot
>hold off; // mengakhiri mode overwrite
>insimg; // menampilkan gambar ke notebook
\end{eulerprompt}
\eulerimg{27}{images/Hamemayu Hayuningrat_23030630053_EMTPlot2D-001.png}
\begin{eulerprompt}
>reset;
\end{eulerprompt}
\begin{eulercomment}
Ini diperlukan untuk menahan gambar-gambar, karena perintah plot() akan membersihkan layar plot.

Untuk memebersihkan semuanya, kami menggunakan reset().

Untuk menampilkan gambar hasil plot di layar notebook, perintah plot2d() dapat diakhiri dengan titik dua (:).
Cara lain adalah perintah plot2d() diakhiri dengan titik koma (;), kemudian menggunakan perintah insimg() untuk
menampilkan gambar hasil plot.

Untuk contoh lainnya, kami menggambar sebuah plot sebagai sisipan dalam plot lainnya. Ini dapat dilakukan dengan
mendefinisikan sebuah layar plot yang lebih kecil. Catatan bahwa layar ini tidak menyediakan ruang untuk
label-label sumbu yang diluar dari layar plot. Kami juga dapat menambahkan beberapa margin untuk ini apabila
diperlukan.  Catatan bahwa kami menyimpan dan mengembalikan layar secara penuh, dan menahan plot sekarang
sementara kami mensisipkan plot.
\end{eulercomment}
\begin{eulerprompt}
>plot2d("x^3-x");
>xw = 200; yw = 100; ww = 300; hw = 300;
>ow = window();
>window(xw, yw, xw + ww, yw + hw);
>hold on;
>barclear(xw - 50, yw - 10, ww + 60, ww + 60);
>plot2d("x^4 - x", grid = 6);
>hold off;
>window(ow);
\end{eulerprompt}
\begin{eulercomment}
Sebuah plot dengan gambar-gambar yang banyak dapat diperoleh dengan cara yang sama. Terdapat alat fungsi
figure() untuk melakukan ini.

\end{eulercomment}
\eulersubheading{Aspek Plot}
\begin{eulercomment}
Plot bawaan menggunakan sebuah layar plot persegi. Kamu dapat menggubahnya dengan funsi aspect(). Jangan lupa
untuk mengatur ulang aspect nanti. Kamu juga dapat mengubah bawaan dalam menu dengan "Set Aspect" untuk sebuah
rasio aspect secara spesific atau untuk ukuran sekarang dari layar gambar.

Tapi kamu dapat merubahnya juga untuk satu plot. Untuk melakukan ini, ukuran sekarang dari luas plot dapat
berubah, dan layar diatur jadi label-label memiliki ruang yang cukup.

\end{eulercomment}
\begin{eulerprompt}
>aspect(2); // rasio dari panjang dan lebar 2:1
>plot2d(["sin(x)", "cos(x)"], 0, 2pi);
>aspect();
>reset;
\end{eulerprompt}
\begin{eulercomment}
Fungsi reset() mengembalikan plot ke pengaturan bawaaan termasuk rasio aspek.

\begin{eulercomment}
\eulerheading{Plot-plot 2D dalam Euler}
\begin{eulercomment}
EMT Math Toolbox memiliki plot-plot dalam 2D, dua-duanya untuk data dan fungsi-fungsi. EMT menggunakan fungsi
plot2d. Fungsi ini dapat mem-plotkan fungsi-fungsi dan data.

Sangat mungkin untuk mem-plotkan dalam Maxima menggunakan Gnuplot atau dalam Python menggunakan Math Plot Lib.

Euler dapat mem-plot plot-plot 2D dari\\
- ekspresi,\\
- fungsi-fungsi, variabel-variabel, atau kurva-kurva yang terparameter.\\
- vektor-vektor dari nilai-nilai x-y.\\
- Titik-titk dalam bidang.\\
- Kurva-kurva implisit dengan level-level atau level regional.\\
- Fungsi kompleks

Gaya plot termasuk gaya-gaya yang bervariasi untuk garis-garis dan titik-titik, plot-plot batang dan plot-plot
terasir.

\begin{eulercomment}
\eulerheading{Plot Ekspresi atau Variabel}
\begin{eulercomment}
Ekspresi tunggal dalam  "x" (seperti "4*x\textasciicircum{}2") atau nama sebuah fungsi (seperti "f") menghasilkan sebuah grafik
dari fungsi.

Ini adalah contoh umum, yang aman menggunakan selang bawaan dan himpunan-himpunan sesuai selang-y untuk
menyesuaikan plot dari fungsi.

Catatan: Jika kamu mengakhiri sebuah baris perintah dengan sebuah titik dua ":", plot akan dimasukkan kedalam
layar teks. Sebaliknya, tekan TAB untuk melihat plot jika layar plot tertutupi.
\end{eulercomment}
\begin{eulerprompt}
>plot2d("x^2"):
\end{eulerprompt}
\eulerimg{27}{images/Hamemayu Hayuningrat_23030630053_EMTPlot2D-002.png}
\begin{eulerprompt}
>aspect(1.5); plot2d("x^3-x"):
\end{eulerprompt}
\eulerimg{17}{images/Hamemayu Hayuningrat_23030630053_EMTPlot2D-003.png}
\begin{eulerprompt}
>a := 5.6; plot2d("exp(-a*x^2) / a"); insimg(30); // menambilkan gambar hasil plot tertinggi setinggi 25 baris
\end{eulerprompt}
\eulerimg{17}{images/Hamemayu Hayuningrat_23030630053_EMTPlot2D-004.png}
\begin{eulercomment}
Dari beberapa contoh sebelumnya Anda dapat melihat bahwa aslinya gambar plot menggunakan sumbu X dengan rentang
nilai dari -2 sampai dengan 2. Untuk mengubah rentang nilai X dan Y, Anda dapat menambahkan nilai-nilai batas X
(dan Y) di belakang ekspresi yang digambar.

Selang dari plot diatur mengikuti parameter-parameter yang ditujukan

- a,b: x-range (bawaan -2,2)\\
- c,d: y-range (bawaan: skala dengan nilai-nilai)\\
- r: secara alternatif sebuah jari-jari mengelilingi plot lingkaran\\
- cx,cy: koordinat-koordinat dari tengah plot (bawaaan 0,0)
\end{eulercomment}
\begin{eulerprompt}
>plot2d("x^3-x",-1,2):
\end{eulerprompt}
\eulerimg{17}{images/Hamemayu Hayuningrat_23030630053_EMTPlot2D-005.png}
\begin{eulerprompt}
>plot2d("sin(x)",-2*pi,2*pi): // plot sin(x) pada interval [-2pi, 2pi]
\end{eulerprompt}
\eulerimg{17}{images/Hamemayu Hayuningrat_23030630053_EMTPlot2D-006.png}
\begin{eulerprompt}
>plot2d("cos(x)","sin(3*x)",xmin=0,xmax=2pi):
\end{eulerprompt}
\eulerimg{17}{images/Hamemayu Hayuningrat_23030630053_EMTPlot2D-007.png}
\begin{eulercomment}
Sebuah alternatif untuk titik dua adalah perintah insimg(lines), yang mana memasukkan plot-plot yang mempati
sebuah angka yang spesifik dari baris-baris teks.

Dalam pilihan-pilihan, plot-plot dapat diatur untuk dimunculkan

- dalam sebuah jendela yang dapat di ubah ukurannya secara terpisah,\\
- dalam jendela notebook.

Gaya-gaya lainnya yang dapat diperoleh dengan perintah-perintah plot yang spesifik

Dalam kasus yang lainnya, tekan tabulator untuk melihat plot, jika tersembunyi.

Untuk memisahkan jendela menjadi beberapa plot-plot, gunakan perintah figure(). Dalam contoh, kami membuat plot
x\textasciicircum{}1 ke x\textasciicircum{}4 kedalam 4 bagian dari jendela. figure(0) mengatur ulang jendela bawaan.
\end{eulercomment}
\begin{eulerprompt}
>reset;
>figure(2, 2);...
>for n = 1 to 4; figure(n); plot2d("x^"+n); end;...
>figure(0):
\end{eulerprompt}
\eulerimg{27}{images/Hamemayu Hayuningrat_23030630053_EMTPlot2D-008.png}
\begin{eulercomment}
Jika dalam argumen-argumen untuk plot2d() adalah sebuah ekspresi yang diikuti oleh empat angka, ini angka-angka
x- dan y-selang untuk plot.

Sebagai alternatif, a, b, c, d dapat secara spesifik sebagai parameter-parameter yang ditujukan seperti a = ...
dan lain-lain.

Seperti contoh berikut, kami mengubah gaya grid, menambah label-label, dan menggunakan label-label vertikan
untuk sumbu-y.
\end{eulercomment}
\begin{eulerprompt}
>aspect(1.5); plot2d("sin(x)", 0, 2pi, -1.2, 1.2, grid = 3, xl = "x", yl = "sin(x)"):
\end{eulerprompt}
\eulerimg{17}{images/Hamemayu Hayuningrat_23030630053_EMTPlot2D-009.png}
\begin{eulerprompt}
>plot2d("sin(x) + cos(2*x)", 0, 4pi):
\end{eulerprompt}
\eulerimg{17}{images/Hamemayu Hayuningrat_23030630053_EMTPlot2D-010.png}
\begin{eulercomment}
Gambar-gambar yang dihasilkan dengan memasukkan plot kedalam layar teks disimpan kedalam direktori seperti
notebook, secara bawaaan dalam sebuah sub-direktori bernama "images". Mereka juga menggunakan dengan ekspor
HTML.

Kamu dapat secara mudah membuat tanda pada sebarang gambar dan mencetaknya kedalam clibboard dengan Ctrl + c.
Tentu saja, kamu dapat juga mengekspor grafik-grafik sekarang dengan fungsi-fungsi dalam menu File.

Fungsi atau ekspresi dalam plot2d dapat dievaluasi secara adaptif. Untuk kecepatan yang lebih, matikan plot-plot
adaptif dengan \textless{}adaptive dan spesifik angka dalam sub-interval dengan n = ... Ini seharusnya dibutuhkan dalam
kasus yang sangat jarang.
\end{eulercomment}
\begin{eulerprompt}
>plot2d("sign(x) * exp(-x^2)", -1, 1, <adaptive, n = 10000):
\end{eulerprompt}
\eulerimg{17}{images/Hamemayu Hayuningrat_23030630053_EMTPlot2D-011.png}
\begin{eulerprompt}
>plot2d("x^x", r = 1.2, cx = 1, cy = 1):
\end{eulerprompt}
\eulerimg{17}{images/Hamemayu Hayuningrat_23030630053_EMTPlot2D-012.png}
\begin{eulercomment}
Catatan bahwa x\textasciicircum{}x tidak terdifinisi untuk x \textless{}= 0. Fungsi plot2d mengetahui error ini, dan mulai membuat plot
pada fungsi yang terdefinisi. Ini bekerja untuk semua fungsi yang mana mengembalikan NAN dari selang definisi.
\end{eulercomment}
\begin{eulerprompt}
>plot2d("log(x)", -0.1, 2):
\end{eulerprompt}
\eulerimg{17}{images/Hamemayu Hayuningrat_23030630053_EMTPlot2D-013.png}
\begin{eulercomment}
Parameter square=true (atau \textgreater{}square) memilih selang-y secara otomatis jadi bahwa hasil adah jendela kotak plot.
Catatan bahwa secara bawaan, Euler menggunakan sebuah spasi kotak didalam layar plot.
\end{eulercomment}
\begin{eulerprompt}
>plot2d("x^3-x", >square):
\end{eulerprompt}
\eulerimg{17}{images/Hamemayu Hayuningrat_23030630053_EMTPlot2D-014.png}
\begin{eulerprompt}
>plot2d(''integrate("sin(x) *exp(-x^2)", 0, x)'', 0, 2): // plot integral
\end{eulerprompt}
\eulerimg{17}{images/Hamemayu Hayuningrat_23030630053_EMTPlot2D-015.png}
\begin{eulercomment}
Jika kamu membutuhkan ruang yang lebih untuk label-y, panggil shrinkwindow() dengan parameter yang lebih kecil,
atau atur sebuah nilai positif untuk "smaller" dalam plot2d().
\end{eulercomment}
\begin{eulerprompt}
>plot2d("gamma(x)", 1, 10, yl = "y-values", smaller = 6, <vertical):
\end{eulerprompt}
\eulerimg{17}{images/Hamemayu Hayuningrat_23030630053_EMTPlot2D-016.png}
\begin{eulercomment}
\end{eulercomment}
\begin{eulercomment}
Ekspresi-ekspresi simbolik dapat juga digunakan, karena mereka disimpan sebagai ekspresi-ekspresi string yang
sederhana.
\end{eulercomment}
\begin{eulerprompt}
>x=linspace(0,2pi,1000); plot2d(sin(5x),cos(7x)):
\end{eulerprompt}
\eulerimg{17}{images/Hamemayu Hayuningrat_23030630053_EMTPlot2D-017.png}
\begin{eulerprompt}
>a:=5.6; expr &= exp(-a*x^2)/a; // mendefinisikan ekspresi
>plot2d(expr,-2,2): // membuat plot dari -2 ke 2
\end{eulerprompt}
\eulerimg{17}{images/Hamemayu Hayuningrat_23030630053_EMTPlot2D-018.png}
\begin{eulerprompt}
>plot2d(expr,r=1,thickness=2): // membuat plot dalam sebuah persegi sekitar (0,0)
\end{eulerprompt}
\eulerimg{17}{images/Hamemayu Hayuningrat_23030630053_EMTPlot2D-019.png}
\begin{eulerprompt}
>plot2d(&diff(expr,x),>add,style="--",color=red): // menambahkan plot lainnya
\end{eulerprompt}
\eulerimg{17}{images/Hamemayu Hayuningrat_23030630053_EMTPlot2D-020.png}
\begin{eulerprompt}
>plot2d(&diff(expr,x,2),a=-2,b=2,c=-2,d=1): // plot dalam persegi panjang
\end{eulerprompt}
\eulerimg{17}{images/Hamemayu Hayuningrat_23030630053_EMTPlot2D-021.png}
\begin{eulerprompt}
>plot2d(&diff(expr,x),a=-2,b=2,>square): // tetap membuat plot pada persegi
\end{eulerprompt}
\eulerimg{17}{images/Hamemayu Hayuningrat_23030630053_EMTPlot2D-022.png}
\begin{eulerprompt}
>plot2d("x^2",0,1,steps=1,color=red,n=10):
\end{eulerprompt}
\eulerimg{17}{images/Hamemayu Hayuningrat_23030630053_EMTPlot2D-023.png}
\begin{eulerprompt}
>plot2d("x^2",>add,steps=2,color=blue,n=10):
\end{eulerprompt}
\eulerimg{17}{images/Hamemayu Hayuningrat_23030630053_EMTPlot2D-024.png}
\eulerheading{Fungsi-fungsi dalam satu Parameter}
\begin{eulercomment}
Hal yang paling penting dalam membuat fungsi plot adalah plot-plot planar adalah plot2d. Fungsi adalah
pengimplementasian dalam bahawsa Euler dalam berkas "plot.e", yang mana dimuat pada saat program dimulai.

Ini adalah beberapa contoh menggunakan sebuah fungsi. Seperti biasa dalam EMT, fungsi-fungsi itu bekerja untuk
fungsi fungsi lainnya atau ekspresi-ekspresi, kamu dapat menambahkan parameter tambahan (disamping x) yang mana
bukan variabel-variabel global untuk fungsi dengan parameter parameter titik dua atau dengan sebuah koleksi
pemanggilan.
\end{eulercomment}
\begin{eulerprompt}
>function f(x, a) := x^2/a+a*x^2-x; // mendefinisikan sebuah fungsi
>a = 0.3; plot2d("f", 0, 1; a): // plot fungsi f dengan a = 0.3
\end{eulerprompt}
\eulerimg{17}{images/Hamemayu Hayuningrat_23030630053_EMTPlot2D-025.png}
\begin{eulerprompt}
>plot2d("f", 0, 1; 0.4): // plot fungsi f dengan a = 0.4
\end{eulerprompt}
\eulerimg{17}{images/Hamemayu Hayuningrat_23030630053_EMTPlot2D-026.png}
\begin{eulerprompt}
>plot2d(\{\{"f",0.2\}\}, 0, 1): // plot dengan a = 0.2
\end{eulerprompt}
\eulerimg{17}{images/Hamemayu Hayuningrat_23030630053_EMTPlot2D-027.png}
\begin{eulerprompt}
>plot2d(\{\{"f(x,b)",b = 0.1\}\}, 0, 1): // plot dengan 0.1
\end{eulerprompt}
\eulerimg{17}{images/Hamemayu Hayuningrat_23030630053_EMTPlot2D-028.png}
\begin{eulerprompt}
>function f(x) := x^3-x; ...
>plot2d("f", r = 1):
\end{eulerprompt}
\eulerimg{17}{images/Hamemayu Hayuningrat_23030630053_EMTPlot2D-029.png}
\begin{eulercomment}
Ini adalah rangkuman dari fungsi-fungsi yang diterima

- ekspresi-ekspresi atau ekspresi-ekspresi simbolik dalam x\\
- fungsi-fungsi atau simbolik fungsi-fungsi dengan nama sebagai "f"\\
- fungsi-fungsi simbolik hanya dengan nama f

fungsi plot2d() juga menerima fungsi-fungsi simbolik. Untuk simbolik fungsi, nama bekerja secara sendiri.
\end{eulercomment}
\begin{eulerprompt}
>function f(x) &= diff(x^x, x)
\end{eulerprompt}
\begin{euleroutput}
  
                              x
                             x  (log(x) + 1)
  
\end{euleroutput}
\begin{eulerprompt}
>plot2d(f, 0, 2):
\end{eulerprompt}
\eulerimg{17}{images/Hamemayu Hayuningrat_23030630053_EMTPlot2D-030.png}
\begin{eulerprompt}
>expr &= sin(x) * exp(-x)
\end{eulerprompt}
\begin{euleroutput}
  
                                - x
                               E    sin(x)
  
\end{euleroutput}
\begin{eulerprompt}
>plot2d(expr, 0, 3pi):
\end{eulerprompt}
\eulerimg{17}{images/Hamemayu Hayuningrat_23030630053_EMTPlot2D-031.png}
\begin{eulerprompt}
>function f(x) &= x^x;
>plot2d(f, r = 1, cx = 1, cy = 1, color = blue, thickness = 2):
\end{eulerprompt}
\eulerimg{17}{images/Hamemayu Hayuningrat_23030630053_EMTPlot2D-032.png}
\begin{eulerprompt}
>plot2d(&diff(f(x), x), >add, color = red, style="-.-"):
\end{eulerprompt}
\eulerimg{17}{images/Hamemayu Hayuningrat_23030630053_EMTPlot2D-033.png}
\begin{eulercomment}
Untuk gaya garis disini memiliki banyak opsi.

- style = "...". Memilih dari "-", "--", "-.", ".", ".-.", "-.-".\\
- color: Lihat dibawah untuk warna-warna.\\
- thickness: Bawaan adalah 1.

Warna warna dapat dipilih sebagai satu dari warna bawaan, atau sebagai sebuah warna RGB.

- 0..5: indeks warna bawaan.\\
- konstanta warna: white, black, red, green, blue, cyan, olive, lightgray, gray, darkgray, orange, lightgreen,
turquoise, lightblue, lightorange, yellow.\\
- rgb(red, green, blue): parameter marameter adalah bilangan rill dalam [0, 1].
\end{eulercomment}
\begin{eulerprompt}
>plot2d("exp(-x^2)", r = 2, color = red, thickness = 3, style = "--"):
\end{eulerprompt}
\eulerimg{17}{images/Hamemayu Hayuningrat_23030630053_EMTPlot2D-034.png}
\begin{eulercomment}
Ini adalah pandangan dari color yang telah terdefinisi dari EMT.
\end{eulercomment}
\begin{eulerprompt}
>aspect(2); columnsplot(ones(1, 16), lab = 0:15, grid = 0, color = 0:15):
\end{eulerprompt}
\eulerimg{13}{images/Hamemayu Hayuningrat_23030630053_EMTPlot2D-035.png}
\begin{eulercomment}
Tapi kamu dapat menggunakan warna apapun.
\end{eulercomment}
\begin{eulerprompt}
>columnsplot(ones(1, 16), grid = 0, color = rgb(0, 0, linspace(0, 1, 15))):
\end{eulerprompt}
\eulerimg{13}{images/Hamemayu Hayuningrat_23030630053_EMTPlot2D-036.png}
\eulerheading{Menggambar Beberapa Kurva pada bidang koordinat yang sama}
\begin{eulercomment}
Membuat plot lebih dari satu fungsi (fungsi-fungsi yang banyak) kedalam satu jendela dapat dilakukan dengan
cara-cara yang berbeda. Salah satu metode yang digunakan \textgreater{}add untuk beberapa pemanggilan ke plot2d dalam semua,
tetapi pemanggilan pertama. Kami telah menggunakan fitur ini dalam contoh diatas.
\end{eulercomment}
\begin{eulerprompt}
>aspect(); plot2d("cos(x)", r = 2, grid = 6); plot2d("x", style = "x", >add):
\end{eulerprompt}
\eulerimg{27}{images/Hamemayu Hayuningrat_23030630053_EMTPlot2D-037.png}
\begin{eulerprompt}
>aspect(1.5); plot2d("sin(x)",0,2pi); plot2d("cos(x)",color=blue,style="--",>add):
\end{eulerprompt}
\eulerimg{17}{images/Hamemayu Hayuningrat_23030630053_EMTPlot2D-038.png}
\begin{eulercomment}
Salah satu kegunaan \textgreater{}add adalah untuk menambahkan titik pada kurva.
\end{eulercomment}
\begin{eulerprompt}
>plot2d("sin(x)", 0, pi); plot2d(2, sin(2), >points, >add):
\end{eulerprompt}
\eulerimg{17}{images/Hamemayu Hayuningrat_23030630053_EMTPlot2D-039.png}
\begin{eulercomment}
Kami menambahkan titik singgung dengan sebuah label (pada posisi "cl" untuk kiri tengah), dan memasukkan hasil
kedalam notebook. Kami juga menambahkan sebuah judul kedalam plot.
\end{eulercomment}
\begin{eulerprompt}
>plot2d(["cos(x)", "x"], r = 1.1, cx = 0.5, cy = 0.5, ...
>color = [black, blue], style = ["-", "."], ...
>grid = 1):
\end{eulerprompt}
\eulerimg{17}{images/Hamemayu Hayuningrat_23030630053_EMTPlot2D-040.png}
\begin{eulerprompt}
>x0 = solve("cos(x) - x", 1);...
>plot2d(x0, x0, >points, >add, title = "Intersection Demo"); ...
>label("cos(x) = x", x0, x0, pos = "cl", offset = 20):
\end{eulerprompt}
\eulerimg{17}{images/Hamemayu Hayuningrat_23030630053_EMTPlot2D-041.png}
\begin{eulercomment}
Mengikuti demo, kami plot fungsi sinc(x) = sin(x) / x dan ke-8 dan ke-16 ekspansi Taylor. Kami menghitung
ekspansi ini menggunakan Maxima melalui ekspresi simbolik.

Plot ini dibuat mengikuti perintah baris-banyak dengan tiga kali pemanggilan untuk plot2d(). Pengaturan kedua
dan ketiga \textgreater{}add, yang mana membuat plot menggunakan rentang sebelumnya.

Kita menambahkan sebuah boks label untuk menjelaskan fungsi.
\end{eulercomment}
\begin{eulerprompt}
>$taylor(sin(x) / x, x, 0, 4)
\end{eulerprompt}
\begin{eulerformula}
\[
\frac{x^4}{120}-\frac{x^2}{6}+1
\]
\end{eulerformula}
\begin{eulerprompt}
>plot2d("sinc(x)", 0, 4pi, color = green, thickness = 2);...
>plot2d(&taylor(sin(x) / x, x, 0, 8), >add, color = blue, style = "--");...
>plot2d(&taylor(sin(x) / x, x, 0, 16), >add, color = red, style = "-.-");...
>labelbox(["sinc", "T8", "T16"], styles=["-", "--", "-.-"],...
>colors=[black, blue, red]):
\end{eulerprompt}
\eulerimg{17}{images/Hamemayu Hayuningrat_23030630053_EMTPlot2D-043.png}
\begin{eulercomment}
Contoh berikut, kami membuat Bernstein-Polynomials.

\end{eulercomment}
\begin{eulerformula}
\[
B_i(x) = \binom{n}{i} x^i (1-x)^{n-i}
\]
\end{eulerformula}
\begin{eulerprompt}
>plot2d("(1-x)^10",0,1): // plot fungsi pertama
\end{eulerprompt}
\eulerimg{17}{images/Hamemayu Hayuningrat_23030630053_EMTPlot2D-044.png}
\begin{eulerprompt}
>for i=1 to 10; plot2d("bin(10,i)*x^i*(1-x)^(10-i)",>add); end;
>insimg;
\end{eulerprompt}
\eulerimg{17}{images/Hamemayu Hayuningrat_23030630053_EMTPlot2D-045.png}
\begin{eulercomment}
Metode kedua menggunakan sepasang dari sebuah matriks dari nilai-x dan sebuah matriks dari nilai-y dari ukuran
yang sama.

Kami menghasilkan sebuah matriks dari nilai-nilai dengan satu Bernstein-Polynomial dalam setiap bari. Untuk ini,
kami mempersingkat menggunakan sebuah vektor kolom dari i. Lihat kembali ke pengantar tentang bahasa matriks
untuk mempelajari lebih rinci.
\end{eulercomment}
\begin{eulerprompt}
>x = linspace(0, 1, 500);
>n = 10; k = (0:n)'; // n adalah vektor baris, k adalah kolom vektor
>y = bin(n, k)*x^k*(1-x)^(n-k); // y adalah sebuah matriks lalu
>plot2d(x, y):
\end{eulerprompt}
\eulerimg{17}{images/Hamemayu Hayuningrat_23030630053_EMTPlot2D-046.png}
\begin{eulercomment}
Catatan bahwa parameter warna dapat berupa sebuah vektor. Lalu setiap warna dapat digunakan untuk setiap baris
dari matriks.
\end{eulercomment}
\begin{eulerprompt}
>x=linspace(0,1,200); y=x^(1:10)'; plot2d(x,y,color=1:10):
\end{eulerprompt}
\eulerimg{17}{images/Hamemayu Hayuningrat_23030630053_EMTPlot2D-047.png}
\begin{eulercomment}
Metode lain adalah menggunakan vektor ekspresi (string). Kamu dapat menggunakan sebuah array warna, sebuah array
dari gaya, dan sebuah array dari ketebalan dari panjang yang sama.
\end{eulercomment}
\begin{eulerprompt}
>plot2d(["sin(x)","cos(x)"],0,2pi,color=4:5):
\end{eulerprompt}
\eulerimg{17}{images/Hamemayu Hayuningrat_23030630053_EMTPlot2D-048.png}
\begin{eulerprompt}
>plot2d(["sin(x)","cos(x)"],0,2pi): // plot ekspresi vektor
\end{eulerprompt}
\eulerimg{17}{images/Hamemayu Hayuningrat_23030630053_EMTPlot2D-049.png}
\begin{eulercomment}
Kami dapat mendapatkan sebuah vektor dari Maxima menggunakan makelist() dan mxm2str().
\end{eulercomment}
\begin{eulerprompt}
>v &= makelist(binomial(10,i)*x^i*(1-x)^(10-i),i,0,10) // membuat list
\end{eulerprompt}
\begin{euleroutput}
  
                 10            9              8  2             7  3
         [(1 - x)  , 10 (1 - x)  x, 45 (1 - x)  x , 120 (1 - x)  x , 
             6  4             5  5             4  6             3  7
  210 (1 - x)  x , 252 (1 - x)  x , 210 (1 - x)  x , 120 (1 - x)  x , 
            2  8              9   10
  45 (1 - x)  x , 10 (1 - x) x , x  ]
  
\end{euleroutput}
\begin{eulerprompt}
>mxm2str(v) // mendapatkan sebuah vektor dari string dari vektor simbolik
\end{eulerprompt}
\begin{euleroutput}
  (1-x)^10
  10*(1-x)^9*x
  45*(1-x)^8*x^2
  120*(1-x)^7*x^3
  210*(1-x)^6*x^4
  252*(1-x)^5*x^5
  210*(1-x)^4*x^6
  120*(1-x)^3*x^7
  45*(1-x)^2*x^8
  10*(1-x)*x^9
  x^10
\end{euleroutput}
\begin{eulerprompt}
>plot2d(mxm2str(v),0,1): // fungsi-fungsi plot
\end{eulerprompt}
\eulerimg{17}{images/Hamemayu Hayuningrat_23030630053_EMTPlot2D-050.png}
\begin{eulercomment}
Alternatif lainnya adalah menggunakan bahasa matriks dari Euler.

Jika sebuah ekspresi menghasilkan sebuah matriks dari fungsi, dengan satu fungsi dalam setiap baris, semua
fungsi-fungsi ini akan di plotkan menjadi satu plot.

Untuk ini, gunakan sebuah vektor parameter dalam bentuk dari vektor kolom. Jika sebuah warna ditambahkan akan
digunakan untuk setiap baris dari plot.
\end{eulercomment}
\begin{eulerprompt}
>n=(1:10)'; plot2d("x^n",0,1,color=1:10):
\end{eulerprompt}
\eulerimg{17}{images/Hamemayu Hayuningrat_23030630053_EMTPlot2D-051.png}
\begin{eulercomment}
Ekspresi-ekspresi dan fungsi-fungsi satu-baris dapat dilihat variabel
global.

Jika kamu tidak menggunakan variabel global, kamu membutuhkan untuk
menggunakan sebuah fungsi dengan parameter ekstra, dan memasukkan
parameter ini seperti sebuah parameter titik dua.

Berhati-hati, untuk menggambil semua parameter yang terpasang ke akhir
dari perintah plot2d. Dalam contoh kami memasukkan a = 5 ke fungsi f,
yang mana kami plot dari -10 ke 10.
\end{eulercomment}
\begin{eulerprompt}
>function f(x,a) := 1/a *exp(-x^2/a); ...
>plot2d("f",-10,10;5,thickness=2,title="a=5"):
\end{eulerprompt}
\eulerimg{17}{images/Hamemayu Hayuningrat_23030630053_EMTPlot2D-052.png}
\begin{eulercomment}
Sebagai gantinya, gunakan sebuah collection dengan nama  fungsi dan
semua parameter-parameter tambahan. Daftar spesial ini disebuat
collection dan ini adalah cara yang disarankan untuk memasukkan
argumen kedalam sebuah fungsi yang mana itu sendiri dimasukkan sebagai
sebuah argumen kedalam fungsi lainnya.

Contoh berikut, kami menggunakan sebuah iterasi untuk membuat plot
beberapa fungsi (lihat tutorial tentang pemrograman untuk iterasi).
\end{eulercomment}
\begin{eulerprompt}
>plot2d(\{\{"f",1\}\}, -10, 10); ...
>for a = 2:10; plot2d(\{\{"f", a\}\}, >add); end:
\end{eulerprompt}
\eulerimg{17}{images/Hamemayu Hayuningrat_23030630053_EMTPlot2D-053.png}
\begin{eulercomment}
Kami dapat menghasilkan hasil yang sama dengan menggunakan cara
seperti menggunakan bahasa matriks EMT. Setiap baris dari matriks f(x,
a) adalah satu fungsi. Terlebih, kita dapat mengatur warna untuk
setiap baris dari matriks. Untuk penjelasan klik dua kali ke fungsi
getspectral().
\end{eulercomment}
\begin{eulerprompt}
>x = -10:0.01:10; a = (1 : 10)'; plot2d(x, f(x,a), color = getspectral(a/10)):
\end{eulerprompt}
\eulerimg{17}{images/Hamemayu Hayuningrat_23030630053_EMTPlot2D-054.png}
\eulersubheading{Label Teks}
\begin{eulercomment}
Dekorasi sederhana dapat berupa

- sebuah judul dengan title="..."\\
- label-x dan label-y dengan xl="...", yl="..."\\
- teks label lainnya dengan label("...", x, y)

perintah label akan memplot ke plot sekarang sebagai koordinat plot
(x, y). Ini dapat menggambil argumen posisi.
\end{eulercomment}
\begin{eulerprompt}
>plot2d("x^3 - x", -1, 2, title="y = x^3 - x", yl="y", xl="x"):
\end{eulerprompt}
\eulerimg{17}{images/Hamemayu Hayuningrat_23030630053_EMTPlot2D-055.png}
\begin{eulerprompt}
>expr := "log(x) / x";...
>plot2d(expr, 0.5, 5, title = "y =" +expr, xl = "x", yl = "y");...
>label("(1, 0)", 1, 0); label("Max", E, expr(E), pos="1c"):
\end{eulerprompt}
\eulerimg{17}{images/Hamemayu Hayuningrat_23030630053_EMTPlot2D-056.png}
\begin{eulercomment}
Ini juga fungsi labelbox(), yang mana dapat memperlihatkan
fungsi-fungsi dan sebuah teks. Dengan mengambil vektor dari string dan
warna, satu dari setiap fungsi.
\end{eulercomment}
\begin{eulerprompt}
>function f(x) &= x^2*exp(-x^2); ...
>plot2d(&f(x), a = -3, b = 3, c = -1, d = 1); ...
>plot2d(&diff(f(x), x), >add, color = blue, style = "--");...
>labelbox(["function", "derivative"], styles = ["-", "--"], ...
>colors = [black, blue], w = 0.4):
\end{eulerprompt}
\eulerimg{17}{images/Hamemayu Hayuningrat_23030630053_EMTPlot2D-057.png}
\begin{eulercomment}
Box tersebut diposisikan pada kanan atas secara bawaan, tetapi posisi
\textgreater{}left membuatnya pada kiri atas. Kamu dapat mengubahnya sebarang
tempat yang kamu sukai. Posisi terdapat pada pojok kanan atas dari
box, dana angka-angkanya adalah pecahan dari ukuran jendela. Lebar
secara otomatis diatur.

Untuk titik-titik plot, label box juga demikian. Tambahkan parameter
\textgreater{}points, atau sebuah vektor dari pernyataan, satu untuk setiap label.

Seperti contoh yang ditunukkan, ini hanya satu fungsi. Jadi kami dapat
menggunakan strings daripada vektor string. Kami mengatur warna teks
ke hitam untuk contoh ini.
\end{eulercomment}
\begin{eulerprompt}
>n = 10;  plot2d(0: n, bin(n, 0: n), >addpoints);...
>labelbox("Binomials", styles = "[]", >points, x = 0, y = 0.1, ...
>tcolor = black, >left):
\end{eulerprompt}
\eulerimg{17}{images/Hamemayu Hayuningrat_23030630053_EMTPlot2D-058.png}
\begin{eulercomment}
Gaya dari plot ini juga tersedia dalam statplot(). Seperti dalam
plot2d() warna-warna dapat diatur untuk setiap baris dari plot. Ini
adalah plot-plot spesial untuk tujuan statistika (lihat tutorial
tentang statistika).
\end{eulercomment}
\begin{eulerprompt}
>statplot(1: 10, random(2, 10), color = [red, blue]):
\end{eulerprompt}
\eulerimg{17}{images/Hamemayu Hayuningrat_23030630053_EMTPlot2D-059.png}
\begin{eulercomment}
Fitur yang sama dari fungsi textbox().

Secara bawaan lebar diatur setara dengan lebar maksimal dari lebar
baris teks. Tapi ini dapat diatur oleh pengguna juga.
\end{eulercomment}
\begin{eulerprompt}
>function f(x) &= exp(-x) * sin(2 * pi * x); ...
>plot2d("f(x)", 0, 2pi); ...
>textbox(latex("\(\backslash\)text\{Example of a  dampedoscillation\}\(\backslash\) f(x) = e^\{-x\}sin(2 \(\backslash\) pi x)"), w = 0.85):
\end{eulerprompt}
\eulerimg{17}{images/Hamemayu Hayuningrat_23030630053_EMTPlot2D-060.png}
\begin{eulercomment}
Label-label teks, judul, label box dan teks lainnya dapat berupa
string Unicode (lihat sintak dari EMT untuk lebih lanjut tentang
string Unicode).
\end{eulercomment}
\begin{eulerprompt}
>plot2d("x^3 - x", title = u"x &rarr; x&sup3; - x"):
\end{eulerprompt}
\eulerimg{17}{images/Hamemayu Hayuningrat_23030630053_EMTPlot2D-061.png}
\begin{eulercomment}
Label-label dalam sumbu x dan y dapat secara vertikal, mengikuti
sumbunya.
\end{eulercomment}
\begin{eulerprompt}
>plot2d("sinc(x)", 0, 2pi, xl = "x", yl = u"x &rarr; sinc(x)", >vertical):
\end{eulerprompt}
\eulerimg{17}{images/Hamemayu Hayuningrat_23030630053_EMTPlot2D-062.png}
\eulersubheading{LaTeX}
\begin{eulercomment}
Kamu juga dapat mem-plot formula LaTex jika kamu telah menginstall
sistem LaTex. Saya merekomendasikan MiKTeX. Jalur ke binari "latex"
dan "dvipng" harus ada dalam jalur sistem, taua kamu telah mengatur
LaTeX dalam menu options.

Catatan, bahwa pengubah LaTeX itu lambat. Jika kamu ingin LaTeX dalam
plot yang memiliki animai, kamu harus memanggil latex() sebelum
iterasi pertama dan gunakan hasilnya (sebuah gambar dalam sebuah
matriks RGB).

Seperti plot berikut, kita menggunakan LaTeX untuk label x dan label
y, sebuah label, sebuah label box dan judul plot.
\end{eulercomment}
\begin{eulerprompt}
>plot2d("exp(-x) * sin(x) / x", a = 0, b = 2pi, c = 0, d = 1, grid = 6, color = blue, ...
>title = latex("\(\backslash\)text\{Function $\(\backslash\)Phi$\}"), ...
>xl = latex("\(\backslash\)phi"), yl = latex("\(\backslash\)Phi(\(\backslash\)phi)"));...
>textbox(...
>latex("\(\backslash\)Phi(\(\backslash\)phi) = e^\{-\(\backslash\)phi\} \(\backslash\)frac\{\(\backslash\)sin(\(\backslash\)phi)\}\{\(\backslash\)phi\}"), x = 0.8, y = 0.5);...
>label(latex("\(\backslash\)Phi", color = blue), 1, 0.4):
\end{eulerprompt}
\eulerimg{17}{images/Hamemayu Hayuningrat_23030630053_EMTPlot2D-063.png}
\begin{eulercomment}
Terkadang, kita ingin spasi yang non-conformal dan label teks dalam
sumbu-x. Kita dapat menggunakan xaxis() dan yaxis() seperti yang akan
ditunjukkan kemudian.

Cara termudah untuk melakuakn sebuah plot yang tidak ada dengan sebuah
frame menggunakan grid = 4 dan lalu tambahkan grid dengan ygrid() dan
xgrid(). Seperti contoh berikut, kita gunakan tiga string LaTeX untuk
label dalam sumbu-x dengan xtick().
\end{eulercomment}
\begin{eulerprompt}
>plot2d("sinc(x)", 0, 2pi, grid = 4, <ticks);...
>ygrid(-2:0.5:2, grid = 6); ...
>xgrid([0:2] * pi, <ticks, grid = 6); ...
>xtick([0, pi, 2pi], ["0", "\(\backslash\)pi", "2\(\backslash\)pi"], >latex):
\end{eulerprompt}
\eulerimg{17}{images/Hamemayu Hayuningrat_23030630053_EMTPlot2D-064.png}
\begin{eulercomment}
Tentu saja, fungsi-fungsi dapat juga digunakan.
\end{eulercomment}
\begin{eulerprompt}
>function map f(x) ...
\end{eulerprompt}
\begin{eulerudf}
  if x > 0 then return x^4
  else return x^2
  endif
  endfunction
\end{eulerudf}
\begin{eulercomment}
Parameter "map" membantu untuk menggunakan fungsi dari vektor. Untuk
membuat plot, ini tidak akan digunakan. Tetapi untuk demonstrasi bahwa
vektorisasi sangat berguna, kita tambahkan beberapa titik kedalam plot
pada x = -1, x = 0 dan x = 1. Seperti plot berikutm kita juga
memasukkan beberapa code LaTeX. Kami gunakan ini untuk dua label dan
sebuah teks box. Tentu saja, kamu hanya ingin menggunanakan LaTeX jika
kamu telah menginstal LaTeX secara benar.
\end{eulercomment}
\begin{eulerprompt}
>plot2d("f", -1, 1, xl = "x", yl = "f(x)", grid = 6);...
>plot2d([-1 ,0, 1], f([-1, 0, 1]), >points, >add);...
>label(latex("x^3"), 0.72, f(0.72));...
>label(latex("x^2"), -0.52, f(-0.52), pos = "11");...
>textbox(...
>latex("f(x) = \(\backslash\)begin\{cases\} x^3 & x > 0 \(\backslash\)\(\backslash\) x^2 & x \(\backslash\)le 0 \(\backslash\)end\{cases\}"), ...
>x = 0.7, y = 0.2):
\end{eulerprompt}
\eulerimg{17}{images/Hamemayu Hayuningrat_23030630053_EMTPlot2D-065.png}
\eulersubheading{Interaksi Pengguna}
\begin{eulercomment}
Ketika membuat plot sebuah fungsi atau sebuah ekspresi, parameter
\textgreater{}user memperbolehkan pengguna untuk memperbesar dan memindahkan plot
dengan kursor atau mouse.

Pengguna dapat\\
- memperbesar dengan + or -\\
- memindahkan plot dengan kursor\\
- memilih jendela sebuah plot dengan mouse\\
- mengatur ulang pandangan dengan spasi\\
- keluar dengan return

Spasi akan mengatur ulang plot menjadi jendela plot asli.

Ketika membuat plot sebuah data, \textgreater{}user parameter akan secara sederhana
menunggu tombol diketikkan.
\end{eulercomment}
\begin{eulerprompt}
>plot2d(\{\{"x^3 - a*x", a = 1\}\}, >user, title = "Press any key!"):
\end{eulerprompt}
\eulerimg{17}{images/Hamemayu Hayuningrat_23030630053_EMTPlot2D-066.png}
\begin{eulerprompt}
>plot2d("exp(x) * sin(x)", user = true, ...
>title="+ / - or cursor keys (return to exit)"):
\end{eulerprompt}
\eulerimg{17}{images/Hamemayu Hayuningrat_23030630053_EMTPlot2D-067.png}
\begin{eulercomment}
Demonstrasi berikut sebuah cara lanjutan dari interaksi pengguna
(lihat tutorial tentang pemrograman untuk detailnya).

Fungsi yang sudah ada mousedrag() menunggu mouse atau pergerakan
keyboard. ini akan melaporkan mouse yang bergerak dan keyboard
tertekan. fungsi dragpoints() menggunakan ini, dan memungkinkan
pengguna menyeret sebarang titik dalam sebuah plot.

Kita butuh sebuah fungsi plot terlebih dahulu. Sebagai contoh, kita
menginterpolasi dalam 5 titik oleh sebuah polinomial. Fungsi akan
memplot sampai sebuah luasan plot yang tetap.
\end{eulercomment}
\begin{eulerprompt}
>function plotf(xp, yp, select)...
\end{eulerprompt}
\begin{eulerudf}
  d=interp(xp, yp);
  plot2d("interpval(xp, d, x)"; d, xp, r = 2);
  plot2d(xp, yp, >points, >add);
  if select > 0 then
     plot2d(xp[select], yp[select], color = red, >points, >add);
  endif;
  title("Drag one point, or press space or return!");
  endfunction
\end{eulerudf}
\begin{eulercomment}
Catatan paramter semicolon dalam plot2d(d dan xp), yang mana
dimasukkan ke evaluasi dari fungsi interp(). Tanpa ini, kita harus
menuliskan sebuah fungsi plotinterp() terlebih dahulu, mengakses nilai
secara global.

Sekarang kita menghasilkan beberapa nilai acak dan biarkan pengguna
menggeser titik-titiknya.
\end{eulercomment}
\begin{eulerprompt}
>t = -1:0.5:1; dragpoints("plotf", t, random(size(t)) - 0.5):
\end{eulerprompt}
\eulerimg{17}{images/Hamemayu Hayuningrat_23030630053_EMTPlot2D-068.png}
\begin{eulercomment}
Ini juga sebuah fungsi, yang mana membuat plot fungsi lainnya
tergantung dalam sebuah vektor dari parameter, dan membiarkan pengguna
menyesuaikan parameter-parameternya.

Pertama-tama kita butuh fungsi plot.
\end{eulercomment}
\begin{eulerprompt}
>function plotf([a, b]) := plot2d("exp(a * x) * cos(2pi * b * x)", 0, 2pi; a, b);
\end{eulerprompt}
\begin{eulercomment}
Lalu kita butuhkan nama untuk parameter-parameter, nilai awal dn
sebuah matriks nx2 dari selang, secara pilihan sebuah baris judul.

Ini adalah slider yang interaktif, dimana dapat merubah nilai oleh
pengguna. Fungsi dragvalues() tersedia untuk hal ini.
\end{eulercomment}
\begin{eulerprompt}
>dragvalues("plotf", ["a", "b"], [-1, 2], [[-2, 2]; [1, 10]], ...
>heading="Drag these values: ", hcolor = black):
\end{eulerprompt}
\eulerimg{17}{images/Hamemayu Hayuningrat_23030630053_EMTPlot2D-069.png}
\begin{eulercomment}
Ini memungkinkan untuk membatasi nilai yang dapat digerakkan hanya
bilangan bulat. Sebagai contoh, kita menulis sebuah fungsi plot, yang
mana membuat plot sebuah polinomial Taylor berderajat n ke fungsi
cosinus.
\end{eulercomment}
\begin{eulerprompt}
>function plotf(n):
\end{eulerprompt}
\begin{eulerudf}
     plot2d("cos(x)", 0, 2pi, >square, grid=6);
     plot2d(&"taylor(cos(x), x, 0, @n)", color = blue, >add);
     textbox("Taylor polynomial of degree " + n, 0.1, 0.2, style = "t", >left);
  endfunction
\end{eulerudf}
\begin{eulercomment}
Sekarang kita bisa membuat derajat n menjadi bervarisi dari 0 ke 20
dengan 20 pemberhentian. Hasil dari dragvalues() digunakan untuk
membuat plot sketsa dengan n, dan memasukkan plot kedalam notebook.
\end{eulercomment}
\begin{eulerprompt}
>nd = dragvalues("plotf", "degree", 2, [0, 20], 20, y = 0.8, ...
>heading="Drag the value: "); ...
>plotf(nd):
\end{eulerprompt}
\eulerimg{17}{images/Hamemayu Hayuningrat_23030630053_EMTPlot2D-070.png}
\begin{eulercomment}
Demonstrasi sederhana ini dari sebuah fungsi. Pengguna dapat
menggambar diatas jendela plot, meninggalkan titik-titik lintasan.
\end{eulercomment}
\begin{eulerprompt}
>function dragtest...
\end{eulerprompt}
\begin{eulerudf}
     plot2d(none, r = 1, title = "Drag with the mouse, or press any key!");
     start = 0;
     repeat
         \{flag, m, time\} = mousedrag();
         if flag == 0 then return; endif;
         if flag == 2 then
             hold on; mark(m[1], m[2]); hold off;
         endif;
     end
  endfunction
\end{eulerudf}
\begin{eulerprompt}
>dragtest
\end{eulerprompt}
\eulersubheading{Gaya-Gaya Plot 2D}
\begin{eulercomment}
Secara bawaan, EMT secara otomatis menghitung titik sumbu dan
menambahkan label ke titiknya. Ini dapat diubah dengan parameter grid.
Gaya bawaan dari sumbu dan labelnya dapat diubah. Sebagai tambahan,
label dan judul dapat ditambahkan secara manual. Untuk mengatur ulang
gaya bawaan, gunakan reset().
\end{eulercomment}
\begin{eulerprompt}
>aspect();
>figure(3, 4);...
>figure(1); plot2d("x^3-x", grid=0);... // tidak ada garis bantu, rangka atau sumbu
>figure(2); plot2d("x^3-x", grid=1);... // sumbu-x-y
>figure(3); plot2d("x^3-x", grid=2);... // titik bantu bawaan
>figure(4); plot2d("x^3-x", grid=3);... // sumbu x-y dengan label didalam
>figure(5); plot2d("x^3-x", grid=4);... // tidak ada titik bantu, hanya label
>figure(6); plot2d("x^3-x", grid=5);... // bawaan, tetapi tidak ada margin
>figure(7); plot2d("x^3-x", grid=6);... // hanya sumbu
>figure(8); plot2d("x^3-x", grid=7);... // hanya sumbu, titik bantu pada sumbu
>figure(9); plot2d("x^3-x", grid=8);... // hanya sumbu, titik bantu lebih mulus pada sumbu
>figure(10); plot2d("x^3-x", grid=9);... // bawaaan, titik bantu kecil didalam
>figure(11); plot2d("x^3-x", grid=10);... // tidak ada titik bantu, hanya sumbu
>figure(0):
\end{eulerprompt}
\eulerimg{27}{images/Hamemayu Hayuningrat_23030630053_EMTPlot2D-071.png}
\begin{eulercomment}
Parameter \textless{}frame membuat  frame tidak ada, dan famecolor = blue
mengubah frame menjadi berwarna biru.\\
Jika kamu ingin titik-titik kamu, kamu dapat menggunakan style = 0,
dan tambahkan kesemuanya nanti.
\end{eulercomment}
\begin{eulerprompt}
>aspect(1.5);
>plot2d("x^3 - x", grid = 0): // plot
\end{eulerprompt}
\eulerimg{17}{images/Hamemayu Hayuningrat_23030630053_EMTPlot2D-072.png}
\begin{eulerprompt}
>frame; xgrid([-1, 0, 1]); ygrid(0): // menambahkan kerangka dan garis bantu
\end{eulerprompt}
\eulerimg{17}{images/Hamemayu Hayuningrat_23030630053_EMTPlot2D-073.png}
\begin{eulercomment}
Untuk judul plot dan sumbu label, perhatikan contoh berikut
\end{eulercomment}
\begin{eulerprompt}
>plot2d("exp(x)", -1, 1);
>textcolor(black); // atur warna teks menjadi hitam
>title(latex("y = e^x")); // judul diatas plot
>xlabel(latex("x")); // "x" untuk sumbu-x
>ylabel(latex("y"), >vertical); // "y" secara vertikal untuk sumbu-y
>label(latex("(0, 1)"), 0, 1, color = blue): // label sebuah titik
\end{eulerprompt}
\eulerimg{17}{images/Hamemayu Hayuningrat_23030630053_EMTPlot2D-074.png}
\begin{eulercomment}
Sumbu-sumbu dapat digambarkan secara terpisah dengan xaxis() dan
yaxis().
\end{eulercomment}
\begin{eulerprompt}
>plot2d("x^3 - 3", <grid, <frame):
\end{eulerprompt}
\eulerimg{17}{images/Hamemayu Hayuningrat_23030630053_EMTPlot2D-075.png}
\begin{eulerprompt}
>xaxis(0, xx = -2:1, style = "->"); yaxis(0, yy = -5:5, style = "->"):
\end{eulerprompt}
\eulerimg{17}{images/Hamemayu Hayuningrat_23030630053_EMTPlot2D-076.png}
\begin{eulercomment}
Teks dalam plot dapat diatur dengan label(). Seperti contoh, "lc"
berarti tengah bawah. Ini mengatur posisi dari label secara relatif ke
dalam koordinat plot.
\end{eulercomment}
\begin{eulerprompt}
>function f(x) &= x^3 - x
\end{eulerprompt}
\begin{euleroutput}
  
                                   3
                                  x  - x
  
\end{euleroutput}
\begin{eulerprompt}
>plot2d(f, -1, 1, >square);
>x0 = fmin(f, 0, 1); // menghitung titik minimum
>label("Rel. Min. ", x0, f(x0), pos = "lc"): // menambahkan sebuah label disini
\end{eulerprompt}
\eulerimg{17}{images/Hamemayu Hayuningrat_23030630053_EMTPlot2D-077.png}
\begin{eulercomment}
Ini juga teks box.
\end{eulercomment}
\begin{eulerprompt}
>plot2d(&f(x), -1, 1, -2, 2); // fungsi
>plot2d(&diff(f(x), x), >add, style = "--", color = red); // turunan
>labelbox(["f", "f'"], ["-", "--"], [black, red]): // box label
\end{eulerprompt}
\eulerimg{17}{images/Hamemayu Hayuningrat_23030630053_EMTPlot2D-078.png}
\begin{eulerprompt}
>gridstyle("->", color = gray, textcolor = gray, framecolor = gray); ...
>plot2d("x^3 - x", grid = 1); ...
>settitle("y = x^3 - x", color = black); ...
>label("x", 2, 0, pos = "bc", color = gray); ...
>label("y", 0, 6, pos = "cl", color = gray); ...
>reset():
\end{eulerprompt}
\eulerimg{27}{images/Hamemayu Hayuningrat_23030630053_EMTPlot2D-079.png}
\begin{eulercomment}
Untuk kontrol lebih, sumbu-x dan sumbu-y dapat dilakukan secara
manual. Perintah fullwindow() menyebarkan jendela plot karena kita
tidak membutuhkan tempat untuk label luar pada judul plot. Gunakan
shrinkwindow() atau reset() untuk mengatur ulang ke pengaturan bawaan.
\end{eulercomment}
\begin{eulerprompt}
>fullwindow; ...
>gridstyle(color = darkgray, textcolor = darkgray); ...
>plot2d(["2^x", "1", "2^(-x)"], a=-2, b=2, c=0, d=4, <grid, color=4.6, <frame);...
>xaxis(0, -2:1, style="->"); xaxis(0, 2, "x", <axis); ...
>yaxis(0, 4, "y", style="->"); yaxis(-2, 1:4, >left); ...
>labelbox(["2^x", "1", "2^(-x)"], colors=4:6, x=0.8, y=0.2);...
>reset:
\end{eulerprompt}
\eulerimg{27}{images/Hamemayu Hayuningrat_23030630053_EMTPlot2D-080.png}
\begin{eulercomment}
Ini adalah contoh lainnya, dimana string Unicode digunakan dan sumbu
diluar luasan plot.
\end{eulercomment}
\begin{eulerprompt}
>aspect(1.5);
>plot2d(["sin(x)", "cos(x)"], 0, 2pi, color=[red, green], <grid, <frame); ...
>xaxis(-1.1, (0:2)*pi, xt=["0", u"&pi;", u"&pi;"], style="-", >ticks, >zero);...
>xgrid((0:0.5:2)*pi, <ticks);...
>yaxis(-0.1*pi, -1:0.2:1, style="-", >zero, >grid);...
>labelbox(["sin", "cos"], colors=[red, green], x=0.5, y=0.2, >left);...
>xlabel(u"&phi;"); ylabel(u"f(&phi;)"):
\end{eulerprompt}
\eulerimg{17}{images/Hamemayu Hayuningrat_23030630053_EMTPlot2D-081.png}
\eulerheading{Memploting data 2D}
\begin{eulercomment}
Jika x dan y adalah vektor-vektor data, data ini akan digunakan
sebagai koordinat x dan y dari kurva. Dalam kasus ini, a, b, c dan d
atau sebuah radius r dapat dispesifikkan atau jendela plot akan
menyesuaikan data secara otomatis.

Membuat plot ekspresi hanya sebuah singkatan untuk plot data. Untuk
plot data, kamu membutuhkan satu atau lebih barisan dari nilai-x dan
satu atau lebih barisan dari nilai-y. Dari rentang dan nilai-x fungsi
plot2d akan menghitung data untuk plot, secara bawaan dengan evaluasi
adaptasi dari fungsi. Untuk plot-plot titik gunakan "\textgreater{}points", untuk
campuran garis dan titik gunakan "\textgreater{}addpoints".

Tetapi kamu dapat memasukkan data secara langsung:\\
- Gunakan vektor baris untuk x dan y untuk satu fungsi.\\
- Matriks-matriks untuk x dan y diplotkan garis dengan garis.

Ini adalah contoh untuk satu baris untuk x dan y.
\end{eulercomment}
\begin{eulerprompt}
>x = -10:0.1:10; y = exp(-x^2)*x; plot2d(x, y):
\end{eulerprompt}
\eulerimg{17}{images/Hamemayu Hayuningrat_23030630053_EMTPlot2D-082.png}
\begin{eulercomment}
Data juga dapat diplotkan sebagai titik-titik. Gunakan points=true
untuk melakukan ini. Plot akan bekerja seperti poligon, tetapi
menggambar hanya sudut-sudutnya.\\
- style = "...": pilih dari "[]", "\textless{}\textgreater{}", "o", ".", "..", "+", "[]",
"\textless{}\textgreater{}", "o", "..", "", "\textbar{}".

Untuk barisan plot dari titik-titik gunakan \textgreater{}points. Jika warna adalah
sebuah vektor warna, setiap titik akan memiliki warna yang berbeda.
Untuk sebuah matriks koordinat dan vektor kolom, warna diaplikasikan
ke barisan dari matriks.

Parameter \textgreater{}addpoint menambahkan titik ke segmen bari untuk plot data.
\end{eulercomment}
\begin{eulerprompt}
>xdata = [1, 1.5, 2.5, 3, 4]; ydata = [3, 3.1, 2.8, 2.9, 2.7]; // data
>plot2d(xdata, ydata, a=0.5, b=4.5, c=2.5, d=3.5, style="."); // garis
>plot2d(xdata, ydata, >points, >add, style="o"): // menambahkan titik
\end{eulerprompt}
\eulerimg{17}{images/Hamemayu Hayuningrat_23030630053_EMTPlot2D-083.png}
\begin{eulerprompt}
>p = polyfit(xdata, ydata, 1); // mendapatkan garis regresi
>plot2d("polyval(p, x)", >add, color=red): // menambah plot garis
\end{eulerprompt}
\eulerimg{17}{images/Hamemayu Hayuningrat_23030630053_EMTPlot2D-084.png}
\eulerheading{Menggambar Daerah yang dibatasi Kurva}
\begin{eulercomment}
Plot-plot data memang benar-benar poligon. Kita dapat juga membuat
plot kurva atau kurva yang berisi.

- filled = true, akan mengisi plot.\\
- style = "...": memilih dari "", "/", "\textbackslash{}", "\textbackslash{}/".\\
- fillcolor: Lihat atas untuk warna yang tersedia.

Warna isian dapat ditentukan dengan argumen "fillcolor" dan secara
pilihan \textless{}outline mencegah membuat batasan dari setiap gaya tetapi
salah satunya bawaan.
\end{eulercomment}
\begin{eulerprompt}
>t = linspace(0, 2pi, 1000); // parameter untuk kurva
>x = sin(t) * exp(t/pi); y = cos(t) * exp(t/pi); // x(t) dan y(t)
>figure(1, 2); aspect(16/9);
>figure(1); plot2d(x, y, r=10); // kurva plot
>figure(2); plot2d(x, y, r=10, >filled, style="/", fillcolor=red); // kurva isian
>figure(0):
\end{eulerprompt}
\eulerimg{14}{images/Hamemayu Hayuningrat_23030630053_EMTPlot2D-085.png}
\begin{eulercomment}
Contoh berikut kami membut plot sebuah ellips yang terisi dan dua
segidelapan yang terisi menggunakan kurva tertutup dengan 6
titik-titik dengan gaya isian yang berbeda.
\end{eulercomment}
\begin{eulerprompt}
>x = linspace(0, 2pi, 1000); plot2d(sin(x), cos(x) * 0.5, r=1, >filled, style="/"):
\end{eulerprompt}
\eulerimg{14}{images/Hamemayu Hayuningrat_23030630053_EMTPlot2D-086.png}
\begin{eulerprompt}
>t = linspace(0, 2pi, 6); ...
>plot2d(cos(t), sin(t), >filled, style="/", fillcolor=red, r=1.2):
\end{eulerprompt}
\eulerimg{14}{images/Hamemayu Hayuningrat_23030630053_EMTPlot2D-087.png}
\begin{eulerprompt}
>t = linspace(0, 2pi, 6); plot2d(cos(t), sin(t), >filled, style="#"):
\end{eulerprompt}
\eulerimg{14}{images/Hamemayu Hayuningrat_23030630053_EMTPlot2D-088.png}
\begin{eulercomment}
Contoh lainnya adalah sebuah segi-7, yang mana kami membuat 7
titik-titik dalam lingkaran.
\end{eulercomment}
\begin{eulerprompt}
>t = linspace(0, 2pi, 7); ...
>plot2d(cos(t), sin(t), r=1, >filled, style="/", fillcolor=red):
\end{eulerprompt}
\eulerimg{14}{images/Hamemayu Hayuningrat_23030630053_EMTPlot2D-089.png}
\begin{eulercomment}
Contoh ini adalah barisan dari nilai maksimal dari empat kondisi
linear kurang atau sama dengan 3. Ini adalah A[k].v \textless{}= 3 untuk semua
barisan A. Untuk membuat sudut yang bagus, kita menggunakan n dengan
relatif yang besar.
\end{eulercomment}
\begin{eulerprompt}
>A = [2, 1; 1, 2; -1, 0; 0, -1];
>function f(x, y) := max([x, y].A');
>plot2d("f", r=4, level=[0;3], color=green, n=111):
\end{eulerprompt}
\eulerimg{14}{images/Hamemayu Hayuningrat_23030630053_EMTPlot2D-090.png}
\begin{eulercomment}
Titik utama dari bahasa matriks adalah membiarkannya untuk membuat
tabel  dari fungsi secara mudah.
\end{eulercomment}
\begin{eulerprompt}
>t = linspace(0, 2pi, 1000); x = cos(3*t); y = sin(4*t);
\end{eulerprompt}
\begin{eulercomment}
Kami sekarang memiliki vektor x dan y dari nilai-nilai. plot2d() dapat
membuat plot nilai-nilai ini sebagai sebuah kurva yang terkoneksi
dengan titik-titik. Plot dapat diisi. Dalam kasus ini menghasilkan
hasil yang bagus karena aturan berliku, yang mana digunakan untuk
pengisian.
\end{eulercomment}
\begin{eulerprompt}
>plot2d(x, y, <grid, <frame, >filled):
\end{eulerprompt}
\eulerimg{14}{images/Hamemayu Hayuningrat_23030630053_EMTPlot2D-091.png}
\begin{eulercomment}
Sebuah vektor dari interval-interval diplotkan dengan nilai-nilai x
sebagai wilayah antara batas bawah dan batas atas dari interval.

Ini dapat berguna untuk membuat galat plot dari komputasi. Tapi ini
dapat juga digunakan untuk galat statistik.
\end{eulercomment}
\begin{eulerprompt}
>t = 0:0.1:1;...
>plot2d(t, interval(t - random(size(t)), t + random(size(t))), style="|"); ...
>plot2d(t, t, >add):
\end{eulerprompt}
\eulerimg{14}{images/Hamemayu Hayuningrat_23030630053_EMTPlot2D-092.png}
\begin{eulercomment}
Jika x adalah vektor yang terurut dan y adalah interval vektor, lalu
plot2d akan diplotkan pada rentang yang terisi dari interval dalam
bidang. Gaya-gaya pengisian sama seperti gaya-gaya dari poligon.
\end{eulercomment}
\begin{eulerprompt}
>t = -1:0.01:1; x=~t - 0.01, t + 0.01~; y = x^3-x;
>plot2d(t, y):
\end{eulerprompt}
\eulerimg{14}{images/Hamemayu Hayuningrat_23030630053_EMTPlot2D-093.png}
\begin{eulercomment}
Sangat mungkin untuk mengisi nilai nilai untuk sebuah fungsi speifik.
Untuk ini, level harus sebuah matriks 2xn. Baris pertama adalah batas
bawah dan baris kedua mencakup batas atas.
\end{eulercomment}
\begin{eulerprompt}
>expr := "2*x^2+x*y+3*y^4+y"; // mendefinisikan ekspresi f(x, y)
>plot2d(expr, level=[0; 1], style="-", color=blue): // 0 <= f(x, y) <= 1
\end{eulerprompt}
\eulerimg{14}{images/Hamemayu Hayuningrat_23030630053_EMTPlot2D-094.png}
\begin{eulercomment}
Kami dapat juga mengisi rentang dengan nilai seperti\\
\end{eulercomment}
\begin{eulerformula}
\[
-1 \leq (x^2 + y^2)^2 - x^2 + y^2 \leq 0
\]
\end{eulerformula}
\begin{eulerprompt}
>plot2d("(x^2 + y^2)^2 - x^2 + y^2", r=1.2, level=[-1; 0], style="/"):
\end{eulerprompt}
\eulerimg{14}{images/Hamemayu Hayuningrat_23030630053_EMTPlot2D-095.png}
\begin{eulerprompt}
>plot2d("cos(x)", "sin(x)^3", xmin=0, xmax=2pi, >filled, style="/"):
\end{eulerprompt}
\eulerimg{14}{images/Hamemayu Hayuningrat_23030630053_EMTPlot2D-096.png}
\eulerheading{Grafik Fungsi Parametrik}
\begin{eulercomment}
Nilai x butuh tidak diurutkan. (x, y) secara sederhana menggambarkan
sebuah kurva. Jika x terurut, kurva adalah sebuah grafik dari sebuah
fungsi. Contoh ini, kami akan membuat plot spiral

\end{eulercomment}
\begin{eulerformula}
\[
\gamma(t) = t \cdot (cos(2\pi t), sin(2 \pi t))
\]
\end{eulerformula}
\begin{eulercomment}
Kami juga membutuhkan banyak titik-titik untuk memperhalus atau fungsi
adaptive() untuk mengevaluasi ekspresi (lihat fungsi adaptive() untuk
lebih detail).
\end{eulercomment}
\begin{eulerprompt}
>t = linspace(0, 1, 1000); ...
>plot2d(t * cos(2*pi*t), t*sin(2*pi*t), r=1):
\end{eulerprompt}
\eulerimg{14}{images/Hamemayu Hayuningrat_23030630053_EMTPlot2D-097.png}
\begin{eulercomment}
Sebagai alternatif, yang memungkinkan untuk menggunakan 2 ekspresi
untuk kurva. Contoh plot ini sama seperti kurva diatas.
\end{eulercomment}
\begin{eulerprompt}
>plot2d("x*cos(2*pi*x)", "x*sin(2*pi*x)", xmin=0, xmax=1, r=1):
\end{eulerprompt}
\eulerimg{14}{images/Hamemayu Hayuningrat_23030630053_EMTPlot2D-098.png}
\begin{eulerprompt}
>t = linspace(0, 1, 1000); r = exp(-t); x = r*cos(2pi * t); y = r*sin(2pi * t);
>plot2d(x, y, r=1):
\end{eulerprompt}
\eulerimg{14}{images/Hamemayu Hayuningrat_23030630053_EMTPlot2D-099.png}
\begin{eulercomment}
Contoh selanjutnya, kami membuat plot dari kurva

\end{eulercomment}
\begin{eulerformula}
\[
\gamma(t) = (r(t)cos(t), r(t)sin(t))
\]
\end{eulerformula}
\begin{eulercomment}
dengan

\end{eulercomment}
\begin{eulerformula}
\[
r(t) = 1 + \frac{sin(3t)}{2}
\]
\end{eulerformula}
\begin{eulerprompt}
>t = linspace(0, 2pi, 1000); r = 1+sin(3*t) / 2; x = r*cos(t); y = r*sin(t); ...
>plot2d(x, y, >filled, fillcolor=red, style="/", r=1.5):
\end{eulerprompt}
\eulerimg{14}{images/Hamemayu Hayuningrat_23030630053_EMTPlot2D-100.png}
\eulerheading{Menggambar Grafik Bilangan Kompleks}
\begin{eulercomment}
Sebuah array dari angka-angka kompleks juga dapat dibuat plot. Lalu
titik-titik batas akan terkoneksi. Jika sebuah angkka dari batas garis
yang dispesifikasikan (atau sebuah 1x2 vektor dari batas garis) dalam
sebuah argumen cgrid hanya terlihat baris batas ini.

Sebuah matriks dari angka-angka kompleks akan secara otomatis
diplotkan sebagai sebuah batas dari bidang kompleks.

Sebagai contoh, kami membuat plot gambar dengan lingkaran unit dari
fungsi eksponensial. Parameter cgrid menyembunyikan beberapa kurva
batasnya.
\end{eulercomment}
\begin{eulerprompt}
>aspect(); r = linspace(0, 1, 50); a = linspace(0, 2pi, 80)'; z = r*exp(I*a); ...
>plot2d(z, a=-1.25, b=1.25, c=-1.25, d=1.25, cgrid=10):
\end{eulerprompt}
\eulerimg{27}{images/Hamemayu Hayuningrat_23030630053_EMTPlot2D-101.png}
\begin{eulerprompt}
>aspect(1.25); r = linspace(0, 1, 50); a = linspace(0, 2pi, 200)'; z = r*exp(I*a);
>plot2d(exp(z), cgrid=[40, 10]):
\end{eulerprompt}
\eulerimg{21}{images/Hamemayu Hayuningrat_23030630053_EMTPlot2D-102.png}
\begin{eulerprompt}
>r = linspace(0, 1, 10); a = linspace(0, 2pi, 40)'; z = r*exp(I*a);
>plot2d(exp(z), >points, >add):
\end{eulerprompt}
\eulerimg{21}{images/Hamemayu Hayuningrat_23030630053_EMTPlot2D-103.png}
\begin{eulercomment}
Sebuah vektor dari angka-angka kompleks secara otomatis terplotkan
sebagai sebuah kurva dalam bidang kompleks dengan bagian rill dan
bagian imajiner.

Contoh berikut, kami membuat plot dengan unit lingkaran dengan\\
\end{eulercomment}
\begin{eulerformula}
\[
\gamma(t) = e^{it}
\]
\end{eulerformula}
\begin{eulerprompt}
>t = linspace(0, 2pi, 1000); ...
>plot2d(exp(I*t) + exp(4*I*t), r=2):
\end{eulerprompt}
\eulerimg{21}{images/Hamemayu Hayuningrat_23030630053_EMTPlot2D-104.png}
\eulersubheading{Plot-Plot Statistik}
\begin{eulercomment}
Terdapat banyak fungsi yang terspesialisasikan untuk plot-plot
statistik. Salah satu yang sering digunakan adalah plot kolom.

Sebuah jumlahan kumulatif dari sebuah 0-1 distribusi nilai normal
menghasilkan pergerakan secara acak.
\end{eulercomment}
\begin{eulerprompt}
>plot2d(cumsum(randnormal(1, 1000))):
\end{eulerprompt}
\eulerimg{21}{images/Hamemayu Hayuningrat_23030630053_EMTPlot2D-105.png}
\begin{eulercomment}
Menggunakan dua baris untuk menunjukkan dua dimensi.
\end{eulercomment}
\begin{eulerprompt}
>X = cumsum(randnormal(2, 1000)); plot2d(X[1], X[2]):
\end{eulerprompt}
\eulerimg{21}{images/Hamemayu Hayuningrat_23030630053_EMTPlot2D-106.png}
\begin{eulerprompt}
>columnsplot(cumsum(random(10)), style="/", color=blue):
\end{eulerprompt}
\eulerimg{21}{images/Hamemayu Hayuningrat_23030630053_EMTPlot2D-107.png}
\begin{eulercomment}
Ini dapat juga menunjukkan string sebagai label.
\end{eulercomment}
\begin{eulerprompt}
>months = ["Jan", "Feb", "Mar", "Apr", "May", "Jun", ...
>"Jul", "Aug", "Sep", "Oct", "Nov", "Dec"];
>values = [10, 12, 12, 18, 22, 28, 30, 26, 22, 18, 12, 8];
>columnsplot(values, lab=months, color=red, style="-");
>title("Temperature"):
\end{eulerprompt}
\eulerimg{21}{images/Hamemayu Hayuningrat_23030630053_EMTPlot2D-108.png}
\begin{eulerprompt}
>k = 0:10;
>plot2d(k, bin(10, k), >bar):
\end{eulerprompt}
\eulerimg{21}{images/Hamemayu Hayuningrat_23030630053_EMTPlot2D-109.png}
\begin{eulerprompt}
>plot2d(k, bin(10, k)); plot2d(k, bin(10, k), >points, >add):
\end{eulerprompt}
\eulerimg{21}{images/Hamemayu Hayuningrat_23030630053_EMTPlot2D-110.png}
\begin{eulerprompt}
>plot2d(normal(1000), normal(1000), >points, grid=6, style="-"):
\end{eulerprompt}
\eulerimg{21}{images/Hamemayu Hayuningrat_23030630053_EMTPlot2D-111.png}
\begin{eulerprompt}
>plot2d(normal(1, 1000), >distribution, style="O"):
\end{eulerprompt}
\eulerimg{21}{images/Hamemayu Hayuningrat_23030630053_EMTPlot2D-112.png}
\begin{eulerprompt}
>plot2d("qnormal", 0, 5; 2.5, 0.5, >filled):
\end{eulerprompt}
\eulerimg{21}{images/Hamemayu Hayuningrat_23030630053_EMTPlot2D-113.png}
\begin{eulercomment}
Untuk plot sebuah distribusi eksperimen statistik, kamu juga dapat
menggunakan distribution=n dengan plot2d.
\end{eulercomment}
\begin{eulerprompt}
>w = randexponential(1, 1000); // distribusi eksponensial
>plot2d(w, >distribution): // atau distribution=n dengan interval n
\end{eulerprompt}
\eulerimg{21}{images/Hamemayu Hayuningrat_23030630053_EMTPlot2D-114.png}
\begin{eulercomment}
Atau kamu juga menghitung bentuk distribusi data dan plot hasil dengan
\textgreater{}bar dalam plot3d, atau dengan sebuah plot kolom.
\end{eulercomment}
\begin{eulerprompt}
>w = normal(1000); // 0-1-distribusi normal
>\{x, y\} = histo(w, 10, v=[-6, -4, -2, -1, 0, 1, 2, 4, 6]); // batas interval v
>plot2d(x, y, >bar):
\end{eulerprompt}
\eulerimg{21}{images/Hamemayu Hayuningrat_23030630053_EMTPlot2D-115.png}
\begin{eulercomment}
fungsi statplot() mengatur gaya dengan string yang sederhana.
\end{eulercomment}
\begin{eulerprompt}
>statplot(1:10, cumsum(random(10)), "b"):
\end{eulerprompt}
\eulerimg{21}{images/Hamemayu Hayuningrat_23030630053_EMTPlot2D-116.png}
\begin{eulerprompt}
>n = 10; i = 0:n; ...
>plot2d(i, bin(n, i) / 2^n, a=0, b=10, c=0, d=0.3);...
>plot2d(i, bin(n, i) / 2^n, >points, >add, style="ow", color=blue):
\end{eulerprompt}
\eulerimg{21}{images/Hamemayu Hayuningrat_23030630053_EMTPlot2D-117.png}
\begin{eulercomment}
Selebihnya, data dapat diplotkan sebagai baris. Dalam kasus ini, x
akan diurutkan dan satu elemen lebih panjang daripada y. Barang akan
memanjang dari x[i] ke x[i+1] dengan nilai y[i]. Jika x memiliki
ukuran yang sama seperti y, akan memanjang dengan satu elemen dengan
spasi dibagian akhir.

Gaya isian akan digunakan seperti diatas.
\end{eulercomment}
\begin{eulerprompt}
>n = 10; k = bin(n, 0:n); ...
>plot2d(-0.5:n + 0.5, k, >bar, fillcolor=lightgray):
\end{eulerprompt}
\eulerimg{21}{images/Hamemayu Hayuningrat_23030630053_EMTPlot2D-118.png}
\begin{eulercomment}
Data untuk plot batang (bar = 1) dan histogram (histogram = 1) dapat
secara eksplisit diberikan dalam xv dan yv atau dapat dihitung dari
sebuah distribusi empiris dalam xv dengan \textgreater{}distribution (or
distribution = n). Histogram dari nilai xv dapat dihitung secara
otomatis dengan \textgreater{}histogram. Jika \textgreater{}even secara spesifik, nilai xv akan
dihitung interval bilangan bulat.
\end{eulercomment}
\begin{eulerprompt}
>plot2d(normal(10000, distribution=50)):
\end{eulerprompt}
\eulerimg{21}{images/Hamemayu Hayuningrat_23030630053_EMTPlot2D-119.png}
\begin{eulerprompt}
>k = 0:10; m = bin(10, k); x = (0:11)-0.5; plot2d(x, m, >bar):
\end{eulerprompt}
\eulerimg{21}{images/Hamemayu Hayuningrat_23030630053_EMTPlot2D-120.png}
\begin{eulerprompt}
>columnsplot(m, k):
\end{eulerprompt}
\eulerimg{21}{images/Hamemayu Hayuningrat_23030630053_EMTPlot2D-121.png}
\begin{eulerprompt}
>plot2d(random(600) * 6, histogram=6):
\end{eulerprompt}
\eulerimg{21}{images/Hamemayu Hayuningrat_23030630053_EMTPlot2D-122.png}
\begin{eulercomment}
Untuk distribusi, parameter distribusi = n ini, yang mana menghitung
nilai-nilai secara otomatis dan mencetak distribusi relatif dengan n
sub-interval.
\end{eulercomment}
\begin{eulerprompt}
>plot2d(normal(1,1000),distribution=10,style="\(\backslash\)/"):
\end{eulerprompt}
\eulerimg{21}{images/Hamemayu Hayuningrat_23030630053_EMTPlot2D-123.png}
\begin{eulercomment}
Dengan parameter even=true, ini akan menggunakan interval bilangan
bulat.
\end{eulercomment}
\begin{eulerprompt}
>plot2d(intrandom(1, 1000, 10), distribution=10, >even):
\end{eulerprompt}
\eulerimg{21}{images/Hamemayu Hayuningrat_23030630053_EMTPlot2D-124.png}
\begin{eulercomment}
Catat bahwa terdapat banyak plot statistik, yang mana mungkin akan
berguna. Lihatlah tutorial tentang statistik.
\end{eulercomment}
\begin{eulerprompt}
>columnsplot(getmultiplicities(1:6, intrandom(1, 6000, 6))):
\end{eulerprompt}
\eulerimg{21}{images/Hamemayu Hayuningrat_23030630053_EMTPlot2D-125.png}
\begin{eulerprompt}
>plot2d(normal(1, 1000), >distribution); ...
>plot2d("qnormal(x)", color=red, thickness=2, >add):
\end{eulerprompt}
\eulerimg{21}{images/Hamemayu Hayuningrat_23030630053_EMTPlot2D-126.png}
\begin{eulercomment}
Ini juga terdapat banyak plot spesial untuk statistik. Sebuah boxplot
memperlihatkan kuartil dari distribusi ini dan banyak luaran. Dengan
definisi, luaran dalam sebuah boxplot adalah data yang melebihi 1.5
kali dari tengahan 50\% rentang dari plot.
\end{eulercomment}
\begin{eulerprompt}
>M = normal(5, 1000); boxplot(quartiles(M)):
\end{eulerprompt}
\eulerimg{21}{images/Hamemayu Hayuningrat_23030630053_EMTPlot2D-127.png}
\eulersubheading{Fungsi-Fungsi Implisit}
\begin{eulercomment}
Plot implisit menunjukkan barisan level dengan menyelesaikan f(x, y) =
level, dimana "level" dapat berupa nilai tunggal atau sebuah vektor
dari nilai. Jika level="auto", ini akan menjadi nc baris level, yang
mana akan menyebar diantara nilai minimum dan maksimum dari fungsi.
Gelap atau terangnya warna dapat ditambahkan dengan \textgreater{}huw untuk
menunjukkan nilai dari fungsi. Untuk fungsi implisit, xv harus menjadi
sebuah fungsi dari paramter x dan y, atau, secara alternatif, xv dapat
menjadi sebuah matriks dari nilai.

Euler dapat menandai garis level\\
\end{eulercomment}
\begin{eulerformula}
\[
f(x, y) = c
\]
\end{eulerformula}
\begin{eulercomment}
dari sebarang fungsi

Untuk menggambar himpunan f(x, y) = c untuk satu atau lebih konstan c
kamu dapat menggunakan plot2d() dengan ini plot implisit dalam bidang.
Parameter untuk c adalah level = c, dimana c dapat berupa vektor dari
baris level. Dengan tambahan, sebuah skema warna dapat digambarkan
dalam latar belakang untuk mengindikasi nilai dari fungsi untuk setiap
titik dalam plot. Parameter "n" menentukan kemulusan dari plot.
\end{eulercomment}
\begin{eulerprompt}
>aspect(1.5);
>plot2d("x^2 + y^2 - x*y - x", r=1.5, level=0, contourcolor=red):
\end{eulerprompt}
\eulerimg{17}{images/Hamemayu Hayuningrat_23030630053_EMTPlot2D-128.png}
\begin{eulerprompt}
>expr := "2*x^2 + x*y + x*y^4 + y"; // mendefinisikan sebuah ekspresi f(x, y)
>plot2d(expr, level=0): // solusi dari f(x, y) = 0
\end{eulerprompt}
\eulerimg{17}{images/Hamemayu Hayuningrat_23030630053_EMTPlot2D-129.png}
\begin{eulerprompt}
>plot2d(expr, level=0:0.5:20, >hue, contourcolor=white, n=200): // bagus
\end{eulerprompt}
\eulerimg{17}{images/Hamemayu Hayuningrat_23030630053_EMTPlot2D-130.png}
\begin{eulerprompt}
>plot2d(expr, level=0:0.5:20, >hue, >spectral, n=200, grid=4): // lebih bagus
\end{eulerprompt}
\eulerimg{17}{images/Hamemayu Hayuningrat_23030630053_EMTPlot2D-131.png}
\begin{eulercomment}
Ini bekerja untuk data plot juga. Tetapi kamu harus menspesifikasikan
rentang untuk sumbu-sumbu label.
\end{eulercomment}
\begin{eulerprompt}
>x = -2:0.05:1; y = x'; z = expr(x, y);
>plot2d(z, level=0, a=-1, b=2, c=-2, d=1, >hue):
\end{eulerprompt}
\eulerimg{17}{images/Hamemayu Hayuningrat_23030630053_EMTPlot2D-132.png}
\begin{eulerprompt}
>plot2d("x^3-y^2", >contour, >hue, >spectral):
\end{eulerprompt}
\eulerimg{17}{images/Hamemayu Hayuningrat_23030630053_EMTPlot2D-133.png}
\begin{eulerprompt}
>plot2d("x^3-y^2", level=0, contourwidth=3, >add, contourcolor=red):
\end{eulerprompt}
\eulerimg{17}{images/Hamemayu Hayuningrat_23030630053_EMTPlot2D-134.png}
\begin{eulerprompt}
>z = z + normal(size(z)) * 0.2;
>plot2d(z, level=0.5, a=-1, b=2, c=-2, d=1):
\end{eulerprompt}
\eulerimg{17}{images/Hamemayu Hayuningrat_23030630053_EMTPlot2D-135.png}
\begin{eulerprompt}
>plot2d(expr, level=[0:0.2:5; 0.05:0.2:5.05], color=lightgray):
\end{eulerprompt}
\eulerimg{17}{images/Hamemayu Hayuningrat_23030630053_EMTPlot2D-136.png}
\begin{eulerprompt}
>plot2d("x^2 + y^3 + x*y", level=1, r=4, n=100):
\end{eulerprompt}
\eulerimg{17}{images/Hamemayu Hayuningrat_23030630053_EMTPlot2D-137.png}
\begin{eulerprompt}
>plot2d("x^2 + 2*y^2 - x*y", level=0:0.1:10, n=100, contourcolor=white, >hue):
\end{eulerprompt}
\eulerimg{17}{images/Hamemayu Hayuningrat_23030630053_EMTPlot2D-138.png}
\begin{eulercomment}
Ini juga memungkinkan untuk mengisi himpunan

\end{eulercomment}
\begin{eulerformula}
\[
a \leq f(x, y) \leq b
\]
\end{eulerformula}
\begin{eulercomment}
dengan sebuah rentang level.

Ini memungkinkan untuk mengisi wilayah dari nilai-nilai untuk fungsi
tertentu. Untuk ini, level haruslah sebuah matriks 2xn. Baris pertama
adalah batasan bawah dan baris kedua mengandung batasan atas.
\end{eulercomment}
\begin{eulerprompt}
>plot2d(expr, level=[0; 1], style="-", color=blue): // 0 <= f(x, y) <= 1
\end{eulerprompt}
\eulerimg{17}{images/Hamemayu Hayuningrat_23030630053_EMTPlot2D-139.png}
\begin{eulercomment}
Plot implisit juga dapat memperlihatkan rentang dari level. Lalu level
haruslah sebuah matriks 2xn dari interval level, dimana baris pertama
mengandung awal dan baris kedua adalah akhir dari setiap interval.
Sebagai gantinya, sebuah baris vektor dapat digunakan untuk level, dan
sebuah dl parameter meluaskan nilai level untuk interval.
\end{eulercomment}
\begin{eulerprompt}
>plot2d("x^4 + y^4", r=1.5, level=[0;1], color=blue, style="/"):
\end{eulerprompt}
\eulerimg{17}{images/Hamemayu Hayuningrat_23030630053_EMTPlot2D-140.png}
\begin{eulerprompt}
>plot2d("x^2 + y^3 + x*y", level=[0, 2, 4; 1, 3, 5], style="/", r=2, n=100):
\end{eulerprompt}
\eulerimg{17}{images/Hamemayu Hayuningrat_23030630053_EMTPlot2D-141.png}
\begin{eulerprompt}
>plot2d("x^2 + y^3 + x*y", level=-10:20, r=2, style="-", dl=0.1, n=100):
\end{eulerprompt}
\eulerimg{17}{images/Hamemayu Hayuningrat_23030630053_EMTPlot2D-142.png}
\begin{eulerprompt}
>plot2d("sin(x) * cos(y)", r=pi, >hue, >levels, n=100):
\end{eulerprompt}
\eulerimg{17}{images/Hamemayu Hayuningrat_23030630053_EMTPlot2D-143.png}
\begin{eulercomment}
Ini juga memungkinkan untuk menandai sebuah wilayah

\end{eulercomment}
\begin{eulerformula}
\[
a \leq f(x, y) \leq b
\]
\end{eulerformula}
\begin{eulercomment}
ini dapat dilakukan dengan menambahkan sebuah level dengan 2 baris.
\end{eulercomment}
\begin{eulerprompt}
>plot2d("(x^2 + y^2 - 1)^3 - x^2*y^3", r=1.3,...
>style="#", color=red, <outline, ...
>level=[-2; 0], n=100):
\end{eulerprompt}
\eulerimg{17}{images/Hamemayu Hayuningrat_23030630053_EMTPlot2D-144.png}
\begin{eulercomment}
Ini juga memungkinkan untuk secara level spesifik. Sebagai contoh,
kami akan membuat plot solusi dari sebuah persamaan seperti

\end{eulercomment}
\begin{eulerformula}
\[
x^3 - xy + x^2y^2 = 6
\]
\end{eulerformula}
\begin{eulerprompt}
>plot2d("x^3 - x*y + x^2 * y^2", r=6, level=1, n=100):
\end{eulerprompt}
\eulerimg{17}{images/Hamemayu Hayuningrat_23030630053_EMTPlot2D-145.png}
\begin{eulerprompt}
>function starplot1 (v, style="/", color=green, lab=none) ...
\end{eulerprompt}
\begin{eulerudf}
    if !holding() then clg; endif;
    w=window(); window(0,0,1024,1024);
    h=holding(1);
    r=max(abs(v))*1.2;
    setplot(-r,r,-r,r);
    n=cols(v); t=linspace(0,2pi,n);
    v=v|v[1]; c=v*cos(t); s=v*sin(t);
    cl=barcolor(color); st=barstyle(style);
    loop 1 to n
      polygon([0,c[#],c[#+1]],[0,s[#],s[#+1]],1);
      if lab!=none then
        rlab=v[#]+r*0.1;
        \{col,row\}=toscreen(cos(t[#])*rlab,sin(t[#])*rlab);
        ctext(""+lab[#],col,row-textheight()/2);
      endif;
    end;
    barcolor(cl); barstyle(st);
    holding(h);
    window(w);
  endfunction
\end{eulerudf}
\begin{eulercomment}
Tidak ada titik-titik atau batasan sumbu disini. Selebihnya, kami
gunakan dengan jendela penuh untuk plot. Kami memanggil reset
sebelumnya, kami test plot ini untuk mengatur ulang grafik menjadi
seperti bawaan. Ini tidak dibutuhkan, jika kamu memastikan bahwa plot
kamu berkerja.
\end{eulercomment}
\begin{eulerprompt}
>reset; starplot1(normal(1, 10) + 5, color=red, lab=1:10):
\end{eulerprompt}
\eulerimg{27}{images/Hamemayu Hayuningrat_23030630053_EMTPlot2D-146.png}
\begin{eulercomment}
Terkadang, kamu ingin untuk membuat plot sesuatu dengan plot2d tidak
dapat melakukannya, tetapi hampir. Contoh fungsi berikut, kami
melakukan plot impuls logaritmik. plot2d juga dapat mem plot
logaritmik, tetapi tidak untuk batang impuls.
\end{eulercomment}
\begin{eulerprompt}
>function logimpulseplot1 (x,y) ...
\end{eulerprompt}
\begin{eulerudf}
    \{x0,y0\}=makeimpulse(x,log(y)/log(10));
    plot2d(x0,y0,>bar,grid=0);
    h=holding(1);
    frame();
    xgrid(ticks(x));
    p=plot();
    for i=-10 to 10;
      if i<=p[4] and i>=p[3] then
         ygrid(i,yt="10^"+i);
      endif;
    end;
    holding(h);
  endfunction
\end{eulerudf}
\begin{eulercomment}
Mari kita tes ini dengan nilai yang terdistribusi secara eksponensial.

\end{eulercomment}
\begin{eulerprompt}
>aspect(1.5); x = 1:10; y = -log(random(size(x))) * 200;...
>logimpulseplot(x, y):
\end{eulerprompt}
\eulerimg{17}{images/Hamemayu Hayuningrat_23030630053_EMTPlot2D-147.png}
\begin{eulercomment}
Mari kita animasikan sebuah kurva 2D menggunakan plot langsung.
Perintah plot(x, y) secara sederhana membuat sebuah plot menjadi
jendela plot. setplot(a, b, c, d) mengatur jendela ini.

Fungsi wait(0) memaksa plot untuk muncul dalam gambar windows.
Sebaliknya, menggambar ulang mengambil tempat interval nilai untuk
membuatnya.
\end{eulercomment}
\begin{eulerprompt}
>function animliss(n, m) ...
\end{eulerprompt}
\begin{eulerudf}
  t=linspace(0,2pi,500);
  f=0;
  c=framecolor(0);
  l=linewidth(2);
  setplot(-1,1,-1,1);
  repeat
    clg;
    plot(sin(n*t),cos(m*t+f));
    wait(0);
    if testkey() then break; endif;
    f=f+0.02;
  end;
  framecolor(c);
  linewidth(l);
  endfunction
\end{eulerudf}
\begin{eulercomment}
Tekan tombol apapun untuk menghentikan animasi ini.
\end{eulercomment}
\begin{eulerprompt}
>animliss(2, 3); // lihat hasilnya, jika sudah puas, tekan ENTER
\end{eulerprompt}
\eulerheading{Plot-Plot  Logaritmik}
\begin{eulercomment}
EMT menggunakan parameter "logplot" untuk skala logaritmik.

Plot logaritmik dapat diplotkan dengan menggunakan skala logaritmik
dalam y dengan logplot=1 atau menggunakan skala logaritmik dalam x dan
y dengan logplot=2 atau dalam x dengan logplot=3.

- logplot=1: logaritmik-y\\
- logplot=2: logaritmik-x-y\\
- logplot=3: logaritmik-y
\end{eulercomment}
\begin{eulerprompt}
>plot2d("exp(x^3-x)*x^2", 1, 5, logplot=1):
\end{eulerprompt}
\eulerimg{17}{images/Hamemayu Hayuningrat_23030630053_EMTPlot2D-148.png}
\begin{eulerprompt}
>plot2d("exp(x + sin(x))", 0, 100, logplot=1):
\end{eulerprompt}
\eulerimg{17}{images/Hamemayu Hayuningrat_23030630053_EMTPlot2D-149.png}
\begin{eulerprompt}
>plot2d("exp(x + sin(x))", 10, 100, logplot=2):
\end{eulerprompt}
\eulerimg{17}{images/Hamemayu Hayuningrat_23030630053_EMTPlot2D-150.png}
\begin{eulerprompt}
>plot2d("gamma(x)", 1, 10, logplot=1):
\end{eulerprompt}
\eulerimg{17}{images/Hamemayu Hayuningrat_23030630053_EMTPlot2D-151.png}
\begin{eulerprompt}
>plot2d("log(x * (2 + sin(x / 100)))", 10, 1000, logplot=3):
\end{eulerprompt}
\eulerimg{17}{images/Hamemayu Hayuningrat_23030630053_EMTPlot2D-152.png}
\begin{eulercomment}
Ini juga dapat bekerja dengan plot-plot data.
\end{eulercomment}
\begin{eulerprompt}
>x = 10^(1:20); y = x^2 - x;
>plot2d(x, y, logplot=2):
\end{eulerprompt}
\eulerimg{17}{images/Hamemayu Hayuningrat_23030630053_EMTPlot2D-153.png}
\eulerheading{Rujukan Lengkap Fungsi plot2d()}
\begin{eulercomment}
function plot2d (xv, yv, btest, a, b, c, d, xmin, xmax, r, n,..\\
logplot, grid, frame, framecolor, square, color, thickness, style,..\\
auto, add, user, delta, points, addpoints, pointstyle, bar,
histogram,..\\
distribution, even, steps, own, adaptive, hue, level, contour,  ..\\
nc, filled, fillcolor, outline, title, xl, yl, maps, contourcolor, ..\\
contourwidth, ticks, margin, clipping, cx, cy, insimg, spectral,  ..\\
cgrid, vertical, smaller, dl, niveau, levels)

fungsi plot serbaguna untuk membuat plot dalam bidang (plot 2D).
Fungsi ini dapat melakukan plotting fungsi satu variabel, plot data,
kurva bidang, plot garis, batas-batas angka kompleks, dan plot
implisit dari fungsi dua variabel.

Parameter\\
\end{eulercomment}
\begin{eulerttcomment}
 x, y       : persamaan, fungsi atau vektor data
 a, b, c, d : luasan plot (secara bawaan a = -2, b = 2)
 r          : jika r diatur, maka a=cx-r, b=cx+r, c=cy-r, d=cy+r
              r dapat berupa sebuah vektor [rx, ry] atau sebuah vektor
              [rx1, rx2, ry1, ry2]
 xmin,xmax  : selang dari parameter untuk kurva
 auto       : Menentukan selang-y secara otomatis (bawaan)
 square     : jika true, akan mempertahankan kotak selang-x-y
 n          : jumlah rentang interval (adaptif secara default)
 grid       : 0 = tidak mempunyai batas dan label,
              1 = hanya sumbu,
              2 = batas normal(lihat bawah untuk jumlah baris batasan)
              3 = didalam sumbu
              4 = tidak ada batas
              5 = batas penuh termasuk margin
              6 = titik bantu pada frame
              7 = hanya sumbu
              8 = hanya sumbu, sub-titik bantu
 frame      : 0 = tidak ada frame
 framecolor : warna dari frame dan batas
 margin     : angka antara 0 dan 0.4 untuk margin sekeliling plot
 color      : Warna kurva. Jika ini adalah sebuah vektor warna, ini
              akan digunakan pada setiap baris dari sebuah plot
              matriks.
              Dalam kasus ini titik-titik plot, seharusnya sebuah
              vektor kolom. Jika sebuag baris vektor atau sebuah
              matriks warna lengkap yang digunakan untuk titik plot,
              ini akan digunakan untuk setiap titik data.
 thickness  : ketebalan garis dari kurva
              Nilai ini dapat lebih kecil dari 1 untuk garis yang
              sangat tipis.
 style      : Gaya plot untuk garis, tanda dan isian.
              Untuk titik gunakan
             "[]", "<>", ".", "..", "...",
             "*", "+", "|", "-", "o"
             "[]#", "<>#", "o#" (bentuk isian)
             "[]w", "<>w", "ow" (tidak-transparan)
             Untuk penggunaan garis
             "-", "--", "-.", ".", ".-.", "-.-", "->"
             Untuk penggunaan isian poligon atau plot garis
             "#", "#O", "O", "/", "\(\backslash\)", "\(\backslash\)/",
             "+", "|", "-", "t"
 points    : membuat plot titik tunggal daripada baris segmen
 addpoints : Jika true, segmen garis plot dan titik
 add       : menambahkan plot kedalam plot yang telah ada
 user      : membuat pengguna untuk berinteraksi pada fungsi
 delta     : ukuran langkah untuk interaksi pengguna
 bar       : plot batang (x adalah batas interval, y nilai interval)
 histogram : plot frekuensi dari x dalam subinterval n
 distribution=n : plot distribusi dari x dalam subinterval n
 even      : gunakan dalam nilai untuk histogram otomatis
 steps     : plot  fungsi sebagai sebuah fungsi langkah (steps=1,2)
 adaptive  : penggunaan plot adaptif (n adalah jumlah minimal langkah)
 level     : plot garis level dari sebuah fungsi implisit dua variabel
 outline   : menggambar batasan dari selang level
\end{eulerttcomment}
\begin{eulercomment}

\end{eulercomment}
\begin{eulerttcomment}
  Jika nilai level adalah sebuah matriks 2xn, selang dari level akan
\end{eulerttcomment}
\begin{eulercomment}
digambarkan dalam warna menggunakan gaya isian. Jika luaran bernilai
true, ini akan digambarkan sebagai warna kontur. Gunakan fitur ini,
wilayah dari f(x, y) diantara limit dapat dicatat.

\end{eulercomment}
\begin{eulerttcomment}
 hue       : tambahkan warna hue untuk plot level untuk
             mengindikasikan nilai fungsi
 contour   : Gunakan plot level dengan level otomatis
 nc        : Banyaknya baris level otomatis
 title     : Judul plot (bawaan "")
 xl, yl    : label untuk sumbu-x dan sumbu-y
 smaller   : jika > 0, akan memberi ruang lebih ke kiri untuk label
 vertical  :
   Mengubah arah label secara vertikal. Ini mengubah variabel global
   verticallabel secara local pada satu plot. Nilai 1 mengatur hanya
   teks vertikal, nilai 2 digunakan untuk label numerik vertical pada
   sumbu y.
 filled    : mengisi plot dari sebuah kurva
 fillcolor : mengisi warna untuk batang dan kurva yang terisi
 outline   : batasan untuk poligon yang terisi
 logplot   : mengatur plot logaritmik
             1 = plot logartimik dalam y,
             2 = plot logaritmik dalam xy,
             3 = plot logaritmik dalam x
 own       :
   Sebuah string, yang mana titik-titik ke sebuah rutin plot sendiri.
   Dengan >user, kamu akan mendapatkan interaksi pengguna seperti
   dalam plot2d. Rentang akan diatur sebelum tiap-tiap panggilan
   pada fungsi.
 maps      : ekspresi peta (0 tercepat), fungsi selalu terpetakan
 contourcolor : warna dari garis kontur
 contourwidth : tebal dari garis kontur
 clipping  : menggalihkan cetakan (bawaan adalah true)
 title     :
   Ini dapat digunakan untuk mendeskripsikan plot. Judul akan muncul
   diatas plot. Terlebih, sebuah label untuk sumbu x dan y akan
   ditambahkan dengan xl="string" atau yl="string". Label lainnya akan
   ditambahkan dengan fungsi label() atau labelbox(). Judul dapat
   berupa sebuah string unicode atau sebuah gambar dari sebuah formula
   Latex.
 cgrid     :
   Menentukan banyaknya baris batasan untuk plot dari batasan
   kompleks. Seharusnya sebuah pembagi dari ukuran matriks
   dikurangi 1 (banyak nya sub-interval). cgrid dapat berupa
   sebuah vektor [cx, cy].
\end{eulerttcomment}
\begin{eulercomment}

\end{eulercomment}
\begin{eulerttcomment}
 Pratinjau
\end{eulerttcomment}
\begin{eulercomment}

\end{eulercomment}
\begin{eulerttcomment}
 Fungsi ini dapat membuat plot
\end{eulerttcomment}
\begin{eulercomment}

\end{eulercomment}
\begin{eulerttcomment}
 - Ekspresi, pemanggilan collection atau fungsi dari satu variabel,
 - Kurva parametrik,
 - data x beserta data y,
 - fungsi implisit,
 - plot batang,
 - batasan kompleks,
 - poligon.
\end{eulerttcomment}
\begin{eulercomment}

\end{eulercomment}
\begin{eulerttcomment}
  Jika sebuah fungsi atau ekspresi untuk xv diberikan, plot2d akan
  menghitung nilai yang rentang berikan menggunakan fungsi atau
  ekspresi. Ekspresi haruslah sebuah ekspresi dalam variabel x.
  Selang haruslah didefinisikan dalam parameter a dan b kecuali
  memiliki rentang bawaan [-2, 2] akan digunakan. Selang-y akan
  secara otomatis dihitung, kecuali c dan d dispesifikasikan, atau
  sebuah jari jari r, yang mana menghasilkan selang [-r, r] untuk
  x dan y. Untuk plot dari fungsi, plot2d akan menggunakan evaluasi
  adaptif dari fungsi secara bawaan. Untuk mempercepat membuat plot
  pada fungsi yang rumit, matikan <adaptive ini, dan secara pilihan
  mengurangi banyaknya interval n. Selebihnya, plot2d() akan secara
  bawaan menggunakan pemetaan. Dengan kata lain, ini akan menghitung
  elemen plot untuk elemen. Jika ekspresi kamu atau fungsi kamu
  dapat dipegang oleh vektor x, kamu dapat mengubah <maps agar
  tidak aktif untuk evaluasi yang lebih cepat.
\end{eulerttcomment}
\begin{eulercomment}

\end{eulercomment}
\begin{eulerttcomment}
  Catatan bahwa plot adaptif selalu menghitung elemen untuk elemen.
  Jika fungsi atau ekspresi untuk keduanya xv dan untuk yv
  dispesifikan, plot2d() akan dihitung sebuah kurva dengan nilai xv
  sebagai koordinat-x dan nilai yv sebagai koordinat-y. Dalam kasus
  ini, selang harusnya didefinisikan untuk parameter menggunakan
  xmin, xmax. Ekspresi yang dimuat dalam string harus selalu
  berupa ekspresi dengan parameter variabel x.
\end{eulerttcomment}
\eulerheading{Beberapa Materi untuk Fungsi dan Plot}
\begin{eulercomment}
Dalam matematika plot tidak dapat dipisahkan dengan sebuah fungsi.
Karena plot itu sendiri merupakan gambaran/sketsa yang dihasilkan oleh
fungsi tersebut. Dalam kasus ini, EMT menyediakan fungsi pemrograman
yaitu plot2d yang mana dapat membuat plot untuk beberapa kondisi dalam
suatu bidang dua dimensi. Ini berarti fungsi matematika yang akan kita
pakai adalah fungsi untuk satu variabel, sederhananya f(x) = y.

Dalam bahasan selanjutnya, saya akan membahas beberapa fungsi yang ada
di matematika, seperti:\\
- fungsi linear\\
- fungsi kuadrat\\
- fungsi polinomial\\
- fungsi eksponensial\\
- fungsi logaritmik\\
- fungsi trigonometri

Kemudian akan membahas fungsi khusus seperti fungsi tangga, fungsi
atap, dan fungsi absolut.

Selanjutnya, akan dibahas beberapa plot dalam statistika, seperti plot
batang, plot pencar, dan plot distribusi normal

Terakhir disuguhkan beberapa hal yang bisa dilakukan menggunakan EMT
ini dengan sedikit bermain.
\end{eulercomment}
\eulersubheading{Fungsi Linear}
\begin{eulercomment}
Dalam matematika, sebuah fungsi dapat dikatakan linear apabila
membentuk sebuah garis lurus. Dimana fungsi tersebut merupakan fungsi
polinomial yang variabel-variabelnya memiliki pangkat tertinggi adalah
1.

Bentuk umum fungsi linear adalah

\end{eulercomment}
\begin{eulerformula}
\[
f(x) = y = mx + c
\]
\end{eulerformula}
\begin{eulercomment}
dengan:\\
- y adalah variabel tak bebas\\
- m adalah gradien\\
- x adalah variabel bebas\\
- c sebarang bilangan

Contoh:

1. Buatlah plot dengan garis lurus yang melalui titik (4, 2) dan (2,
6)

Penyelesaian:

Dapat menggunakan persamaan umum dari pencarian gradien, yaitu

\end{eulercomment}
\begin{eulerformula}
\[
\frac{(y - y_1)}{(y_2 - y_1)} = \frac{(x - x_1)}{(x_2 - x_1)}
\]
\end{eulerformula}
\begin{eulercomment}
dengan mengubah menjadi persamaan y = ...
\end{eulercomment}
\begin{eulerprompt}
>persamaanY &= solve((y - y1) / (y2 - y1) = (x - x1) / (x2 - x1), y)
\end{eulerprompt}
\begin{euleroutput}
  
                         - (x1 - x) y2 - (x - x2) y1
                    [y = ---------------------------]
                                   x2 - x1
  
\end{euleroutput}
\begin{eulercomment}
Kemudian masukkan titik titiknya, misalkan\\
\end{eulercomment}
\begin{eulerformula}
\[
y_1 = 2, y_2 = 6, x_1 = 4, x_2 = 2
\]
\end{eulerformula}
\begin{eulerprompt}
>&persamaanY with [y1 = 2, y2 = 6, x1 = 4, x2 = 2]
\end{eulerprompt}
\begin{euleroutput}
  
                            2 (x - 2) + 6 (4 - x)
                       [y = ---------------------]
                                      2
  
\end{euleroutput}
\begin{eulercomment}
Kemudian ubah menjadi ekspresi program
\end{eulercomment}
\begin{eulerprompt}
>ekspresiPersamaanY = "(2 * (x - 2) + 6 * (4 - x)) / 2"
\end{eulerprompt}
\begin{euleroutput}
  (2 * (x - 2) + 6 * (4 - x)) / 2
\end{euleroutput}
\begin{eulercomment}
Terakhir adalah buat plotnya dengan memastikan titik-titik yang
dilaluinya.
\end{eulercomment}
\begin{eulerprompt}
>y1 = 2;...
>y2 = 6;
>x1 = solve(ekspresiPersamaanY, 0, 1, y1); ...
>x2 = solve(ekspresiPersamaanY, 0, 1, y2); ...
>plot2d(ekspresiPersamaanY, 1, 5); ...
>plot2d(x1, y1, >points, >add, color=blue, style="ow");...
>label(latex("A(4, 2)"), x1, y1, pos = "cl", offset=10); ...
>plot2d(x2, y2, >points, >add, color=blue, style="ow");...
>label(latex("B(2, 6)"), x2, y2, pos = "cl", offset=10):
\end{eulerprompt}
\eulerimg{17}{images/Hamemayu Hayuningrat_23030630053_EMTPlot2D-154.png}
\begin{eulercomment}
Jadi, persamaan garis lurus yang melalui titik (4, 2) dan (2, 6)
adalah

\end{eulercomment}
\begin{eulerformula}
\[
y = \frac{2(x - 2) + 6(4 - x)}{2}
\]
\end{eulerformula}
\begin{eulercomment}
dengan plot seperti diatas

\end{eulercomment}
\begin{eulercomment}
2. Buatlah plot persamaan grafik fungsi linear melalui titik (2, 4)
dengan gradien 2.

Pembahasan:\\
Menggunakan rumus

\end{eulercomment}
\begin{eulerformula}
\[
y - y_1 = m(x - x_1)
\]
\end{eulerformula}
\begin{eulercomment}
dengan:\\
\end{eulercomment}
\begin{eulerformula}
\[
y_1 = 4, m = 2, x_1 = 2
\]
\end{eulerformula}
\begin{eulerprompt}
>persamaan &= "y - y1 = m*(x - x1)"
\end{eulerprompt}
\begin{euleroutput}
  
                           y - y1 = m (x - x1)
  
\end{euleroutput}
\begin{eulerprompt}
>persamaanY &= solve(persamaan, y)
\end{eulerprompt}
\begin{euleroutput}
  
                          [y = y1 - m x1 + m x]
  
\end{euleroutput}
\begin{eulerprompt}
>&persamaanY with [m = 2, y1 = 4, x1 = 2]
\end{eulerprompt}
\begin{euleroutput}
  
                                [y = 2 x]
  
\end{euleroutput}
\begin{eulerprompt}
>ekspresiPersamaanY = "y = 2*x";
>y1 = 4;
>x1 = solve(ekspresiPersamaanY, 0, 1, y1);...
\end{eulerprompt}
\begin{euleroutput}
  Error in Evaluate, superfluous characters found.
  Error in expression: y = 2*x
  secant:
      y0=f$(x0,args())-y; y1=f$(x1,args())-y;
  Try "trace errors" to inspect local variables after errors.
  solve:
      if eps then return secant(f$,a,b,y;args(),eps=eps);
\end{euleroutput}
\begin{eulerprompt}
>plot2d(ekspresiPersamaanY, 0, 6); ...
>plot2d(x1, y1, >points, >add, color=blue, style="ow");...
>label(latex("A(2, 4)"), x1, y1, pos="cl", offset=10):
\end{eulerprompt}
\begin{eulercomment}
Jadi persamaan garis lurus bergradien 2 yang melalui (2, 4) adalah

\end{eulercomment}
\begin{eulerformula}
\[
y = 2x
\]
\end{eulerformula}
\begin{eulercomment}
dengan plot seperti diatas.
\end{eulercomment}
\eulersubheading{Fungsi Kuadrat}
\begin{eulercomment}
Biasanya fungsi kuadrat ini berupa parabola apabila di gambarkan.
Definisi umum dari fungsi ini merupakan fungsi polinomial dengan
pangkat  tertinggi variabelnya adalah 2. Bentuk umumnya dinotasikan
sebagai berikut

\end{eulercomment}
\begin{eulerformula}
\[
f(x) = ax^{2} + bx + c, a\neq0
\]
\end{eulerformula}
\begin{eulercomment}
\end{eulercomment}
\begin{eulercomment}
Contoh soal

Buatlah grafik fungsi\\
\end{eulercomment}
\begin{eulerformula}
\[
f(x) = -x^{2} + 5x + 3
\]
\end{eulerformula}
\begin{eulercomment}
dalam sistem koordinat. Temukan titik puncak, titik potong terhadap
sumbu-x dan sumbu-y.

Penyelesaian:

- Grafik fungsi
\end{eulercomment}
\begin{eulerprompt}
>function f(x) &= -(x)^2 + 5*x + 3;
>plot2d(f, -3, 7, -5, 10):
\end{eulerprompt}
\eulerimg{17}{images/Hamemayu Hayuningrat_23030630053_EMTPlot2D-155.png}
\begin{eulercomment}
- Titik puncak

Dapat dihitung dengan 

\end{eulercomment}
\begin{eulerformula}
\[
(\frac{-b}{2a}, f(\frac{-b}{2a}))
\]
\end{eulerformula}
\begin{eulercomment}
dengan:\\
\end{eulercomment}
\begin{eulerformula}
\[
a = -1, b = 5, c = 3
\]
\end{eulerformula}
\begin{eulerprompt}
>titikPuncak &= "-b/2*a"
\end{eulerprompt}
\begin{euleroutput}
  
                                    a b
                                  - ---
                                     2
  
\end{euleroutput}
\begin{eulerprompt}
>titikPuncakX := &titikPuncak with [b = 5, a = -1]
\end{eulerprompt}
\begin{euleroutput}
  
                                    5
                                    -
                                    2
  
\end{euleroutput}
\begin{eulerprompt}
>titikPuncakY = f(5/2)
\end{eulerprompt}
\begin{euleroutput}
  9.25
\end{euleroutput}
\begin{eulercomment}
Jadi titik puncak dari fungsi kuadrat tersebut adalah\\
\end{eulercomment}
\begin{eulerformula}
\[
(2.5, 9.25)
\]
\end{eulerformula}
\begin{eulercomment}
- Titik Potong sumbu-x
\end{eulercomment}
\begin{eulerprompt}
>ekspresiPersamaan &= "y = -x^2 + 5*x + 3";
>titikPotongY := &solve(ekspresiPersamaan with y = 0, x)
\end{eulerprompt}
\begin{euleroutput}
  
                        5 - sqrt(37)      sqrt(37) + 5
                   [x = ------------, x = ------------]
                             2                 2
  
\end{euleroutput}
\begin{eulercomment}
Jadi diperoleh titik potong persamaan kuadrat tersebut terhadap
sumbu-x adalah

\end{eulercomment}
\begin{eulerformula}
\[
(\frac{5 - \sqrt{37}}{2}, 0), (\frac{\sqrt{37} + 5}{2}, 0)
\]
\end{eulerformula}
\begin{eulercomment}
\end{eulercomment}
\begin{eulercomment}
- Titik potong sumbu-y
\end{eulercomment}
\begin{eulerprompt}
>$&solve(ekspresiPersamaan with x = 0, y)
\end{eulerprompt}
\begin{eulerformula}
\[
\left[ y=3 \right] 
\]
\end{eulerformula}
\begin{eulercomment}
Jadi diperoleh titik potong persamaan kuadrat tersebut terhadap
sumbu-y adalah

\end{eulercomment}
\begin{eulerformula}
\[
(0, 3)
\]
\end{eulerformula}
\begin{eulercomment}
\end{eulercomment}
\eulersubheading{Fungsi Polinomial}
\begin{eulercomment}
fungsi polinomial adalah sebuah fungsi yang dapat ditulis dengan
bentuk umum:

\end{eulercomment}
\begin{eulerformula}
\[
f(x) = a_{n}x^{n} + a_{n-1}x^{n-1} + \dots + a_{1}x + a_0
\]
\end{eulerformula}
\begin{eulercomment}
untuk n adalah bilangan bulat tidak negatif, dengan koofisiennya 

\end{eulercomment}
\begin{eulerformula}
\[
a_0, a_1, ...
\]
\end{eulerformula}
\begin{eulercomment}
adalah bilangan rill, dan 

\end{eulercomment}
\begin{eulerformula}
\[
a_n \neq 0
\]
\end{eulerformula}
\begin{eulercomment}
\end{eulercomment}
\begin{eulercomment}
Soal:

Buatlah plot dari fungsi

\end{eulercomment}
\begin{eulerformula}
\[
f(x) = x^{3} - 4x^{2} + 6x - 1
\]
\end{eulerformula}
\begin{eulercomment}
penyelesaian:
\end{eulercomment}
\begin{eulerprompt}
>function f(x) := x^3 - 4*x^2 + 6*x - 1;...
>plot2d(&f, -2, 5, -5, 10):
\end{eulerprompt}
\eulerimg{17}{images/Hamemayu Hayuningrat_23030630053_EMTPlot2D-157.png}
\end{eulernotebook}
\end{document}
