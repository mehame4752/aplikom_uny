\documentclass[a4paper,10pt]{article}
\usepackage{eumat}

\begin{document}
\begin{eulernotebook}
\eulerheading{Menggambar Plot 3D dengan EMT}
\begin{eulercomment}
Ini adalah pengenalan untuk plot 3D pada Euler. Kita membutuhkan plot
3D untuk memvisualisasikan sebuah fungsi dari dua variabel.

Euler menggunakan fungsi ini untuk algoritma pengurutan untuk
menyembunyikan bagian dalam latar belakang. Umumnya, Euler menggunakan
proyeksi sentral. Secara bawaan dari kuadran x-y positif melewati
titik asal x=y=z=0, tetapi sudut=0° terlihat arah dari sumbu-y.
Pandangan sudut dan tinggi dapat berubah.

Euler dapat memplotkan

- permukaan dengan bayangan dan garis level atau rentang level\\
- titik titik yang tersebar\\
- kurva parametrik\\
- permukaan implisit

Sebuah plot 3D dari sebuah fungsi menggunakan plot3d. Cara paling
mudah untuk membuatnya adalah dengan ekspresi dalam x dan y. Parameter
r dapat diatur dalam rentang dari sekeliling plot (0, 0).
\end{eulercomment}
\begin{eulerprompt}
>aspect(1.5); plot3d("x^2+sin(y)",-5,5,0,6*pi):
\end{eulerprompt}
\eulerimg{17}{images/Hamemayu Hayuningrat_23030630053_EMTPlot3D-001.png}
\begin{eulerprompt}
>plot3d("x^2+x*sin(y)",-5,5,0,6*pi):
\end{eulerprompt}
\eulerimg{17}{images/Hamemayu Hayuningrat_23030630053_EMTPlot3D-002.png}
\begin{eulercomment}
Silakan lakukan modifikasi agar gambar "talang bergelombang" tersebut tidak lurus melainkan melengkung/melingkar, baik
melingkar secara mendatar maupun melingkar turun/naik (seperti papan peluncur pada kolam renang. Temukan rumusnya.
\end{eulercomment}
\eulerheading{Fungsi fungsi dari Dua Variabel}
\begin{eulercomment}
Untuk membuat grafik dari sebuah fungsi, gunakan

- sebuah ekspresi sederhana dalam x dan y,\\
- nama fungsi dari dua variabel\\
- atau matriks data

Secara bawaan adalah kabel yang terisi dengan warna-warna yang berbeda
dalam kedua sisi. Catatan bahwa nomor bawaan dari interval grid adalah
10, tetapi plot menggunakan angka bawaan dari persegi 40x40 untuk
membuat permukaan. Ini dapat diubah.

- n=40, n=[40,40]: banyaknya dari garis grid dalam setiap arah\\
- grid=10, grid=[10,10]: banyaknya garis grid dalam setiap arah

Kami menggunakan bawaan n = 40 dan grid = 40.
\end{eulercomment}
\begin{eulerprompt}
>plot3d("x^2+y^2"):
\end{eulerprompt}
\eulerimg{17}{images/Hamemayu Hayuningrat_23030630053_EMTPlot3D-003.png}
\begin{eulercomment}
Interaksi pengguna dapat memungkinkan dengan mengatur parameter \textgreater{}user.
Pengguna dapat menekan tombol berikut.

- left,right,up,down: mengubah pandangan sudut\\
- +,-: memperbesar atau mengecilkan\\
- a: mengeluarkan sebuah anaglyph (lihat bawah)\\
- l: mengalihkan asal cahaya (lihat bawah)\\
- space: atur ulang ke bawaan\\
- return: selesaikan interaksi
\end{eulercomment}
\begin{eulerprompt}
>plot3d("exp(-x^2+y^2)",>user, ...
>  title="Turn with the vector keys (press return to finish)"):
\end{eulerprompt}
\eulerimg{17}{images/Hamemayu Hayuningrat_23030630053_EMTPlot3D-004.png}
\begin{eulercomment}
Plot selang untuk fungsi dapat dispesifikan dengan

- a,b: selang-x\\
- c,d: selang-y\\
- r: sebuah persegi sekitar yang simetri (0,0)\\
- n: banyaknya subinterval untuk plot

Ini beberapa parameter untuk menskalakan fungsi atau mengubah tampak
dari grafik.

fscale: menskalakan fungsi nilai (bawaan \textless{}fscale)\\
scale: angka atau vektor 1x2 untuk menskalakan ke arah x atau y\\
frame: tipe frame (bawaan 1)
\end{eulercomment}
\begin{eulerprompt}
>plot3d("exp(-(x^2+y^2)/5)",r=10,n=80,fscale=4,scale=1.2,frame=3,>user):
\end{eulerprompt}
\eulerimg{17}{images/Hamemayu Hayuningrat_23030630053_EMTPlot3D-005.png}
\begin{eulercomment}
Pandangan dapat diubah dengan banyak cara.

- distance: arah pandangan ke plot.\\
- zoom: nilai pembesaran.\\
- angle: arah ke sumbu-y negatif dalam radian.\\
- height: the height of the view in radians.

Nilai bawaan dapat diinspeksi atau diubah dengan fungsi view(). Ini
akan mengembalikan parameter dalam urutan diatas.
\end{eulercomment}
\begin{eulerprompt}
>view
\end{eulerprompt}
\begin{euleroutput}
  [5,  2.6,  2,  0.4]
\end{euleroutput}
\begin{eulercomment}
Jarak terdekat membutuhkan pembesaran yang lebih sedikit. Efeknya kira
kira seperti sebuah lensa sudut yang lebar.

Seperti contoh berikut, angle=0 dan height=0 terlihat dari sumbu-y
negatif. Label sumbu untuk y disembunyikan dalam kasus ini.
\end{eulercomment}
\begin{eulerprompt}
>plot3d("x^2+y",distance=3,zoom=1,angle=pi/2,height=0):
\end{eulerprompt}
\eulerimg{17}{images/Hamemayu Hayuningrat_23030630053_EMTPlot3D-006.png}
\begin{eulercomment}
Plot selalu terlihat ke titik tengah dari kubus plot. Kamu dapat
menggerakkan tengah dengan parameter center.
\end{eulercomment}
\begin{eulerprompt}
>plot3d("x^4+y^2",a=0,b=1,c=-1,d=1,angle=-20°,height=20°, ...
>  center=[0.4,0,0],zoom=5):
\end{eulerprompt}
\eulerimg{17}{images/Hamemayu Hayuningrat_23030630053_EMTPlot3D-007.png}
\begin{eulercomment}
Plot dapat diskalakan untuk menyesuaikan ke bagian kubus untuk
dilihat. Jadi ini tidak membutuhkan perubahan jarak atau memperbesar
tergantung ukuran dari plot. Label ini merujuk pada ukuran sebenarnya.

Jika kamu mematikan dengan scale=false, kamu harus berhati-hati, yang
mana plot masih menyesuaikan ke jendela plot, dengan mengubah
pandangan arah atau perbesaran dan menggerakkan titik tengah.
\end{eulercomment}
\begin{eulerprompt}
>plot3d("5*exp(-x^2-y^2)",r=2,<fscale,<scale,distance=13,height=50°, ...
>  center=[0,0,-2],frame=3):
\end{eulerprompt}
\eulerimg{17}{images/Hamemayu Hayuningrat_23030630053_EMTPlot3D-008.png}
\begin{eulercomment}
Plot kutub juga tersedia. Parameter polar=true menggambar sebuah plot
polar. Fungsi harus tetap sebagai fungsi dari x dan y. Parameter
"fscale" menskalakan fungsi dengan skala tersendiri. Sebaliknya fungsi
diskalakan mengikuti sebuah kubus.
\end{eulercomment}
\begin{eulerprompt}
>plot3d("1/(x^2+y^2+1)",r=5,>polar, ...
>fscale=2,>hue,n=100,zoom=4,>contour,color=blue):
\end{eulerprompt}
\eulerimg{17}{images/Hamemayu Hayuningrat_23030630053_EMTPlot3D-009.png}
\begin{eulerprompt}
>function f(r) := exp(-r/2)*cos(r); ...
>plot3d("f(x^2+y^2)",>polar,scale=[1,1,0.4],r=pi,frame=3,zoom=4):
\end{eulerprompt}
\eulerimg{17}{images/Hamemayu Hayuningrat_23030630053_EMTPlot3D-010.png}
\begin{eulercomment}
Parameter rotate memutar sebuah fungsi dalam x mengelilingi sumbu-x.

- rotate=1: Menggunakan sumbu-x\\
- rotate=2: Menggunakan sumbu-z
\end{eulercomment}
\begin{eulerprompt}
>plot3d("x^2+1",a=-1,b=1,rotate=true,grid=5):
\end{eulerprompt}
\eulerimg{17}{images/Hamemayu Hayuningrat_23030630053_EMTPlot3D-011.png}
\begin{eulerprompt}
>plot3d("x^2+1",a=-1,b=1,rotate=2,grid=5):
\end{eulerprompt}
\eulerimg{17}{images/Hamemayu Hayuningrat_23030630053_EMTPlot3D-012.png}
\begin{eulerprompt}
>plot3d("sqrt(25-x^2)",a=0,b=5,rotate=1):
\end{eulerprompt}
\eulerimg{17}{images/Hamemayu Hayuningrat_23030630053_EMTPlot3D-013.png}
\begin{eulerprompt}
>plot3d("x*sin(x)",a=0,b=6pi,rotate=2):
\end{eulerprompt}
\eulerimg{17}{images/Hamemayu Hayuningrat_23030630053_EMTPlot3D-014.png}
\begin{eulercomment}
Ini adalah plot dengan tiga fungsi.
\end{eulercomment}
\begin{eulerprompt}
>plot3d("x","x^2+y^2","y",r=2,zoom=3.5,frame=3):
\end{eulerprompt}
\eulerimg{17}{images/Hamemayu Hayuningrat_23030630053_EMTPlot3D-015.png}
\eulerheading{Plot Kontur}
\begin{eulercomment}
Untuk plot, Euler menambahkan garis bata. Ini memungkinkan untuk
menggunakan level garis dan sebuah-warna hue atau sebuah spektral
berwarna hue. Euler dapat menggambar tinggi dari fungsi dalam sebuah
plot dengan bayangan. Dalam semua plot 3D Euler dapat menghasilkan
anaglyph merah/cyan.

- \textgreater{}hue: Menggubah bayangan cahaya daripada garis.\\
- \textgreater{}contour: Membuat plot garis kontur otomatis dalam sebuah plot.\\
- level=... (or levels): Sebuah vektor dari nilai untuk garis kontur.

Bawaannya level="auto", yang mana menghitung beberapa garis level
secara otomatis. Seperti yang kamu lihat dalam plot, level dalam
rentang level.

Gaya bawan dapat diubah. Untuk plot kontur, kami menggunakan sebuah
batasan yang halus untuk titik titik 100x100, menskalakan fungsi dan
plot, dan menggunakan sudut yang berbeda dari pandangan.
\end{eulercomment}
\begin{eulerprompt}
>plot3d("exp(-x^2-y^2)",r=2,n=100,level="thin", ...
> >contour,>spectral,fscale=1,scale=1.1,angle=45°,height=20°):
\end{eulerprompt}
\eulerimg{17}{images/Hamemayu Hayuningrat_23030630053_EMTPlot3D-016.png}
\begin{eulerprompt}
>plot3d("exp(x*y)",angle=100°,>contour,color=green):
\end{eulerprompt}
\eulerimg{17}{images/Hamemayu Hayuningrat_23030630053_EMTPlot3D-017.png}
\begin{eulercomment}
Bayangan bawaan menggunakan warna abu-abu. Tetapi selang spektral dari
warna juga tersedia.

- \textgreater{}spectral: Digunakan skema spektral bawaan\\
- color=...: Menggunakan warna spesial atau skema spektral

Untuk plot berikut, kami menggunakan skema spektral bawaan dan
meningkatkan angka dari titik untuk mendapatkan tampilan yang halus.
\end{eulercomment}
\begin{eulerprompt}
>plot3d("x^2+y^2",>spectral,>contour,n=100):
\end{eulerprompt}
\eulerimg{17}{images/Hamemayu Hayuningrat_23030630053_EMTPlot3D-018.png}
\begin{eulercomment}
Selain garis level otomatis, kami dapat juga mengatur nilai dari garis
level. Ini akan menghasilkan garis level yang tipis daripada rentang
level.
\end{eulercomment}
\begin{eulerprompt}
>plot3d("x^2-y^2",0,5,0,5,level=-1:0.1:1,color=redgreen):
\end{eulerprompt}
\eulerimg{17}{images/Hamemayu Hayuningrat_23030630053_EMTPlot3D-019.png}
\begin{eulercomment}
Plot berikut, kami menggunakan dua level yang sangat luas dari -0.1 ke
1 dan dari 0.9 ke 1. Ini dimasukkan sebagai sebuah matrik dengan level
batas seperti kolom.

Selebihnya, kami melapisi sebuah batasan dengan 10 interval pada
setiap arah.
\end{eulercomment}
\begin{eulerprompt}
>plot3d("x^2+y^3",level=[-0.1,0.9;0,1], ...
>  >spectral,angle=30°,grid=10,contourcolor=gray):
\end{eulerprompt}
\eulerimg{17}{images/Hamemayu Hayuningrat_23030630053_EMTPlot3D-020.png}
\begin{eulercomment}
Contoh berikut, kami mengatur plot, dimana

\end{eulercomment}
\begin{eulerformula}
\[
f(x,y) = x^y-y^x = 0
\]
\end{eulerformula}
\begin{eulercomment}
Kami menggunakan garis tipis tunggal untuk garis level.
\end{eulercomment}
\begin{eulerprompt}
>plot3d("x^y-y^x",level=0,a=0,b=6,c=0,d=6,contourcolor=red,n=100):
\end{eulerprompt}
\eulerimg{17}{images/Hamemayu Hayuningrat_23030630053_EMTPlot3D-021.png}
\begin{eulercomment}
Ini memungkinkan untuk menampilkan sebuah bidang kontur dibawah plot.
Warna dan jarak ke plot dapat di spesifikan.
\end{eulercomment}
\begin{eulerprompt}
>plot3d("x^2+y^4",>cp,cpcolor=green,cpdelta=0.2):
\end{eulerprompt}
\eulerimg{17}{images/Hamemayu Hayuningrat_23030630053_EMTPlot3D-022.png}
\begin{eulercomment}
Ini beberapa gaya yang lainnya. Kami selalu mematikan frame dan
menggunakan skema warna yang bervariasi untuk plot dan garis batas.
\end{eulercomment}
\begin{eulerprompt}
>figure(2,2); ...
>expr="y^3-x^2"; ...
>figure(1);  ...
>  plot3d(expr,<frame,>cp,cpcolor=spectral); ...
>figure(2);  ...
>  plot3d(expr,<frame,>spectral,grid=10,cp=2); ...
>figure(3);  ...
>  plot3d(expr,<frame,>contour,color=gray,nc=5,cp=3,cpcolor=greenred); ...
>figure(4);  ...
>  plot3d(expr,<frame,>hue,grid=10,>transparent,>cp,cpcolor=gray); ...
>figure(0):
\end{eulerprompt}
\eulerimg{17}{images/Hamemayu Hayuningrat_23030630053_EMTPlot3D-023.png}
\begin{eulercomment}
Ini beberapa skema spektral yang lainnya, dihitung dari 1 ke 9. Tetapi
kamu juga menggunakan color=value, dimana value

- spectral: untuk sebuah selang dari biru ke merah\\
- white: untuk selang pewarna\\
- yellowblue,purplegreen,blueyellow,greenred\\
- blueyellow, greenpurple,yellowblue,redgreen
\end{eulercomment}
\begin{eulerprompt}
>figure(3,3); ...
>for i=1:9;  ...
>  figure(i); plot3d("x^2+y^2",spectral=i,>contour,>cp,<frame,zoom=4);  ...
>end; ...
>figure(0):
\end{eulerprompt}
\eulerimg{17}{images/Hamemayu Hayuningrat_23030630053_EMTPlot3D-024.png}
\begin{eulercomment}
Sumber cahaya dapat diubah dengan 1 dan kunci kursor selama interaksi
pengguna. Ini juga dapat diubah dengan parameter.

- light: arah cahaya\\
- amb: cahaya sekitar antara 0 dan 1

Catatan bahwa program tidak membuat perbedaan antara sisi dari plot.
Tidak ada bayangan. Untuk hal ini akan membutuhkan Povray.
\end{eulercomment}
\begin{eulerprompt}
>plot3d("-x^2-y^2", ...
>  hue=true,light=[0,1,1],amb=0,user=true, ...
>  title="Press l and cursor keys (return to exit)"):
\end{eulerprompt}
\eulerimg{17}{images/Hamemayu Hayuningrat_23030630053_EMTPlot3D-025.png}
\begin{eulercomment}
Parameter color mengubah warba dari permukaan. Warna dari garis level
juga dapat diubah.
\end{eulercomment}
\begin{eulerprompt}
>plot3d("-x^2-y^2",color=rgb(0.2,0.2,0),hue=true,frame=false, ...
>  zoom=3,contourcolor=red,level=-2:0.1:1,dl=0.01):
\end{eulerprompt}
\eulerimg{17}{images/Hamemayu Hayuningrat_23030630053_EMTPlot3D-026.png}
\begin{eulercomment}
Color 0 memberikan spesial pelangi efek.
\end{eulercomment}
\begin{eulerprompt}
>plot3d("x^2/(x^2+y^2+1)",color=0,hue=true,grid=10):
\end{eulerprompt}
\eulerimg{17}{images/Hamemayu Hayuningrat_23030630053_EMTPlot3D-027.png}
\begin{eulercomment}
Permukaan juga dapat transparan.
\end{eulercomment}
\begin{eulerprompt}
>plot3d("x^2+y^2",>transparent,grid=10,wirecolor=red):
\end{eulerprompt}
\eulerimg{17}{images/Hamemayu Hayuningrat_23030630053_EMTPlot3D-028.png}
\eulerheading{Plot Implisit}
\begin{eulercomment}
Ini juga plot implisit dalam dimensi tiga. Euler menghasilkan potongan
melalui objek. Fitur dari plot3d mencakup plot implisit. Plot ini
menunjukkan penggaturan 0 dari fungsi dalam variabel 3.

Solusi dari

\end{eulercomment}
\begin{eulerformula}
\[
f(x,y,z) = 0
\]
\end{eulerformula}
\begin{eulercomment}
dapat divisualisasikan dalam potongan paralel ke bidang x-y-, x-z- dan
y-z.

- implicit=1: memotong secara paralel ke bidang y-z\\
- implicit=2: memotong secara paralel ke bidang x-z\\
- implicit=4: memotong secara paralel ke bidang x-y

Menambahkan nilai-nilai ini, jika kamu suka. Dalam contoh ini kami
membuat plot

\end{eulercomment}
\begin{eulerformula}
\[
M = \{ (x,y,z) : x^2+y^3+zy=1 \}
\]
\end{eulerformula}
\begin{eulerprompt}
>plot3d("x^2+y^3+z*y-1",r=5,implicit=3):
\end{eulerprompt}
\eulerimg{17}{images/Hamemayu Hayuningrat_23030630053_EMTPlot3D-029.png}
\begin{eulerprompt}
>c=1; d=1;
>plot3d("((x^2+y^2-c^2)^2+(z^2-1)^2)*((y^2+z^2-c^2)^2+(x^2-1)^2)*((z^2+x^2-c^2)^2+(y^2-1)^2)-d",r=2,<frame,>implicit,>user): 
\end{eulerprompt}
\eulerimg{17}{images/Hamemayu Hayuningrat_23030630053_EMTPlot3D-030.png}
\begin{eulerprompt}
>plot3d("x^2+y^2+4*x*z+z^3",>implicit,r=2,zoom=2.5):
\end{eulerprompt}
\eulerimg{17}{images/Hamemayu Hayuningrat_23030630053_EMTPlot3D-031.png}
\eulerheading{Membuat plot data 3D}
\begin{eulercomment}
Seperti plot2d, plot3d menerima data. Untuk objek 3D, kamu harus
menyediakan sebuah matriks dari nilai x, y dan z, atau fungsi tiga
atau ekspresi fx(x, y), fy(x, y), fz(x, y).

\end{eulercomment}
\begin{eulerformula}
\[
\gamma(t,s) = (x(t,s),y(t,s),z(t,s))
\]
\end{eulerformula}
\begin{eulercomment}
Karena x, y, z merupakan matriks, kami mengasumsikan bahwa (t, s)
melewati sebuah batasan persegi. Sebagai hasilnya, kamu dapat membuat
gambar persegi dalam ruang.

Kamu dapat menggunakan bahasa matriks Euler untuk menghasilkan
koordinat secara efektif.

Seperti contoh berikut, kami menggunakan vektor dari nilai t dan
sebuah kolom vektor dari nilai s untuk memparameterisasi wilayah dari
bola. dalam penggambaran kami menandai wilayah-wilayah, dalam kasus
kami wilayah polar.
\end{eulercomment}
\begin{eulerprompt}
>t=linspace(0,2pi,180); s=linspace(-pi/2,pi/2,90)'; ...
>x=cos(s)*cos(t); y=cos(s)*sin(t); z=sin(s); ...
>plot3d(x,y,z,>hue, ...
>color=blue,<frame,grid=[10,20], ...
>values=s,contourcolor=red,level=[90°-24°;90°-22°], ...
>scale=1.4,height=50°):
\end{eulerprompt}
\eulerimg{17}{images/Hamemayu Hayuningrat_23030630053_EMTPlot3D-032.png}
\begin{eulercomment}
Ini sebuah contoh, yang mana grafik dari sebuah fungsi.
\end{eulercomment}
\begin{eulerprompt}
>t=-1:0.1:1; s=(-1:0.1:1)'; plot3d(t,s,t*s,grid=10):
\end{eulerprompt}
\eulerimg{17}{images/Hamemayu Hayuningrat_23030630053_EMTPlot3D-033.png}
\begin{eulercomment}
Namun, kamu juga membuat semua pengurutan dari wilayah. Ini wilayah
yang sama seperti fungsi

\end{eulercomment}
\begin{eulerformula}
\[
x = y \, z
\]
\end{eulerformula}
\begin{eulerprompt}
>plot3d(t*s,t,s,angle=180°,grid=10):
\end{eulerprompt}
\eulerimg{17}{images/Hamemayu Hayuningrat_23030630053_EMTPlot3D-034.png}
\begin{eulercomment}
Dengan usaha yang lebih, kami menghasilkan lebih banyak permukaan.

Seperti contoh berikut kami membuat pendangan bayangan untuk bola
terdistorsi. Koordinat biasa untuk bola adalah

\end{eulercomment}
\begin{eulerformula}
\[
\gamma(t,s) = (\cos(t)\cos(s),\sin(t)\sin(s),\cos(s))
\]
\end{eulerformula}
\begin{eulercomment}
dengan

\end{eulercomment}
\begin{eulerformula}
\[
0 \le t \le 2\pi, \quad \frac{-\pi}{2} \le s \le \frac{\pi}{2}.
\]
\end{eulerformula}
\begin{eulercomment}
Kami mendistorsi ini dengan sebuah faktor

\end{eulercomment}
\begin{eulerformula}
\[
d(t,s) = \frac{\cos(4t)+\cos(8s)}{4}.
\]
\end{eulerformula}
\begin{eulerprompt}
>t=linspace(0,2pi,320); s=linspace(-pi/2,pi/2,160)'; ...
>d=1+0.2*(cos(4*t)+cos(8*s)); ...
>plot3d(cos(t)*cos(s)*d,sin(t)*cos(s)*d,sin(s)*d,hue=1, ...
>  light=[1,0,1],frame=0,zoom=5):
\end{eulerprompt}
\eulerimg{17}{images/Hamemayu Hayuningrat_23030630053_EMTPlot3D-035.png}
\begin{eulercomment}
Tentu saja, titik awan juga mungkin. Untuk membuat data titik dalam
ruang, kami membutuhkan tiga vektor untuk koordiat dari titik-titik.

Gaya seperti dalam plot3d dengan points=true;
\end{eulercomment}
\begin{eulerprompt}
>n=500;  ...
>  plot3d(normal(1,n),normal(1,n),normal(1,n),points=true,style="."):
\end{eulerprompt}
\eulerimg{17}{images/Hamemayu Hayuningrat_23030630053_EMTPlot3D-036.png}
\begin{eulercomment}
Ini juga mungkin untuk membuat plot sebuah kurva 3D. Dalam kasus ini,
mudah untuk menghitung titik-titik dari kurva sebelumnya. Untuk kurva
dalam bidang kamu menggunakan urutan dari koordinat dan parameter
wire=true.
\end{eulercomment}
\begin{eulerprompt}
>t=linspace(0,8pi,500); ...
>plot3d(sin(t),cos(t),t/10,>wire,zoom=3):
\end{eulerprompt}
\eulerimg{17}{images/Hamemayu Hayuningrat_23030630053_EMTPlot3D-037.png}
\begin{eulerprompt}
>t=linspace(0,4pi,1000); plot3d(cos(t),sin(t),t/2pi,>wire, ...
>linewidth=3,wirecolor=blue):
\end{eulerprompt}
\eulerimg{17}{images/Hamemayu Hayuningrat_23030630053_EMTPlot3D-038.png}
\begin{eulerprompt}
>X=cumsum(normal(3,100)); ...
> plot3d(X[1],X[2],X[3],>anaglyph,>wire):
\end{eulerprompt}
\eulerimg{17}{images/Hamemayu Hayuningrat_23030630053_EMTPlot3D-039.png}
\begin{eulercomment}
EMT dapat juga membuat plot dalam mode anaglyph. Untuk memandang
seperti sebuah plot, kamu membutuhkan kacamata merah/cyan.
\end{eulercomment}
\begin{eulerprompt}
>plot3d("x^2+y^3",>anaglyph,>contour,angle=30°):
\end{eulerprompt}
\eulerimg{17}{images/Hamemayu Hayuningrat_23030630053_EMTPlot3D-040.png}
\begin{eulercomment}
Sering kali, sebuah skema spektral warna menggunakan plot. Menegaskan
tinggi dari fungsi.
\end{eulercomment}
\begin{eulerprompt}
>plot3d("x^2*y^3-y",>spectral,>contour,zoom=3.2):
\end{eulerprompt}
\eulerimg{17}{images/Hamemayu Hayuningrat_23030630053_EMTPlot3D-041.png}
\begin{eulercomment}
Euler dapat membuat permukaan yang terparameter juga, ketika parameter
nilai x, y, dan z dari sebuah gambar dari sebuah batasan persegi dalam
ruang.

Seperti demo berikut, kami mengatur parameter u dan v, dan membuat
koordinat ruang untuk ini.
\end{eulercomment}
\begin{eulerprompt}
>u=linspace(-1,1,10); v=linspace(0,2*pi,50)'; ...
>X=(3+u*cos(v/2))*cos(v); Y=(3+u*cos(v/2))*sin(v); Z=u*sin(v/2); ...
>plot3d(X,Y,Z,>anaglyph,<frame,>wire,scale=2.3):
\end{eulerprompt}
\eulerimg{17}{images/Hamemayu Hayuningrat_23030630053_EMTPlot3D-042.png}
\begin{eulercomment}
Ini merupakan contoh yang lebih kompleks, yang mana terlihat megah
dengan kacamata merah/cyan.
\end{eulercomment}
\begin{eulerprompt}
>u:=linspace(-pi,pi,160); v:=linspace(-pi,pi,400)';  ...
>x:=(4*(1+.25*sin(3*v))+cos(u))*cos(2*v); ...
>y:=(4*(1+.25*sin(3*v))+cos(u))*sin(2*v); ...
> z=sin(u)+2*cos(3*v); ...
>plot3d(x,y,z,frame=0,scale=1.5,hue=1,light=[1,0,-1],zoom=2.8,>anaglyph):
\end{eulerprompt}
\eulerimg{17}{images/Hamemayu Hayuningrat_23030630053_EMTPlot3D-043.png}
\eulerheading{Plot Statistikal}
\begin{eulercomment}
Plot batang juga mungkin untuk dibuat. Seperti ini, kami menyediakan

- x: vektor baris dengan elemen n+1\\
- y: vektor kolom dengan elemen n+1\\
- z: matriks nxn nilai.

z dapat diperbesar, tetapi hanya nilai nxn yang akan digunakan.

Dalam contoh, kami pertama-tama menghitung nilai. Kemudian kita
menyesuaikan x dan y, jadi tengah vektor pada nilai yang digunakan.
\end{eulercomment}
\begin{eulerprompt}
>x=-1:0.1:1; y=x'; z=x^2+y^2; ...
>xa=(x|1.1)-0.05; ya=(y_1.1)-0.05; ...
>plot3d(xa,ya,z,bar=true):
\end{eulerprompt}
\eulerimg{17}{images/Hamemayu Hayuningrat_23030630053_EMTPlot3D-044.png}
\begin{eulercomment}
Ini memungkinkan untuk memisah plot dari permukaan dalam dua bagian
atau lebih
\end{eulercomment}
\begin{eulerprompt}
>x=-1:0.1:1; y=x'; z=x+y; d=zeros(size(x)); ...
>plot3d(x,y,z,disconnect=2:2:20):
\end{eulerprompt}
\eulerimg{17}{images/Hamemayu Hayuningrat_23030630053_EMTPlot3D-045.png}
\begin{eulercomment}
Jika memuat atau menghasilkan sebuah matriks data M dari sebuah file
dan membutuhkan untuk membuat plot itu dalam 3D kamu dapat menskalakan
matriks ke [-1, 1] dengan skala (M), atau skala matriks dengan
\textgreater{}zscale. Ini dapat juga dikombinasikan dengan pemfaktoran skala
tersendiri yang aman dapat diaplikasikan sebagai tambahan.
\end{eulercomment}
\begin{eulerprompt}
>i=1:20; j=i'; ...
>plot3d(i*j^2+100*normal(20,20),>zscale,scale=[1,1,1.5],angle=-40°,zoom=1.8):
\end{eulerprompt}
\eulerimg{17}{images/Hamemayu Hayuningrat_23030630053_EMTPlot3D-046.png}
\begin{eulerprompt}
>Z=intrandom(5,100,6); v=zeros(5,6); ...
>loop 1 to 5; v[#]=getmultiplicities(1:6,Z[#]); end; ...
>columnsplot3d(v',scols=1:5,ccols=[1:5]):
\end{eulerprompt}
\eulerimg{17}{images/Hamemayu Hayuningrat_23030630053_EMTPlot3D-047.png}
\eulerheading{Permukaan Benda Putar}
\begin{eulerprompt}
>plot2d("(x^2+y^2-1)^3-x^2*y^3",r=1.3, ...
>style="#",color=red,<outline, ...
>level=[-2;0],n=100):
\end{eulerprompt}
\eulerimg{17}{images/Hamemayu Hayuningrat_23030630053_EMTPlot3D-048.png}
\begin{eulerprompt}
>ekspresi &= (x^2+y^2-1)^3-x^2*y^3; $ekspresi
\end{eulerprompt}
\begin{eulerformula}
\[
\left(y^2+x^2-1\right)^3-x^2\,y^3
\]
\end{eulerformula}
\begin{eulercomment}
Kami mengharapkan untuk merubah kurva hati secara mengelilingi
sumbu-y. Ini adalah ekspresi, yang mana mendefinisikan hati:

\end{eulercomment}
\begin{eulerformula}
\[
f(x,y)=(x^2+y^2-1)^3-x^2.y^3.
\]
\end{eulerformula}
\begin{eulercomment}
Selanjutnya kami mengatur

\end{eulercomment}
\begin{eulerformula}
\[
x=r.cos(a),\quad y=r.sin(a).
\]
\end{eulerformula}
\begin{eulerprompt}
>function fr(r,a) &= ekspresi with [x=r*cos(a),y=r*sin(a)] | trigreduce; $fr(r,a)
\end{eulerprompt}
\begin{eulerformula}
\[
\left(r^2-1\right)^3+\frac{\left(\sin \left(5\,a\right)-\sin \left(  3\,a\right)-2\,\sin a\right)\,r^5}{16}
\]
\end{eulerformula}
\begin{eulercomment}
Ini memperbolehkan untuk mendefinisikan fungsi numerik, yang mana
menyelesaikan untuk r, jika diberikan a. Dengan fungsi itu kami dapat
membuat plot dengan mengubah hati sebagai permukaan parametrik
\end{eulercomment}
\begin{eulerprompt}
>function map f(a) := bisect("fr",0,2;a); ...
>t=linspace(-pi/2,pi/2,100); r=f(t);  ...
>s=linspace(pi,2pi,100)'; ...
>plot3d(r*cos(t)*sin(s),r*cos(t)*cos(s),r*sin(t), ...
>>hue,<frame,color=red,zoom=4,amb=0,max=0.7,grid=12,height=50°):
\end{eulerprompt}
\eulerimg{17}{images/Hamemayu Hayuningrat_23030630053_EMTPlot3D-051.png}
\begin{eulercomment}
Berikut merupakan plot 3D dari penggambaran diatas diputar
mengelilingi sumbu-y. Kami mendefinisikan fungsi yang mana
mendeskripsikan objek
\end{eulercomment}
\begin{eulerprompt}
>function f(x,y,z) ...
\end{eulerprompt}
\begin{eulerudf}
  r=x^2+y^2;
  return (r+z^2-1)^3-r*z^3;
   endfunction
\end{eulerudf}
\begin{eulerprompt}
>plot3d("f(x,y,z)", ...
>xmin=0,xmax=1.2,ymin=-1.2,ymax=1.2,zmin=-1.2,zmax=1.4, ...
>implicit=1,angle=-30°,zoom=2.5,n=[10,100,60],>anaglyph):
\end{eulerprompt}
\eulerimg{17}{images/Hamemayu Hayuningrat_23030630053_EMTPlot3D-052.png}
\eulerheading{Plot 3D Spesial}
\begin{eulercomment}
Fungsi plot3d sangat bagus untuk dimiliki, tetapi tidak memenuhi semua
yang kita butuhkan. Disamping rutinitas umum yang lainnya, ini mungkin
untuk mendapatkan sebuah plot berframe dari sebarang objek yang kamu
sukai.

Meskipun Euler bukanlah program 3D, ini dapat mengkombinasikan objek
objek sederhana. Kami mencoba untuk memvisualisasikan sebuah
paraboloida dan garis singgung.
\end{eulercomment}
\begin{eulerprompt}
>function myplot ...
\end{eulerprompt}
\begin{eulerudf}
    y=-1:0.01:1; x=(-1:0.01:1)';
    plot3d(x,y,0.2*(x-0.1)/2,<scale,<frame,>hue, ..
      hues=0.5,>contour,color=orange);
    h=holding(1);
    plot3d(x,y,(x^2+y^2)/2,<scale,<frame,>contour,>hue);
    holding(h);
  endfunction
\end{eulerudf}
\begin{eulercomment}
Sekarang framedplot() menyediakan frame dan mengatur pandangan.
\end{eulercomment}
\begin{eulerprompt}
>framedplot("myplot",[-1,1,-1,1,0,1],height=0,angle=-30°, ...
>  center=[0,0,-0.7],zoom=3):
\end{eulerprompt}
\eulerimg{17}{images/Hamemayu Hayuningrat_23030630053_EMTPlot3D-053.png}
\begin{eulercomment}
Dengan jalan yang sama, kamu dapat mem-plotkan bidang kontur secara
manual. Catatan bahwa plot3d() secara bawaaan mengatur jendela ke
fullwindow(), tetapi asumsikan demikian untuk plotcontourplane().
\end{eulercomment}
\begin{eulerprompt}
>x=-1:0.02:1.1; y=x'; z=x^2-y^4;
>function myplot (x,y,z) ...
\end{eulerprompt}
\begin{eulerudf}
    zoom(2);
    wi=fullwindow();
    plotcontourplane(x,y,z,level="auto",<scale);
    plot3d(x,y,z,>hue,<scale,>add,color=white,level="thin");
    window(wi);
    reset();
  endfunction
\end{eulerudf}
\begin{eulerprompt}
>myplot(x,y,z):
\end{eulerprompt}
\eulerimg{27}{images/Hamemayu Hayuningrat_23030630053_EMTPlot3D-054.png}
\eulerheading{Animasi}
\begin{eulercomment}
Euler dapat menggunakan frame untuk menghitung animasi dalam langkah
sebelumya

Satu fungsi, yang mana membuat teknik ini digunakan adalah rotate. Ini
dapat merubah sudut dari pandang dan menggambar ulang sebuah plot 3D.
Pemanggulan fungsi addpage() untuk setiap plot baru. Akhirnya plot ini
dianimasikan.

Tolong pelajari sumber dari rotasi untuk lihat lebih detail.
\end{eulercomment}
\begin{eulerprompt}
>function testplot () := plot3d("x^2+y^3"); ...
>rotate("testplot"); testplot():
\end{eulerprompt}
\eulerimg{27}{images/Hamemayu Hayuningrat_23030630053_EMTPlot3D-055.png}
\eulerheading{Menggambar Povray}
\begin{eulercomment}
Dengan bantuan dari file Euler povray.e, Euler dapat membuat file
Povray. Hasilnya sangat bagus untuk dilihat.

Kamu butuh menginstall Povray (32bit atau 64bit) dari
http://www.povray.org/, dan ambillah sub-direktori "bin" dari Povray ke environtment path atau atur ke variabel "defaultpovray" dengan path yang lengkap untuk menunjukkan ke "pvengine.exe".

Tampilan Povray dari Euler membuat file Povray dalam direktori home
dari pengguna, dan memanggil Povray untuk menguraikan file ini. Nama
file bawaan adalah current.pov dan direktori bawaan adalah
eulerhome(), biasanya c:\textbackslash{}Users\textbackslash{}Username\textbackslash{}Euler. Povray membuat sebuah
file PNG, yang mana dimuat oleh Euler kedalam sebuah notebook. Untuk
membersihkan file ini, gunakan povclear().

Fungsi pov3d sama seperti plot3d. Ini akan menghasilkan grafik dari
fungsi f(x, y) atau sebuah permukaan dengan koordinat X, Y, Z dalam
matriks, termasuk secara opsional garis level. Fungsi ini dimulai
secara otomatis raytracer dan memuat pemandangan kedalam Euler
notebook.

Disamping pov3d(), terdapat banyak sekali fungsi, yang mana
menghasilkan objek Povray. Fungsi ini mengembalikan strings, mencakup
kode Povray untuk objek. Untuk menggunakan fungsi-fungsi ini, mulai
file Povray dengan povstart(). Lalu gunakan writeln(...) untuk menulis
objek ke dalam file pandangan. Terakhir, akhiri file dengan povend().
Secara bawaan, raytracer akan mulai dan file PNG akan dimasukkan
kedalam notebook Euler.

Fungsi objek memiliki sebuah parameter yang bernama "look", yang mana
membutuhkan string dengan kode Povray untuk tekstur dan akhiri objek.
Fungsi povlook() dapat digunakan untuk menghasilkan string ini. Ini
memiliki parameter untuk warna, transparansi, bayangan Phong, dan
lainnya

Catatan bahwa semesta Povray memiliki sistem koordinat lain. Antar
muka ini menerjemahkan semua koordinat kedalam sistem Povray. Jadi
kamu dapat tetap berpikir dalam koordinat sistem Euler dengan
menunjukkan z secara vertikal keatas, dan sumbu x, y, z dalam
penalaran sistem tangan kanan.

Kamu butuh untuk memuat file povray.
\end{eulercomment}
\begin{eulerprompt}
>load povray;
\end{eulerprompt}
\begin{eulercomment}
Pastikan, direktori bin Povray dalam jalur. Jika tidak atur mengikuti
variabel, jadi ini mengandung jalur ke executable povray.
\end{eulercomment}
\begin{eulerprompt}
>defaultpovray="C:\(\backslash\)Program Files\(\backslash\)POV-Ray\(\backslash\)v3.7\(\backslash\)bin\(\backslash\)pvengine.exe"
\end{eulerprompt}
\begin{euleroutput}
  C:\(\backslash\)Program Files\(\backslash\)POV-Ray\(\backslash\)v3.7\(\backslash\)bin\(\backslash\)pvengine.exe
\end{euleroutput}
\begin{eulercomment}
Untuk pertama kali pemakaian, kami mem-plotkan fungsi sederhana.
Perintah berikut membuat sebuah file povray dalam direktori pengguna
kamu, dan menjalankan Povray untuk penelusuran sinar dalam file.

Jika kamu memulai perintah berikut, GUI Povray seharusnya terbuka,
menjalankan file, dan menutup secara otomatis. Karena alasan keamanan,
kamu akan diminta, untuk memberikan izin untuk menjalankan file exe.
Kamu dapat mengklik cancel untuk memberhentikan pertanyaan
selanjutnya. Kamu mungkin akan mengklik OK dalam jendela Povray untuk
menyetujui percakapan awal dari Povray.
\end{eulercomment}
\begin{eulerprompt}
>plot3d("x^2+y^2",zoom=2):
\end{eulerprompt}
\eulerimg{27}{images/Hamemayu Hayuningrat_23030630053_EMTPlot3D-056.png}
\begin{eulerprompt}
>pov3d("x^2+y^2",zoom=3);
\end{eulerprompt}
\eulerimg{27}{images/Hamemayu Hayuningrat_23030630053_EMTPlot3D-057.png}
\begin{eulercomment}
Kami dapat membuat fungsi transparan dan menambahkan pada hasil akhir
lainnya. Kami juga dapat menambahkan garis level ke fungsi plot.
\end{eulercomment}
\begin{eulerprompt}
>pov3d("x^2+y^3",axiscolor=red,angle=-45°,>anaglyph, ...
>  look=povlook(cyan,0.2),level=-1:0.5:1,zoom=3.8);
\end{eulerprompt}
\eulerimg{27}{images/Hamemayu Hayuningrat_23030630053_EMTPlot3D-058.png}
\begin{eulercomment}
Terkadang ini dibutuhkan untuk mencegah penskalaan fungsi dan skala
fungsi dengan tangan.

Kami memplot himpunan titik-titik dalam bidang kompleks, dimana hasil
dari jarak ke 1 dan -1 sama dengan 1.
\end{eulercomment}
\begin{eulerprompt}
>pov3d("((x-1)^2+y^2)*((x+1)^2+y^2)/40",r=2, ...
>  angle=-120°,level=1/40,dlevel=0.005,light=[-1,1,1],height=10°,n=50, ...
>  <fscale,zoom=3.8);
\end{eulerprompt}
\eulerimg{27}{images/Hamemayu Hayuningrat_23030630053_EMTPlot3D-059.png}
\eulerheading{Membuat Plot dengan Koordinat}
\begin{eulercomment}
Selain fungsi, kami dapat mem-plot dengan koordinat. Seperti dalam
plot3d, kami membutuhkan tiga matiks untuk mendefinisikan objek.

Dalam contoh kami memutar mengelilingi sumbu-z.
\end{eulercomment}
\begin{eulerprompt}
>function f(x) := x^3-x+1; ...
>x=-1:0.01:1; t=linspace(0,2pi,50)'; ...
>Z=x; X=cos(t)*f(x); Y=sin(t)*f(x); ...
>pov3d(X,Y,Z,angle=40°,look=povlook(red,0.1),height=50°,axis=0,zoom=4,light=[10,5,15]);
\end{eulerprompt}
\eulerimg{27}{images/Hamemayu Hayuningrat_23030630053_EMTPlot3D-060.png}
\begin{eulercomment}
Dalam contoh berikut, kami membuat plot glombang yang mengecil. Kami
membuat gelombang dengan bahasa matriks Euler.

Kami juga menunjukkan tambahan objek yang dapat ditambahkan ke layar
pov3d. Untuk pembuatan objek, lihat contoh berikut. Catatan bahwa
plot3d menskalakan plot, jadi ini akan menyesuaikan ke unit kubus.
\end{eulercomment}
\begin{eulerprompt}
>r=linspace(0,1,80); phi=linspace(0,2pi,80)'; ...
>x=r*cos(phi); y=r*sin(phi); z=exp(-5*r)*cos(8*pi*r)/3;  ...
>pov3d(x,y,z,zoom=6,axis=0,height=30°,add=povsphere([0.5,0,0.25],0.15,povlook(red)), ...
>  w=500,h=300);
\end{eulerprompt}
\eulerimg{16}{images/Hamemayu Hayuningrat_23030630053_EMTPlot3D-061.png}
\begin{eulercomment}
Dengan metode bayangan lebih lanjut dari Povray, sangat sedikit titik
yang dapat menghasilkan permukaan yang sangat halus. Hanya pada saat
batasan dan dalam bayangan trik ini mungkin akan terlihat jelas.

Untuk ini, kami membutuhkan vektor normal dalam setiap titik matriks.
\end{eulercomment}
\begin{eulerprompt}
>Z &= x^2*y^3
\end{eulerprompt}
\begin{euleroutput}
  
                                   2  3
                                  x  y
  
\end{euleroutput}
\begin{eulercomment}
Persamaan dari permukaan adalah [x, y, Z]. Kami menghitung dua turunan
ke x dan y dari ini dan mengambil hasil kali silang sebagai normal.
\end{eulercomment}
\begin{eulerprompt}
>dx &= diff([x,y,Z],x); dy &= diff([x,y,Z],y);
\end{eulerprompt}
\begin{eulercomment}
Kami mendefinisikan normal sebagai perkalian produk dari turunan ini
dan mendefinisikan fungsi koordinat.
\end{eulercomment}
\begin{eulerprompt}
>N &= crossproduct(dx,dy); NX &= N[1]; NY &= N[2]; NZ &= N[3]; N,
\end{eulerprompt}
\begin{euleroutput}
  
                                 3       2  2
                         [- 2 x y , - 3 x  y , 1]
  
\end{euleroutput}
\begin{eulercomment}
Kami hanya menggunakan 25 titik.
\end{eulercomment}
\begin{eulerprompt}
>x=-1:0.5:1; y=x';
>pov3d(x,y,Z(x,y),angle=10°, ...
>  xv=NX(x,y),yv=NY(x,y),zv=NZ(x,y),<shadow);
\end{eulerprompt}
\eulerimg{27}{images/Hamemayu Hayuningrat_23030630053_EMTPlot3D-062.png}
\begin{eulercomment}
Berikut merupakan knot Trefoil dilakukan dengan A. Busser dalam
Povray. Ini merupakan versi yang telah diperbarui dalam contoh ini.

Contoh: Examples\textbackslash{}Trefoil Knot \textbar{} Trefoil Knot

Untuk pemandangan yang bags dengan tidak beberapa titik, kami
menambahkan vektor normal disini. KAmi menggunakan Maxima untuk
menghitung normal untuk kami. Pertama, tiga fungsi untuk koordinat
sebagai ekspresi simbolik.
\end{eulercomment}
\begin{eulerprompt}
>X &= ((4+sin(3*y))+cos(x))*cos(2*y); ...
>Y &= ((4+sin(3*y))+cos(x))*sin(2*y); ...
>Z &= sin(x)+2*cos(3*y);
\end{eulerprompt}
\begin{eulercomment}
Lalu dua vektor turunan ke x dan y.
\end{eulercomment}
\begin{eulerprompt}
>dx &= diff([X,Y,Z],x); dy &= diff([X,Y,Z],y);
\end{eulerprompt}
\begin{eulercomment}
Sekarang normal, yang mana adalah hasil kali dari dua turunan.
\end{eulercomment}
\begin{eulerprompt}
>dn &= crossproduct(dx,dy);
\end{eulerprompt}
\begin{eulercomment}
Kami sekarang menilai semua ini secara numerik.
\end{eulercomment}
\begin{eulerprompt}
>x:=linspace(-%pi,%pi,40); y:=linspace(-%pi,%pi,100)';
\end{eulerprompt}
\begin{eulercomment}
Vektor-vektor normal merupakan penilaian dari ekspresi simbolik dn[i]
untuk i=1,2,3. Sintaks untuk ini adalah \&"ekspresi"(parameter). Ini
merupakan sebuah alternatif dari metode dalam contoh sebelumnya, kami
dapat mendefinisikan ekspresi simbolic NX, NY, NZ terlebih dahulu.
\end{eulercomment}
\begin{eulerprompt}
>pov3d(X(x,y),Y(x,y),Z(x,y),>anaglyph,axis=0,zoom=5,w=450,h=350, ...
>  <shadow,look=povlook(blue), ...
>  xv=&"dn[1]"(x,y), yv=&"dn[2]"(x,y), zv=&"dn[3]"(x,y));
\end{eulerprompt}
\eulerimg{21}{images/Hamemayu Hayuningrat_23030630053_EMTPlot3D-063.png}
\begin{eulercomment}
Kami juga dapat menghasilkan jaring-jaring dalam 3D.
\end{eulercomment}
\begin{eulerprompt}
>povstart(zoom=4); ...
>x=-1:0.5:1; r=1-(x+1)^2/6; ...
>t=(0°:30°:360°)'; y=r*cos(t); z=r*sin(t); ...
>writeln(povgrid(x,y,z,d=0.02,dballs=0.05)); ...
>povend();
\end{eulerprompt}
\eulerimg{27}{images/Hamemayu Hayuningrat_23030630053_EMTPlot3D-064.png}
\begin{eulercomment}
Dengan povgrid(), kurva-kurva jadi memungkinkan.
\end{eulercomment}
\begin{eulerprompt}
>povstart(center=[0,0,1],zoom=3.6); ...
>t=linspace(0,2,1000); r=exp(-t); ...
>x=cos(2*pi*10*t)*r; y=sin(2*pi*10*t)*r; z=t; ...
>writeln(povgrid(x,y,z,povlook(red))); ...
>writeAxis(0,2,axis=3); ...
>povend();
\end{eulerprompt}
\eulerimg{27}{images/Hamemayu Hayuningrat_23030630053_EMTPlot3D-065.png}
\eulerheading{Objek Povray}
\begin{eulercomment}
Diatas, kami menggunakan pov3d untuk mem-plot permukaan. Antarmka
povray dalam Euler dapat membuat objek Povray. Objek-objek ini
disimpan sebagai string dalam Euler, dan perlu untuk menulis ke file
Povray.

Kami mulai mengeluarkan dengan povstart().
\end{eulercomment}
\begin{eulerprompt}
>povstart(zoom=4);
\end{eulerprompt}
\begin{eulercomment}
Pertama kami definisikan tiga silinder dan menyimpanya ke string dalam
Euler.

Fungsi-fungsi povx() dan lainnya secara sederhana mengembalikan vektor
[1, 0, 0], yang mana dapat digunakan sebagai gantinya.
\end{eulercomment}
\begin{eulerprompt}
>c1=povcylinder(-povx,povx,1,povlook(red)); ...
>c2=povcylinder(-povy,povy,1,povlook(yellow)); ...
>c3=povcylinder(-povz,povz,1,povlook(blue)); ...
\end{eulerprompt}
\begin{eulercomment}
The strings contain Povray code, which we need not understand at that
point.
\end{eulercomment}
\begin{eulerprompt}
>c2
\end{eulerprompt}
\begin{euleroutput}
  cylinder \{ <0,0,-1>, <0,0,1>, 1
   texture \{ pigment \{ color rgb <0.941176,0.941176,0.392157> \}  \} 
   finish \{ ambient 0.2 \} 
   \}
\end{euleroutput}
\begin{eulercomment}
Seperti yang kamu lihat, kami menammbahkan tekstur ke objek dalam tiga
warna yang berbeda.

Ini dapat dilakukan dengan povlook(), yang mana mengembalikan sebuah
stirng dengan code Povray yang relevan. Kami dapat menggunakan warna
bawaan Euler, atau mendefinisikan sendiri warna kami. Kami jua dapat
menambahkan transparansi, atau mengubah cahaya sekitar.
\end{eulercomment}
\begin{eulerprompt}
>povlook(rgb(0.1,0.2,0.3),0.1,0.5)
\end{eulerprompt}
\begin{euleroutput}
   texture \{ pigment \{ color rgbf <0.101961,0.2,0.301961,0.1> \}  \} 
   finish \{ ambient 0.5 \} 
  
\end{euleroutput}
\begin{eulercomment}
Sekarang kami mendefinisikan objek yang saling memotong dan menulis
hasilnya kedalam file.
\end{eulercomment}
\begin{eulerprompt}
>writeln(povintersection([c1,c2,c3]));
\end{eulerprompt}
\begin{eulercomment}
Potongan dari silinder tiga sangat sulit untuk digambarkan, jika kamu
belum pernah melihatnya.
\end{eulercomment}
\begin{eulerprompt}
>povend;
\end{eulerprompt}
\eulerimg{27}{images/Hamemayu Hayuningrat_23030630053_EMTPlot3D-066.png}
\begin{eulercomment}
Fungsi-fungsi berikut membuat sebuah fraktal secara rekursif.

Fungsi pertama menunjukkan, bagaimana Euler menangani objek Povray
sederhana. povbox() fungsi mengembalikan sebuah string, yang mana
memuat koordinat-koordinat boks, tekstur dan selesai.
\end{eulercomment}
\begin{eulerprompt}
>function onebox(x,y,z,d) := povbox([x,y,z],[x+d,y+d,z+d],povlook());
>function fractal (x,y,z,h,n) ...
\end{eulerprompt}
\begin{eulerudf}
   if n==1 then writeln(onebox(x,y,z,h));
   else
     h=h/3;
     fractal(x,y,z,h,n-1);
     fractal(x+2*h,y,z,h,n-1);
     fractal(x,y+2*h,z,h,n-1);
     fractal(x,y,z+2*h,h,n-1);
     fractal(x+2*h,y+2*h,z,h,n-1);
     fractal(x+2*h,y,z+2*h,h,n-1);
     fractal(x,y+2*h,z+2*h,h,n-1);
     fractal(x+2*h,y+2*h,z+2*h,h,n-1);
     fractal(x+h,y+h,z+h,h,n-1);
   endif;
  endfunction
\end{eulerudf}
\begin{eulerprompt}
>povstart(fade=10,<shadow);
>fractal(-1,-1,-1,2,4);
>povend();
\end{eulerprompt}
\eulerimg{27}{images/Hamemayu Hayuningrat_23030630053_EMTPlot3D-067.png}
\begin{eulercomment}
Perbedaan memungkinkan pemotongan satu objek dari objek lainnya.
Seperti perpotongan, ini merupakan bagian dari objek CSG dari Povray.
\end{eulercomment}
\begin{eulerprompt}
>povstart(light=[5,-5,5],fade=10);
\end{eulerprompt}
\begin{eulercomment}
Untuk demonstrasi ini, kami mendefinisikan sebuah objek dalam Povray,
daripada menggunakan sebuah string dalam Euler. Definisi ditulis ke
file secara langsung.

Sebuah koordinat dimensi tiga dari -1 berarti [-1, -1, -1].
\end{eulercomment}
\begin{eulerprompt}
>povdefine("mycube",povbox(-1,1));
\end{eulerprompt}
\begin{eulercomment}
Kami menggunakan objek ini dengan povobject(), yang mana mengembalikan
sebuah string seperti biasanya.
\end{eulercomment}
\begin{eulerprompt}
>c1=povobject("mycube",povlook(red));
\end{eulerprompt}
\begin{eulercomment}
Kami membuat sebuah kubus kedua dan merotasi dan men-skalakan sedikit.
\end{eulercomment}
\begin{eulerprompt}
>c2=povobject("mycube",povlook(yellow),translate=[1,1,1], ...
>  rotate=xrotate(10°)+yrotate(10°), scale=1.2);
\end{eulerprompt}
\begin{eulercomment}
Lalu kami ambil selisih kedua objek tersebut.
\end{eulercomment}
\begin{eulerprompt}
>writeln(povdifference(c1,c2));
\end{eulerprompt}
\begin{eulercomment}
Sekarang tambahkan tiga sumbu.
\end{eulercomment}
\begin{eulerprompt}
>writeAxis(-1.2,1.2,axis=1); ...
>writeAxis(-1.2,1.2,axis=2); ...
>writeAxis(-1.2,1.2,axis=4); ...
>povend();
\end{eulerprompt}
\eulerimg{27}{images/Hamemayu Hayuningrat_23030630053_EMTPlot3D-068.png}
\eulerheading{Fungsi-Fungsi Implisit}
\begin{eulercomment}
Povray dapat mem-plot himpunan yang mana f(x, y, z) = 0, seperti
parameter implisit dalam plot3d. Namun, hasilnya terlihat lebih baik.

Sintaks untuk fungsi-fungsi sedikit berbeda. Kamu tidak dapat
menggunakan keluaran dari Maxima atau ekspresi Euler.

\end{eulercomment}
\begin{eulerformula}
\[
((x^2+y^2-c^2)^2+(z^2-1)^2)*((y^2+z^2-c^2)^2+(x^2-1)^2)*((z^2+x^2-c^2)^2+(y^2-1)^2)=d
\]
\end{eulerformula}
\begin{eulerprompt}
>povstart(angle=70°,height=50°,zoom=4);
>c=0.1; d=0.1; ...
>writeln(povsurface("(pow(pow(x,2)+pow(y,2)-pow(c,2),2)+pow(pow(z,2)-1,2))*(pow(pow(y,2)+pow(z,2)-pow(c,2),2)+pow(pow(x,2)-1,2))*(pow(pow(z,2)+pow(x,2)-pow(c,2),2)+pow(pow(y,2)-1,2))-d",povlook(red))); ...
>povend();
\end{eulerprompt}
\begin{euleroutput}
  Error : Povray error!
  
  Error generated by error() command
  
  povray:
      error("Povray error!");
  Try "trace errors" to inspect local variables after errors.
  povend:
      povray(file,w,h,aspect,exit); 
\end{euleroutput}
\begin{eulerprompt}
>povstart(angle=25°,height=10°); 
>writeln(povsurface("pow(x,2)+pow(y,2)*pow(z,2)-1",povlook(blue),povbox(-2,2,"")));
>povend();
\end{eulerprompt}
\eulerimg{27}{images/Hamemayu Hayuningrat_23030630053_EMTPlot3D-069.png}
\begin{eulerprompt}
>povstart(angle=70°,height=50°,zoom=4);
\end{eulerprompt}
\begin{eulercomment}
Buat permukaan implisit. Catat sintaks yang berdeda dari ekspresi.
\end{eulercomment}
\begin{eulerprompt}
>writeln(povsurface("pow(x,2)*y-pow(y,3)-pow(z,2)",povlook(green))); ...
>writeAxes(); ...
>povend();
\end{eulerprompt}
\eulerimg{27}{images/Hamemayu Hayuningrat_23030630053_EMTPlot3D-070.png}
\eulerheading{Objek Mesh}
\begin{eulercomment}
Contoh ini, kami menunjukkan bagaimana untuk membuat objek mesh, dan
menggambarnya dengan informasi tambahan.

Kami ingin memaksimalkan xt dibawah kondisi x + y = 1 dan
mendemonstrasikan sentuan tangential dari garis level.
\end{eulercomment}
\begin{eulerprompt}
>povstart(angle=-10°,center=[0.5,0.5,0.5],zoom=7);
\end{eulerprompt}
\begin{eulercomment}
Kami tidak dapat menyimpan objek dalam string seperti sebelumnya,
karena terlalu besar. Jadi kami mendefinisikan objek dalam file Povray
menggunakan #declare. Fungsi povtriangle() melakukan ini secara
otomatis. Itu dapat menerima vektor normal seperti pov3d().

Dibawah ini mendefinisikan objek mesh dan menulisnya langsung kedalam
file.
\end{eulercomment}
\begin{eulerprompt}
>x=0:0.02:1; y=x'; z=x*y; vx=-y; vy=-x; vz=1;
>mesh=povtriangles(x,y,z,"",vx,vy,vz);
\end{eulerprompt}
\begin{eulercomment}
Sekarang kami mendefinisikan dua cakram, yang mana akan dipotong
dengan permukaan.
\end{eulercomment}
\begin{eulerprompt}
>cl=povdisc([0.5,0.5,0],[1,1,0],2); ...
>ll=povdisc([0,0,1/4],[0,0,1],2);
\end{eulerprompt}
\begin{eulercomment}
Tulis pemukaan minus dua cakram.
\end{eulercomment}
\begin{eulerprompt}
>writeln(povdifference(mesh,povunion([cl,ll]),povlook(green)));
\end{eulerprompt}
\begin{eulercomment}
Tulis potongan keduanya.
\end{eulercomment}
\begin{eulerprompt}
>writeln(povintersection([mesh,cl],povlook(red))); ...
>writeln(povintersection([mesh,ll],povlook(gray)));
\end{eulerprompt}
\begin{eulercomment}
Tulis maksimal titik.
\end{eulercomment}
\begin{eulerprompt}
>writeln(povpoint([1/2,1/2,1/4],povlook(gray),size=2*defaultpointsize));
\end{eulerprompt}
\begin{eulercomment}
Tambahkan sumbu dan selesai.
\end{eulercomment}
\begin{eulerprompt}
>writeAxes(0,1,0,1,0,1,d=0.015); ...
>povend();
\end{eulerprompt}
\eulerimg{27}{images/Hamemayu Hayuningrat_23030630053_EMTPlot3D-071.png}
\eulerheading{Anaglyphs dalam Povray}
\begin{eulercomment}
Untuk membuat sebuah anaglyph untuk kacamata merah/cyan, Povray harus
menjalankan dua kamera dengan posisi yang berbeda. Ini akan
menghasilkan dua Povray file dan dua file PNG, yang mana dimuat dalam
fungsi loadanaglyph().

Tentu saja, kamu membutuhkan kacamata merah/cyan untuk melihat contoh
berikut secara benar.

Fungsi pov3d() memiliki penggantian sederhana untuk membuat anaglyphs.
\end{eulercomment}
\begin{eulerprompt}
>pov3d("-exp(-x^2-y^2)/2",r=2,height=45°,>anaglyph, ...
>  center=[0,0,0.5],zoom=3.5);
\end{eulerprompt}
\eulerimg{27}{images/Hamemayu Hayuningrat_23030630053_EMTPlot3D-072.png}
\begin{eulercomment}
Jika kamu membuat sebuah pemandangan dengan objek, kamu perlu untuk
menempatkan pembuatan pemandangan ke dalama suatu fungsi dan jalankan
keduannya dengan nilai berbeda untuk parameter anaglyph.
\end{eulercomment}
\begin{eulerprompt}
>function myscene ...
\end{eulerprompt}
\begin{eulerudf}
    s=povsphere(povc,1);
    cl=povcylinder(-povz,povz,0.5);
    clx=povobject(cl,rotate=xrotate(90°));
    cly=povobject(cl,rotate=yrotate(90°));
    c=povbox([-1,-1,0],1);
    un=povunion([cl,clx,cly,c]);
    obj=povdifference(s,un,povlook(red));
    writeln(obj);
    writeAxes();
  endfunction
\end{eulerudf}
\begin{eulercomment}
Fungsi povanglyph() akan menjalankan ini semua. Parameter seperti
povstart() dan povend() dikombinasikan.
\end{eulercomment}
\begin{eulerprompt}
>povanaglyph("myscene",zoom=4.5);
\end{eulerprompt}
\eulerimg{27}{images/Hamemayu Hayuningrat_23030630053_EMTPlot3D-073.png}
\eulerheading{Mendefinisikan Objek milik sendiri}
\begin{eulercomment}
Tampilan povray dari Euler memiliki banyak objek. Tetapi kamu tidak
dibatasi olehnya. Kamu dapat membuat objekmu tersendiri, yang mana
mengombinasikan objek yang lainnya, atau secara penuh objek baru.

Kami mendemonstrasikan sebuah torys. Perintah Povray untuk ini adalah
"torus". Jadi kami mengembalikan sebuah string dengan perintah ini dan
parameternya. Catatan bahwa torus selalu berada ditengan tepat pada
titik asal.
\end{eulercomment}
\begin{eulerprompt}
>function povdonat (r1,r2,look="") ...
\end{eulerprompt}
\begin{eulerudf}
    return "torus \{"+r1+","+r2+look+"\}";
  endfunction
\end{eulerudf}
\begin{eulercomment}
Ini merupakan torus pertama kami.
\end{eulercomment}
\begin{eulerprompt}
>t1=povdonat(0.8,0.2)
\end{eulerprompt}
\begin{euleroutput}
  torus \{0.8,0.2\}
\end{euleroutput}
\begin{eulercomment}
Andaikan kami menggunakan objek ini untuk membuat sebuah torus kedua,
mentranslatikan dan merotasinya.
\end{eulercomment}
\begin{eulerprompt}
>t2=povobject(t1,rotate=xrotate(90°),translate=[0.8,0,0])
\end{eulerprompt}
\begin{euleroutput}
  object \{ torus \{0.8,0.2\}
   rotate 90 *x 
   translate <0.8,0,0>
   \}
\end{euleroutput}
\begin{eulercomment}
Sekarang kita menempatkan objek kedalam sebuah tempat. Untuk
tampilannya, kami menggunakan Phong Shading.
\end{eulercomment}
\begin{eulerprompt}
>povstart(center=[0.4,0,0],angle=0°,zoom=3.8,aspect=1.5); ...
>writeln(povobject(t1,povlook(green,phong=1))); ...
>writeln(povobject(t2,povlook(green,phong=1))); ...
\end{eulerprompt}
\begin{eulerttcomment}
 >povend();
\end{eulerttcomment}
\begin{eulercomment}
memanggil program Povray. Namun, dalam kasus error, ini tidak akan
menampilkan error. Kamu harus melakukan

\end{eulercomment}
\begin{eulerttcomment}
 >povend(<exit);
\end{eulerttcomment}
\begin{eulercomment}

jika apapun tidak bekerja. Ini akan meniggalkan jendela Povray
terbuka.
\end{eulercomment}
\begin{eulerprompt}
>povend(h=320,w=480);
\end{eulerprompt}
\eulerimg{18}{images/Hamemayu Hayuningrat_23030630053_EMTPlot3D-074.png}
\begin{eulercomment}
Ini merupakan contoh yang lebih elaborasi. Kami menyelesaikan

\end{eulercomment}
\begin{eulerformula}
\[
Ax \le b, \quad x \ge 0, \quad c.x \to \text{Max.}
\]
\end{eulerformula}
\begin{eulercomment}
dan melihatkan titik-titik yang mungkin dan optimal dalam plot 3D.
\end{eulercomment}
\begin{eulerprompt}
>A=[10,8,4;5,6,8;6,3,2;9,5,6];
>b=[10,10,10,10]';
>c=[1,1,1];
\end{eulerprompt}
\begin{eulercomment}
Pertama, mari kita cek, jika contoh ini memiliki sebuah solusi pada
semuanya.
\end{eulercomment}
\begin{eulerprompt}
>x=simplex(A,b,c,>max,>check)'
\end{eulerprompt}
\begin{euleroutput}
  [0,  1,  0.5]
\end{euleroutput}
\begin{eulercomment}
Ya, itu mempunyainya.

Selanjutnya kami mendefinisikan dua objek. Pertama merupakan sebuah
bidang

\end{eulercomment}
\begin{eulerformula}
\[
a \cdot x \le b
\]
\end{eulerformula}
\begin{eulerprompt}
>function oneplane (a,b,look="") ...
\end{eulerprompt}
\begin{eulerudf}
    return povplane(a,b,look)
  endfunction
\end{eulerudf}
\begin{eulercomment}
Lalu kami mendefinisikan perpotongan dari pertengahan bidang semua dan
sebuah kubus.
\end{eulercomment}
\begin{eulerprompt}
>function adm (A, b, r, look="") ...
\end{eulerprompt}
\begin{eulerudf}
    ol=[];
    loop 1 to rows(A); ol=ol|oneplane(A[#],b[#]); end;
    ol=ol|povbox([0,0,0],[r,r,r]);
    return povintersection(ol,look);
  endfunction
\end{eulerudf}
\begin{eulercomment}
Sekarang kami dapat membuat plot hasilnya.
\end{eulercomment}
\begin{eulerprompt}
>povstart(angle=120°,center=[0.5,0.5,0.5],zoom=3.5); ...
>writeln(adm(A,b,2,povlook(green,0.4))); ...
>writeAxes(0,1.3,0,1.6,0,1.5); ...
\end{eulerprompt}
\begin{eulercomment}
Berikut merupakan sebuah lingkaran di sekitar optimum.
\end{eulercomment}
\begin{eulerprompt}
>writeln(povintersection([povsphere(x,0.5),povplane(c,c.x')], ...
>  povlook(red,0.9)));
\end{eulerprompt}
\begin{eulercomment}
Dan sebuah error dalam arah dari optimum.
\end{eulercomment}
\begin{eulerprompt}
>writeln(povarrow(x,c*0.5,povlook(red)));
\end{eulerprompt}
\begin{eulercomment}
Kami tambahkan teks ke layar. Teks hanya sebuah objek 3D. Kami butuh
tempat untuk membalikkannya mengikuti pandangan kita.
\end{eulercomment}
\begin{eulerprompt}
>writeln(povtext("Linear Problem",[0,0.2,1.3],size=0.05,rotate=5°)); ...
>povend();
\end{eulerprompt}
\eulerimg{27}{images/Hamemayu Hayuningrat_23030630053_EMTPlot3D-075.png}
\eulerheading{Contoh Lainnya}
\begin{eulercomment}
Kamu dapat menemukan contoh lainnya untuk Povray dalam Euler dengan
file berikut.

See: Examples/Dandelin Spheres\\
See: Examples/Donat Math\\
See: Examples/Trefoil Knot\\
See: Examples/Optimization by Affine Scaling
\end{eulercomment}
\end{eulernotebook}
\end{document}
